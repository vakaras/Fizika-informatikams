\chapter{Mechanika}

\section{Materija ir jos savybės.}

Viskas, kas mus supa, tiek gyva ir negyva, yra vadinama materija.
Materija gali egzistuoti vienoje iš dviejų formų: medžiaga, arba
jėgų laukas. Jėgų laukai gali būti:
\begin{itemize}
  \item gravitacinis;
  \item elektromagnetinis (kai kurie fizikai skirsto į elektrinį
    ir magnetinį);
  \item branduolinis (sąveika tarp nuklonų atomo branduolyje).
\end{itemize}

Viena iš pagrindinių materijos egzistavimo formų yra judėjimas.
(Bet koks materialaus objekto kitimas erdvėje ir laike yra vadinamas
jo judėjimu.) Erdvė nusako kūnų išsidėstymą vienas kito atžvilgiu
bei jų dydį. Laikas nusako įvykių seką bei jo santykinę trukmę.
Materijos judėjimo formos:
\begin{itemize}
  \item mechaninė;
  \item šiluminė;
  \item elektromagnetinė;
  \item branduolinė;
  \item virsmų (pavyzdžiui, elementariųjų dalelių virsmai).
\end{itemize}
Jos charakterizuoja įvarius procesus medžiagose, kurie, pavyzdžiui,
gali būti fizikiniai, cheminiai, biologiniai.

Nusakydami kūnų judėjimo greitį, naudojame šviesos sklidimo vakuume
skalę:
\begin{equation}
  c = 2,998 \cdot 10^{8} \frac{m}{s} \approx 3 \cdot 10^{8} \frac{m}{s}.
  \label{const:c}
\end{equation}
Fundamentali, šios reikšmės savybė yra tai, kad šviesos greitis yra
ribinis bet kokiam materialiam taškui, todėl ši vertė yra vadinama
universaliąją verte arba konstanta. Jeigu
$\left( \frac{v}{c} \right)^{2} \ll 1$, kur $v$ yra materialaus objekto
greitis, tai to objekto judėjimas laikomas nereliatyviuoju.
Priešingu atveju – reliatyviuoju. Šie judėjimai kokybiškai skiriasi.

Pagrindinis mechanikos mokslo tikslas yra nagrinėti kūnų judėjimo
erdvėje dėsningumus. Mechanika yra skirstoma į dvi dalis: klasikinę
ir kvantinę, o kiekviena iš šių dalių dar atitinkamai į reliatyvistinę
ir nereliatyvistinę. Nereliatyvistinės mechanikos dėsningumus gauname
iš reliatyvistinės, riboje kai
$\left( \frac{v}{c} \right)^{2} \ll 1$.

Klasikinių dėsningumų galiojimo ribas nusako Planko konstanta:
\begin{equation}
  \hbar = 1,0549 \cdot 10^{-34} J \cdot s.
  \label{const:hbar}
\end{equation}
Kartais yra naudojama kita išraiška:
\begin{equation}
  h = 2\pi\hbar = 6,6256 \cdot 10^{-34} J \cdot s.
  \label{const:h}
\end{equation}
Makroskopiniai kūnai juda pagal klasikinės mechanikos dėsningumus,
o mikroskopiniai (kurių dydžiai artimi atomų linijiniams dydžiams –
$\sim 10^{-10}$ metro) – kvantiniais. Todėl šalia konstantos $c$, mums
irgi yra svarbi ir universalioji konstanta $\hbar$, kuri atskiria
kvantinės ir klasikinės fizikos dėsningumus.

Judėjimas yra laikomas klasikiniu, jei $mvr \ll \hbar$ ir kvantinis,
jei $mvr \cong \hbar$. Pavyzdžiui, jeigu imsime elektroną vandenilio
atome ($r \approx 10^{-10} m, m_{e} = 9,109 \cdot 10^{-31} kg,
v = 10^{6} \frac{m}{s}$), tai judėjimas bus kvantinis, nes
($mvr = 9,109 \cdot 10^{-31} \cdot 10^{-10} \cdot 10^{6}
\cong 10^{-34} J \cdot s$). Iš kitos pusės, kadangi laikas ir
erdvė visuomet nagrinėjami atžvilgiu kokios nors tai atskaitos sistemos,
tai šie dydžiai visuomet yra reliatyvūs.

Atskaitos sistemos, kuriose laisvieji kūnai juda tiesiaeigiai ir
tolygiai vadinamos inercinėmis. Visi fizikos dėsniai tokiose atskaitos
sistemose turi tuos pačius pavidalus. Kitaip tariant, visos
inercinės sistemos fizikiniu požiūriu yra vienos kitoms ekvivalenčios,
o fizikiniai dėsniai pereinant iš vienos į kitą – invariantiniai.

Nagrinėdami realius kūnus mechanikos kurse, koncentruosime savo mintis,
naudodami materialaus taško bei absoliučiai kieto kūno sąvokas.
Materialiu tašku vadinama dalelė, kuriai judant, jos dydis, forma,
struktūra bei vykstantys joje procesai nedaro įtakos judėjimui.
Absoliučiai kietu kūnu vadiname kūną, kuriam judant, atskiros jo
dalys yra nejudančios viena kitos atžvilgiu.

\section{Vienetų sistemos.}

\subsection{Fundamentaliosios konstantos}

\xtable{
  w [2 | 1 | 2]
  a [p | p | p]
  h [Dydis | Simbolis | Reikšmė]
  e [Avogadro skaičius | $N_{A}$ | $6,02214199 \cdot 10^{23} mol^{-1}$]
}

TODO: Suvesti iš 6 skaidrės.

\subsection{Naudingi fizikiniai duomenys}

TODO: Suvesti iš 6 skaidrės.

\subsection{Dydžių konvertavimas}

TODO: Suvesti iš 7 skaidrės.

\subsection{Priešdėliai skirti pažymėti 10 laipsniui}

TODO: Suvesti iš 7 skaidrės.

\subsection{Pagrindinės matematinės formulės}

TODO: Suvesti iš 7 skaidrės.

\section{Pagrindinės judančiųjų kūnų charakteristikos (greitis, pagreitis).}

Jeigu yra žinoma, kokiu būdu laike kinta kūnų padėtis, tai yra sakoma,
kad tokio judėjimo dėsnis yra nustatytas. Materialaus taško padėtį
erdvėje nusako radius-vektorius. Radius vektorius yra vektorius, kuris
turi ir modulį ir kryptį.

TODO: Brėžinys. (1 pav. iš 8 skaidrės.)

Trimatės erdvės atveju galima sakyti, kad taškas turi 3 laisvės
laipsnius. Jeigu sistema turi $N$ dalelių, o kiekviena gali
judėti viena kitos atžvilgiu, tai tokia sistema turi $3N$ laisvės
laipsnius. Iš čia seka, kad laisvės laipsniu galime vadinti
minimalų parametrų, kurie nusako taško padėtį erdvėje, skaičių.

Judant kūnui, jo radius-vektorius kinta, tai yra:
\begin{equation*}
  \vec{r} = \vec{r}(t) = \left\{ x(t), y(t), z(t) \right\}.
\end{equation*}
Dydis, charakterizuojantis kūnų padėties kitimą laike, vadinamas
greičiu. Jis yra apibrėžiamas, kaip radius-vektoriaus išvestinė:
\begin{align*}
  \vec{v}(t)
    &= \lim_{\Delta t \to 0}
      \frac{\vec{r}(t + \Delta t) - \vec{r}(t)}{\Delta t}
    = \lim_{\Delta t \to 0} \frac{\Delta \vec{r}}{\Delta t}
    = \frac{d \vec{r}}{dt}; \\
  v_{x}(t) &= \frac{dx(t)}{dt}; \\
  v_{y}(t) &= \frac{dy(t)}{dt}; \\
  v_{z}(t) &= \frac{dz(t)}{dt}; \\
\end{align*}
Jo kryptis sutampa su pačio radius-vektoriaus kryptimi.

TODO: Brėžinys. (2 pav. iš 9 skaidrės.)

Judėdami kūnai tuo pačiu metu gali ir suktis, todėl apibrėžiame
kampinį greitį:
\begin{equation*}
  \vec{\omega}(t) = \frac{d\vec{\varphi}}{dt},
\end{equation*}
čia $\varphi$ yra posūkio kampas.

TODO: Brėžinys. (3 pav. iš 10 skaidrės.)

Liestinio ir kampinio greičių sąryšis:
\begin{align*}
  \vec{v}
  &= \frac{dr}{dt} \\
  &= \frac{\left[ d\vec{\varphi} \vec{r} \right]}{dt} \\
  &= \left[ \frac{d\vec{\varphi}}{dt}r \right] \\
  &= \left[ \vec{\omega}\vec{r} \right] \\
\end{align*}
Taip pat, jei $f$ pažymėsime dažnį (matuojamas hercais), o $T$ – periodą
(matuojamas sekundėmis), tai taip pat yra teisingos lygybės:
\begin{align*}
  \omega &= 2 \pi f \\
  \omega &= \frac{2 \pi}{T} \\
\end{align*}

Jei kūnas juda netolygiai, tai sakome, kad kūnas juda su pagreičiu:
\begin{align*}
  \vec{a}
  &= \frac{d\vec{v}}{dt} \\
  &= \frac{d^{2}\vec{r}}{dt^{2}} \\
  \vec{a} &= \left\{ a_{x}, a_{y}, a_{z} \right\} \\
  a_{x}(t) &= \frac{dv_{x}(t)}{dt} = \frac{d^{2}x(t)}{dt^{2}} \\
  a_{y}(t) &= \frac{dv_{y}(t)}{dt} = \frac{d^{2}y(t)}{dt^{2}} \\
  a_{z}(t) &= \frac{dv_{z}(t)}{dt} = \frac{d^{2}z(t)}{dt^{2}} \\
\end{align*}
Esant kampu į horizontą mesto kūno judėjimui, kūno pagreitis
turi du sandus\footnote{sánd|as – sudedamoji dalis, dėmuo (DLKŽ)}:
tangentinį $a_{t}$ ir normalinį $a_{n}$. ($a_{t}$ yra $\vec{a}$ projekcija
į tiesę išvestą per $\vec{v}$, o $a_{n}$ – į jai statmeną.) Kitaip
tariant:
\begin{equation*}
  \vec{a} = a_{t} \vec{n} + a_{n} \vec{m},
\end{equation*}
čia $\vec{n}$ ir $\vec{m}$ yra statmeni vienetiniai vektoriai.

TODO: Brėžinys. (4 pav. iš 11 skaidrės.)

Apskritimu vienodu kampiniu greičiu judantis kūnas turi įcentrinį
pagreitį:
\begin{equation*}
  a_{\t{įc}} = \frac{v^{2}}{R},
\end{equation*}
čia $v$ – linijinis greitis, o $R$ – apskritimo spindulys.

Apskritimu judančio kūno kampinis pagreitis:
\begin{align*}
  \vec{\varepsilon}
  &= \frac{d\vec{\omega}}{dt} \\
  &= \frac{d}{dt} \left( \frac{d\vec{\varphi}}{dt} \right) \\
  &= \frac{d^{2}\vec{\varphi}}{dt^{2}} \\
\end{align*}

\begin{remember}
  \item Kas tai yra kūno greitis, kuo jis charakterizuojamas (trimatėje
    sistemoje) – radius-vektoriaus kitimo sparta tam tikra išreikšta
    kryptimi; jei yra sukamasis judėjimas, tai įvedama kampinio greičio
    sąvoka.
  \item Jeigu turime netolygų judėjimą, tai tam turime dar vieną parametrą
    – pagreitį (apskritiminiam – kampinis pagreitis).
  \item Jei yra kūnas mestas kampu į horizontą, tai pagreitis susideda
    iš tangentinio ir normaliojo.
\end{remember}

\section{Galilėjaus transformacijos.}

Daugelyje atvejų tenka analizuoti fizikinius reiškinius keliose
inercinėse atskaitos sistemose. (Kitaip tariant, turime klasikinį
judėjimo modelį.) Norint tai atlikti, visų pirma, reikia nustatyti
vienos sistemos inertiškumą, po to parinkti kitą sistemą taip, kad
joje galima būtų apibūdinti:
\begin{itemize}
  \item laiko atskaitos pradžią;
  \item koordinačių pradžią;
  \item koordinačių orientaciją;
  \item judančios sistemos greitį.
\end{itemize}

Tarkime, jog turime dvi koordinačių sistemas: $K$ ir $K'$. Jeigu tartume,
kad laikas abiejose sistemose skiriasi dydžiu $t_{0}$, o $K'$ koordinačių
pradžia yra paslinkta per radius-vektorių $\vec{r_{0}}$ sistemos
$K$ atžvilgiu, tai $K$ sistemos dydžius su $K'$ sistemos dydžiais
sieja tokie sąryšiai:
\begin{align*}
  t' &= t + t_{0} \\
  \vec{r}' &= \vec{r} + \vec{r_{0}} \\
\end{align*}

TODO: 6 brėžinys iš 12 skaidrės.

Jei sistemą $K'$ dar pasuktume kampu $\varphi$, pavyzdžiui,
plokštumoje $xy$, tai gautume, jog:
\begin{align*}
  x &= x'\cos \varphi - y' \sin \varphi \\
  y &= x'\sin \varphi + y' \cos \varphi \\
\end{align*}

TODO: 7 brėžinys iš 13 skaidrės.

Jeigu sistema $K'$ yra judanti, tai esant sąlygai
$\left( \frac{v}{c} \right)^{2} \ll 1$, abiejose sistemose
$K$ ir $K'$ laikas slenka vienodai: $t = t'$.

Jeigu sistema $K'$ juda kryptimi $x$, tai
\begin{align*}
  x' &= x - vt \\
  y' &= y \\
  z' &= z \\
\end{align*}
Šie sąryšiai yra vadinami \emph{Galilėjaus transformacijomis}.
Galilėjaus transformacijos galioja nereliatyvistinės mechanikos rėmuose.
Iš Galilėjaus transformacijų seka, kad:
\begin{enumerate}
  \item nereliatyvistinio judėjimo atveju kūnų matmenys nekinta;
  \item $K$ sistemos atžvilgiu atstumas tarp taškų vienoje ir kitoje
    sistemoje gali būti išreikštas:
    \begin{equation*}
      l_{KK'} = \sqrt{(x'-x)^{2} + (y'-y)^{2} + (z'-z)^{2}};
    \end{equation*}
  \item jeigu taško judėjimo greitis vienoje sistemoje yra $v$, o
    kitoje – $v'$ bei sistema $K'$ juda greičiu $\bar{v}$ sistemos
    $K$ atžvilgiu, tai
    \begin{align*}
      \vec{v} &= \vec{v'} + \vec{\bar{v}} \\
      v_{x} &= v'_{x} + \bar{v_{x}} \\
      v_{y} &= v'_{y} + \bar{v_{y}} \\
      v_{z} &= v'_{z} + \bar{v_{z}} \\
      t &= t' \\
    \end{align*}
  \item visi kūnai kitų inercinių sistemų atžvilgiu juda vienodais
    pagreičiais:
    \begin{equation*}
      \frac{dv}{dt} = \frac{dv'}{dt} = \frac{dv'}{dt'}
    \end{equation*}
  \item šviesos greitis įvairiose inercinėse atskaitos sistemose turi tą
    pačią $c$ reikšmę, kurią eksperimentais nustatė A. Maikelsonas ir
    E. Morli 1880-1887 metais.
\end{enumerate}

\section{Lorenco transformacijos.}

Jei turim reliatyvistinį judėjimo modelį. Bus paaiškinti laiko ir
geometrijos kitimas.

Lorenco transformacijos yra taikomos tada, kai turime reliatyvistinį
judėjimą, kai turime kvantinės mechanikos atvejį.

Jei $\frac{v^{2}}{c^{2}} \approx 1$, tai judėjimas $K'$ atžvilgiu
nejudančios $K$ sistemos yra reliatyvistinis.

Reikia prisiminti: esant tokiam judėjimui kinta laikas ir geometrija.
Svarbiausia:
\begin{itemize}
  \item tinka aprašyti judėjimus, kai $\frac{v^{2}}{c^{2}} \approx 1$;
  \item Lorenco transformacijos duoda sampratą apie geometrijos
    kitimą esant reliatyvistiniam judėjimui (judančioje sistemoje
    kūnų geometrija mažėja atžvilgiu nejudančios sistemos);
  \item judančioje sistemoje laikas sutrumpėja nejudančios sistemos.
\end{itemize}

\section{Dinaminiai mechanikos principai.}
Niutono dėsniai. Gravitacija. Energija (kinetinė, potencinė).

Bet kokios N dalelių sistemos kiekvienos dalelės pagreitis yra jų
radius-vektorių bei greičių funkcija:
\begin{align}
  \vec{a_{i}}
  &= \frac{d^{2}\vec{r}_{i}}{dt^{2}} \\
  &= \vec{a}_{i}(%
    \vec{r_{1}},\ldots,\vec{r_{N}}; \vec{v_{1}},\ldots,\vec{v_{N}}) \\
  &= \vec{a_{i}} \left(%
    \vec{r_{1}},\ldots,\vec{r_{N}}; %
    \frac{dr_{1}}{dt},\cdots,\frac{dr_{N}}{dt} \right) \\
  \label{eq:klasikines_mechanikos_judejimo_lygis}
\end{align},
čia $i = 1,2,\ldots,N$.

\ref{eq:klasikines_mechanikos_judejimo_lygis} lygtis yra vadinama
klasikinės mechanikos judėjimo lygtimi. Bet kurios dalelės:
\begin{align*}
  \vec{a}
  &= \frac{d^{2}\vec{r}}{dt^{2}} \\
  &= \vec{a}(\vec{r}, \vec{v}) \\
  &= \vec{a} (\vec{r}, \frac{d\vec{r}}{dt}) \\
\end{align*}

Impulsas: $\vec{p} = m\vec{v}$.


$v_{1}$ ir $v_{2}$ – greičiai prieš susiduriant, o
$v_{1}'$ ir $v_{2}'$ – greičiai po susidūrimo:
\begin{align*}
  m_{1}v_{1} + m_{2}v_{2} &= m_{1}v_{1}' + m_{2}v_{2}' \\
  m_{1}(v_{1}-v_{1}') &= m_{2}(v_{2}' - v_{2}) \\
\end{align*}

I-asis Niutono dėsnis: $\vec{a} = 0$ ir $F = 0$.

III-asis Niutono dėsnis: $\vec{F_{12}} = - \vec{F_{21}}$.
\begin{equation*}
  \frac{d\vec{P_{i}}}{dt}
  = F_{i} (\vec{r_{1}},\ldots,\vec{r_{N}}; \vec{v_{1},\ldots,\vec{v_{N}}})
\end{equation*}

Jėgos, veikdamos kūnus, verčia juos judėti, tai yra atlieka darbą.
Pavyzdžiui:
\begin{align*}
  dA = \vec{F}d\vec{r}'\\
  \Delta A = \int dA = \int \vec{F}d\vec{r}' \\
  \Delta A= F \Delta r \cos \alpha
\end{align*}


\begin{align*}
  \Delta A = U(\vec{r_{0}}) - U(\vec{r}) \\
  U = mgh \\
\end{align*}

U – potencinė energija. Kinetinė energija: $E_{k} = \frac{mv^{2}}{2}$.

Svarbu:
\begin{itemize}
  \item nuo ko priklauso pagreitis (funkcija radius vektoriaus ir greičio);
  \item jei sistemoje yra N dalelių, tai pagreičiai atitinkamai būtų
    nusakomi, kaip funkcijos dalelių radius…;
  \item sąryšis tarp impulso, masės ir greičio;
  \item uždaros sistemos impulsų sistema yra konstanta;
  \item jėgos ir impulso sąryšis;
  \item antrasis Niutono dėsnis (pagreičio apibrėžimas);
  \item pirmasis Niutono dėsnis;
  \item trečiasis Niutono dėsnis;
  \item gravitacijos jėgos išraiška;
  \item ryšys tarp darbo ir jėgos;
  \item ryšys tarp darbo ir potencinės energijos.
\end{itemize}

\section{Tvermės dėsniai.}

Trys tvermės dėsniai:
\begin{itemize}
  \item impulso;
  \item energijos;
  \item impulso momento.
\end{itemize}

\begin{equation*}
  U(x_{1} + x_{0}; x_{2}+x_{0}) = U(x_{1},x_{2}).
\end{equation*}

$F_{21}$ yra vadinama reakcijos jėga (jėgai $F_{12}$).

Reliatyvumo judėjimo atveju:
\begin{equation*}
  \vec{p} = \frac{m\vec{v}}{\sqrt{1 - (\frac{v}{c})^{2}}}
\end{equation*}

\begin{note}
  Kartais žymima: $\left( \frac{v}{c} \right)^{2} = \beta$.
\end{note}

Turint N dalelių sistemą, jos energetinę būseną galime nusakyti:
\begin{equation*}
  E = E(\vec{r_{1}},\ldots,\vec{r_{N}}; \vec{p_{1},\ldots,\vec{p_{N}}}) = H
\end{equation*}
Tokia funkcija yra vadinama Hamiltono funkcija ir žymima H.

Svarbu prisiminti:
\begin{itemize}
  \item energijos ryšys su impulsu;
  \item Hamiltono lygtys;
  \item sąryšis tarp …
  \item reikia mokėti išvesti energijos tvermės dėsnį iš Hamiltono
    lygčių.
\end{itemize}

TODO: Rimties energija ($m*c^{2}$).

\subsection{Impulso momento tvermės dėsnis}

Impulso momentas: $\vec{L} = \left[ \vec{r}\vec{p} \right]$.
Izoliuotam kūnui: $\vec{L} = const.$

Tai seka iš:
\begin{equation*}
  \frac{d\vec{L}}{dt} = \left[ \frac{d\vec{r}}{dt} \vec{p} \right] +
  \left[ \vec{r} \frac{d\vec{p}}{dt} \right]
\end{equation*}

Kadangi greitis $\frac{d\vec{r}}{dt}$ ir $\vec{p}$ kolinearūs, tai,
o be to, $\frac{d\vec{L}}{dt} = 0$ ir $\vec{L}$ yra $const$.
Jei dalelę veikia jėgos, tai atsiranda dalelę veikiantis jėgų
momentas:
\begin{equation*}
  \vec{M} = \left[ \vec{r}\vec{F} \right]
\end{equation*}
