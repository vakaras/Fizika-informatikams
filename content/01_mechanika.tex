\chapter{Mechanika}

\section{Materija ir jos savybės.}

\begin{defn}[Materija]
  Viskas, kas mus supa. Tiek gyva ir negyva.
\end{defn}

Materija gali būti dviejose formose:
\begin{description}
  \item[medžiaga] – ;
  \item[jėgų laukas] – gravitacinis, elektromagnetinis (kai kurie skirsto
    į elektrinį ir magnetinį), branduolinis (sąveika tarp nuklonų
    atomo branduolyje).
\end{description}

Judėjimas betarpiškai siejasi su kinetine energija. Temperatūra yra
dalelių vidutinė kinetinė energija. Prie absoliutaus nulio, bet koks
judėjimas nustoja egzistuoti.

Nusakydami kūnų judėjimo greitį naudojame šviesos sklidimo vakuume skalę.
Šviesos greitis yra laikomas greičio įmanomo greičio viršutiniu 
rėžiu.
\begin{equation*}
  c = 2,998 \cdot 10^{8} \frac{m}{s} \approx 3 \cdot 10 ^{8} \frac{m}{s}.
\end{equation*}

Jei $\frac{v^{2}}{c^{2}} << 1$, tai judėjimas laikomas nereliatyviuoju,
priešingu atveju – reliatyviuoju.

Mechanikos mokslo tikslas yra nagrinėti kūnų judėjimo erdvėje
dėsningumus.

Mechanika yra skirstoma į:
\begin{description}
  \item[klasikinę] –;
  \item[kvantinę] –;
\end{description}

Planko konstanta (TODO: perbraukta h):
\begin{equation*}
  h_{\t{Planko}} = 1,0549 \cdot 10 ^{-34} J \cdot s
\end{equation*}
Kartais naudojama:
\begin{equation*}
  h = 6,6256 \cdot 10 ^{-34} J \cdot s
\end{equation*}
Ryšys:
\begin{equation*}
  h_{\t{Planko}} = \frac{h}{2 \pi}
\end{equation*}


\section{Vienetų sistemos.}

\section{Pagrindinės judančiųjų kūnų charakteristikos (greitis, pagreitis).}

\begin{defn}[Greitis]
  Radius-vektoriaus kitimo sparta.
\end{defn}

Kampinis greitis:
\begin{align*}
  \vec{v}
  &= \frac{dr}{dt} \\
  &= \frac{\left[ d\vec{\varphi} \vec{r} \right]}{dt} \\
  &= \left[ \frac{d\vec{\varphi}}{dt}r \right] \\
  &= \left[ \vec{\omega}\vec{r} \right] \\
\end{align*}

Pagreitis:
\begin{align*}
  \vec{a} = a_{t} \vec{n} + a_{n} \vec{m},
\end{align*}
čia $\vec{n}$ ir $\vec{m}$ ortogonalūs (statmeni) vienetiniai vektoriai. 

Jeigu kūnas juda apskritimu, tada kampinis pagreitis:
\begin{align*}
  \vec{\varepsilon}
  &= \frac{d\vec{\omega}}{t} \\
  &= \frac{d}{dt}\left( \frac{d\vec{\varphi}}{dt} \right) \\
  &= \frac{d^{2} \vec{\varphi}}{dt^{2}} \\
\end{align*}

Svarbiausi klausimai:
\begin{itemize}
  \item kas tai yra kūno greitis, kuo jis charakterizuojamas (trimatėje
    sistemoje) – radius-vektoriaus kitimo sparta tam tikra išreikšta
    kryptimi; jei yra sukamasis judėjimas, tai įvedama kampinio greičio
    sąvoka;
  \item jeigu turime netolygų judėjimą, tai tam turime dar vieną parametrą
    – pagreitį (apskritiminiam – kampinis pagreitis);
  \item jei yra kūnas mestas kampu į horizontą, tai pagreitis susideda
    iš tangentinio ir normaliojo.
\end{itemize}

\section{Galilėjaus transformacijos.}

Jei turim klasikinį judėjimo modelį.

Galilėjaus transformacijos – metodas aprašyti makroskopinių kūnų
judėjimą.

Tarkime, turime dvi judančias sistemas $K$ ir $K'$. Jeigu paimtume
pokytį skalėje dydžiu $t_{0}$ ir koordinačių pradžią paslinktume
per radius vektorių $\vec{r_{0}}$, tai laikas TODO: patikrinti formulių
vektorius.


Jeigu taško judėjimo greitis vienoje sistemoje $V$…:
\begin{align*}
  \vec{V} = \vec{V}' + v;
\end{align*}

Reikia prisiminti, kad $\frac{v^{2}}{c^{2}} << 1$! Nei laikas, nei
geometrija nekinta.

\section{Lorenco transformacijos.}

Jei turim reliatyvistinį judėjimo modelį. Bus paaiškinti laiko ir
geometrijos kitimas.

Lorenco transformacijos yra taikomos tada, kai turime reliatyvistinį
judėjimą, kai turime kvantinės mechanikos atvejį.

Jei $\frac{v^{2}}{c^{2}} \approx 1$, tai judėjimas $K'$ atžvilgiu
nejudančios $K$ sistemos yra reliatyvistinis.

Reikia prisiminti: esant tokiam judėjimui kinta laikas ir geometrija.
Svarbiausia:
\begin{itemize}
  \item tinka aprašyti judėjimus, kai $\frac{v^{2}}{c^{2}} \approx 1$;
  \item Lorenco transformacijos duoda sampratą apie geometrijos
    kitimą esant reliatyvistiniam judėjimui (judančioje sistemoje
    kūnų geometrija mažėja atžvilgiu nejudančios sistemos);
  \item judančioje sistemoje laikas sutrumpėja nejudančios sistemos.
\end{itemize}

\section{Dinaminiai mechanikos principai.}
Niutono dėsniai. Gravitacija. Energija (kinetinė, potencinė).

Bet kokios N dalelių sistemos kiekvienos dalelės pagreitis yra jų
radius-vektorių bei greičių funkcija:
\begin{align}
  \vec{a_{i}}
  &= \frac{d^{2}\vec{r}_{i}}{dt^{2}} \\
  &= \vec{a}_{i}(%
    \vec{r_{1}},\ldots,\vec{r_{N}}; \vec{v_{1}},\ldots,\vec{v_{N}}) \\
  &= \vec{a_{i}} \left(%
    \vec{r_{1}},\ldots,\vec{r_{N}}; %
    \frac{dr_{1}}{dt},\cdots,\frac{dr_{N}}{dt} \right) \\
  \label{eq:klasikines_mechanikos_judejimo_lygis}
\end{align},
čia $i = 1,2,\ldots,N$.

\ref{eq:klasikines_mechanikos_judejimo_lygis} lygtis yra vadinama
klasikinės mechanikos judėjimo lygtimi. Bet kurios dalelės:
\begin{align*}
  \vec{a}
  &= \frac{d^{2}\vec{r}}{dt^{2}} \\
  &= \vec{a}(\vec{r}, \vec{v}) \\
  &= \vec{a} (\vec{r}, \frac{d\vec{r}}{dt}) \\
\end{align*}

Impulsas: $\vec{p} = m\vec{v}$.


$v_{1}$ ir $v_{2}$ – greičiai prieš susiduriant, o
$v_{1}'$ ir $v_{2}'$ – greičiai po susidūrimo:
\begin{align*}
  m_{1}v_{1} + m_{2}v_{2} &= m_{1}v_{1}' + m_{2}v_{2}' \\
  m_{1}(v_{1}-v_{1}') &= m_{2}(v_{2}' - v_{2}) \\
\end{align*}

I-asis Niutono dėsnis: $\vec{a} = 0$ ir $F = 0$.

III-asis Niutono dėsnis: $\vec{F_{12}} = - \vec{F_{21}}$.
\begin{equation*}
  \frac{d\vec{P_{i}}}{dt}
  = F_{i} (\vec{r_{1}},\ldots,\vec{r_{N}}; \vec{v_{1},\ldots,\vec{v_{N}}})
\end{equation*}

Jėgos, veikdamos kūnus, verčia juos judėti, tai yra atlieka darbą.
Pavyzdžiui:
\begin{align*}
  dA = \vec{F}d\vec{r}'\\
  \Delta A = \int dA = \int \vec{F}d\vec{r}' \\
  \Delta A= F \Delta r \cos \alpha
\end{align*}


\begin{align*}
  \Delta A = U(\vec{r_{0}}) - U(\vec{r}) \\
  U = mgh \\
\end{align*}

U – potencinė energija. Kinetinė energija: $E_{k} = \frac{mv^{2}}{2}$.

Svarbu:
\begin{itemize}
  \item nuo ko priklauso pagreitis (funkcija radius vektoriaus ir greičio);
  \item jei sistemoje yra N dalelių, tai pagreičiai atitinkamai būtų
    nusakomi, kaip funkcijos dalelių radius…;
  \item sąryšis tarp impulso, masės ir greičio;
  \item uždaros sistemos impulsų sistema yra konstanta;
  \item jėgos ir impulso sąryšis;
  \item antrasis Niutono dėsnis (pagreičio apibrėžimas);
  \item pirmasis Niutono dėsnis;
  \item trečiasis Niutono dėsnis;
  \item gravitacijos jėgos išraiška;
  \item ryšys tarp darbo ir jėgos;
  \item ryšys tarp darbo ir potencinės energijos.
\end{itemize}

\section{Tvermės dėsniai.}

Trys tvermės dėsniai:
\begin{itemize}
  \item impulso;
  \item energijos;
  \item impulso momento.
\end{itemize}

\begin{equation*}
  U(x_{1} + x_{0}; x_{2}+x_{0}) = U(x_{1},x_{2}).
\end{equation*}

$F_{21}$ yra vadinama reakcijos jėga (jėgai $F_{12}$).

Reliatyvumo judėjimo atveju:
\begin{equation*}
  \vec{p} = \frac{m\vec{v}}{\sqrt{1 - (\frac{v}{c})^{2}}}
\end{equation*}

\begin{note}
  Kartais žymima: $\left( \frac{v}{c} \right)^{2} = \beta$.
\end{note}

Turint N dalelių sistemą, jos energetinę būseną galime nusakyti:
\begin{equation*}
  E = E(\vec{r_{1}},\ldots,\vec{r_{N}}; \vec{p_{1},\ldots,\vec{p_{N}}}) = H
\end{equation*}
Tokia funkcija yra vadinama Hamiltono funkcija ir žymima H.

Svarbu prisiminti:
\begin{itemize}
  \item energijos ryšys su impulsu;
  \item Hamiltono lygtys;
  \item sąryšis tarp …
  \item reikia mokėti išvesti energijos tvermės dėsnį iš Hamiltono
    lygčių.
\end{itemize}

TODO: Rimties energija ($m*c^{2}$).

\subsection{Impulso momento tvermės dėsnis}

Impulso momentas: $\vec{L} = \left[ \vec{r}\vec{p} \right]$.
Izoliuotam kūnui: $\vec{L} = const.$

Tai seka iš:
\begin{equation*}
  \frac{d\vec{L}}{dt} = \left[ \frac{d\vec{r}}{dt} \vec{p} \right] +
  \left[ \vec{r} \frac{d\vec{p}}{dt} \right]
\end{equation*}

Kadangi greitis $\frac{d\vec{r}}{dt}$ ir $\vec{p}$ kolinearūs, tai,
o be to, $\frac{d\vec{L}}{dt} = 0$ ir $\vec{L}$ yra $const$.
Jei dalelę veikia jėgos, tai atsiranda dalelę veikiantis jėgų
momentas:
\begin{equation*}
  \vec{M} = \left[ \vec{r}\vec{F} \right]
\end{equation*}
