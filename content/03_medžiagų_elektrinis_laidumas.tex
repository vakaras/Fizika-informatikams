\chapter{Medžiagų elektrinis laidumas.}
\section{Metalai elektriniame lauke.}
Eksperimentai, kurie parodo, kad pagrindiniai krūvininkai metaluose yra
elektronai. (Rygaki, Lorenco, Tompsono…)
\section{Puslaidininkai ir dielektrikai elektriniame lauke.}
Laidumo analizė naudojantis zoniniais (juostiniais) modeliais.
Savieji ir priemaišiniai puslaidininkai.

Dielektrikai: poliniai ir centrosimetriniai. Elementari kristalinė
gardelė neturi simetrijos centro. Pagrindiniai parametrai: dielektrinė
skvarba (poliarizacijos matas), poliarizacija – fizikinis parametras,
surištas su dielektrine skvarba. Dielektrinė skvarba – kompleksinis dydis
(reali dalis nusako laukų santykį medžiogoje atžvilgiu vakuumo, menama
dalis nusako nuostolių kampą, kuris yra surištas su laidumu.)
\section{Feroelektrikai.}
Kristalinės medžiagos, kurių gardelės tam tikroje temperatūroje neturi
simetrijos centro. Juose vyksta faziniai virsmai, kurie gali būti
dvejopi: pirmos rūšies (laisvosios energijos pirmosios išvestinės
kinta šuoliškai temperatūros Kuri taške) ir antros rūšies (termodinamo
antrosios išvestinės Kuri-Vero taške kinta šuoliškai). Tai yra
aktualu kuriant elektriniu lauku valdomus kondensatorius.
\section{Diamagnetikai, paramegnetikai, feromagnetikai magnetiniame lauke.}
Feromagnetikų visi netiesiniai parametrų kitimai yra valdomi magnetiniais
laukais. Nagrinėsime, kaip kinta magnetinio lauko indukcija, nuo
lauko stiprio. Kaip valdoma magnetinė skvarba magnetiniais laukais.
Histerizės reiškinys. Kam tinka praktikoje feromagnetikai. Iš
feromagnetinės fazės į … Kuri, Lažavenas. Diamagnetikai – medžiagos, kurių
magnetinis momentas yra 0. (Pavyzdžiui, oras.)
\section{Superjonikai.}
Medžiagos, kuriose pagrindiniai krūvininkai yra katijonai ir medžiagos
anijonai. Masės ir elektrinio krūvio perneša. Elektroninis laidumas yra
žymiai mažesnis už joninį. Superjonikais tas medžiagas vadina fizikai,
o kietojo kūno specialistai juos vadina kietaisiais elektrolitais.
Neišeina aprašyti Faradėjaus dėsniais. Yra naudojamos … taškinio teorijos.
\section{Superlaidininkai.}
Skirstomi į dvi grupes:
\begin{itemize}
  \item žematemperatūriai – fazinis virsmas į superlaidžią būsena virsta
    4-6 K temperatūroje;
  \item aukštatemperatūriai – skysto azoto temperatūroje.
\end{itemize}
Meisnerio efektas – iš medžiagos yra išstumiamas magnetinis laukas.
(Jei patalpinsim medžiagą virš magneto ir sukelsim superlaidumą,
tai …)
