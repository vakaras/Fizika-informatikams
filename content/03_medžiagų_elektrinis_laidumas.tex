\chapter{Medžiagų elektrinis laidumas.}
\section{Metalai elektriniame lauke.}

1897 metais Tomsonas atrado elektroną.

Rikke dirbo jau žinodamas Faradėjaus dėsningumus. (Faradėjus parodė, kad
elektrolituose krūvį perneša jonai.) Jo hipotezė buvo, kad krūvį
irgi perneša jonai. Leido ištisus metus tekėti srovę ir nustatė,
kad masė nepakito. Iš eksperimento jis iškėlė hipotezę, kad krūvį
perneša elektronai.

Savitasis elektrono krūvis:
$\frac{\t{elektrono krūvis}}{\t{elektrono masė}}$.

Drudės modelis: metalai būdami kristalais, jų elementariąją kristalinę
gardelę sudaro katijonai, taigi teigiamai įelektrinti jonai.
Tačiau, tai yra kaipo indas į kurį talpinami elektronai. Kiekvieną
katijoną atitinka elektronas. Tokiu būdu katijonas turėdamas teigiamą
krūvį, elektronas – neigiamą krūvį, šis krūvių konglomeratas yra
subalansuotas ir medžiaga yra neutrali elektriškai.

Elektrinio lauko stiprio vektorius yra nukreiptas nuo …

Kadangi elektronai metaluose yra nuolatiniame judėjime, tai jų
laisvojo lėkio laikas:
\begin{equation*}
  \tau = \frac{\Lambda}{\vec{v}},
\end{equation*}
čia $\Lambda$ – vidutinis laisvasis lėkis.

\begin{equation*}
  <U> = \frac{qE\Lambda}{m \vec{v}}
\end{equation*}

$\sigma$ – laidumas.

Smūgių skaičius: $\frac{1}{\tau} = \frac{V}{\Lambda}$.

\begin{equation*}
  <E_{K}>
  = \frac{m<u>^{2}}{2}
  = \frac{q^{2}\Lambda ^{2}}{2 m \vec{v}^{2}}E^{2}
\end{equation*}

Prisiminti.
\begin{itemize}
  \item Rikki eksperimentinis darbas (elektros leidimas per tris
    gabalus metalo).
  \item Mandelštamo-Papaleksi darbas (strypo sukimas ir išnaudojimas
    inercijos momento).
  \item Stiuarto ir Tomsono darbas, kuriuo parodė, jog krūvio nešėjas
    metaluose yra elektronas.
  \item Drudė paskaičiavo judėjimo greiti ne elektriniame lauke.
    Neutralumo koncepcija. Tvarkingasis krūvininkų judėjimas ir
    jo sąryšis su laidumu.
\end{itemize}

Elektrinio lauko stipris. Žymimas $E$. Jo dimensija:
$\frac{V}{m} \left( \frac{\t{voltas}}{\t{metras}} \right)$.

Laidumas. Žymimas $\sigma$. Jo dimensija:
$\frac{S}{m} \left( \frac{\t{Simensas}}{metras} \right)$.

Elektros srovės stipris. Žymimas $I$. Jo dimensija:
$A = \frac{C}{s} \left( \t{Amperas} = \frac{\t{Kulonas}}{sekundė} \right)$.

Elektros srovės tankis. Žymimas $j$. Jo dimensija:
$\frac{A}{m^{2}} = \frac{\t{Amperas}}{kvadratinis metras}$.

Žymimas $\gamma$. Jo dimensija:
$\Ohm \cdot m$.

Džaulio-lenco dėsnis. (Tekant elektros srovei laidininku, jis šyla.)
\begin{align*}
  \frac{dQ}{dt}
  &= n \frac{1}{\tau}<E_{K}> \\
  &= \frac{n q^{2}\Lambda}{2 m \vec{v}} E^{2} \\
  &= R I^{2} \\
\end{align*}

Prisiminti, kad $\frac{dQ}{dt} = RI^{2}$.

Videmano-Franco dėsnis.

Svarbu.
\begin{itemize}
  \item $\frac{\xi}{\sigma} = 3 \left( \frac{k}{q} \right)^{2}T 
    = 2,23 \cdot 10^{-8}T$.
\end{itemize}

Eksperimentai, kurie parodo, kad pagrindiniai krūvininkai metaluose yra
elektronai. (Rikke, Lorenco, Tomsono…)

\section{Puslaidininkai ir dielektrikai elektriniame lauke.}
Laidumo analizė naudojantis zoniniais (juostiniais) modeliais.
Savieji ir priemaišiniai puslaidininkai.

Dielektrikai: poliniai ir centrosimetriniai. Elementari kristalinė
gardelė neturi simetrijos centro. Pagrindiniai parametrai: dielektrinė
skvarba (poliarizacijos matas), poliarizacija – fizikinis parametras,
surištas su dielektrine skvarba. Dielektrinė skvarba – kompleksinis dydis
(reali dalis nusako laukų santykį medžiogoje atžvilgiu vakuumo, menama
dalis nusako nuostolių kampą, kuris yra surištas su laidumu.)
\section{Feroelektrikai.}
Kristalinės medžiagos, kurių gardelės tam tikroje temperatūroje neturi
simetrijos centro. Juose vyksta faziniai virsmai, kurie gali būti
dvejopi: pirmos rūšies (laisvosios energijos pirmosios išvestinės
kinta šuoliškai temperatūros Kuri taške) ir antros rūšies (termodinamo
antrosios išvestinės Kuri-Vero taške kinta šuoliškai). Tai yra
aktualu kuriant elektriniu lauku valdomus kondensatorius.

\section{Diamagnetikai, paramegnetikai, feromagnetikai magnetiniame lauke.}
Feromagnetikų visi netiesiniai parametrų kitimai yra valdomi magnetiniais
laukais. Nagrinėsime, kaip kinta magnetinio lauko indukcija, nuo
lauko stiprio. Kaip valdoma magnetinė skvarba magnetiniais laukais.
Histerizės reiškinys. Kam tinka praktikoje feromagnetikai. Iš
feromagnetinės fazės į … Kuri, Lažavenas. Diamagnetikai – medžiagos, kurių
magnetinis momentas yra 0. (Pavyzdžiui, oras.)

\subsection{Feromagnetikai}

Feromagnetikų kilomodinis … yra didžiausias lyginant su
paramagnetikais ir diamagnetikais. Feromagnetikai tai yra medžiagos,
kurios priklausomai nuo temperatūros, gali būti vienoje iš dviejų
fazių: feromagnetinėje arba paramagnetinėje fazėje. Tai yra
feromagnetikuose, priklausomai nuo temperatūros, vyksta fazinis virsmas
iš feromagnetinės į paramagnetinę fazę. Feromagnetinėje fazėje
egzistuoja domeninė struktūra. Feromagnetikuose domenai, tai yra
sritys, turinčios tam tikros krypties, magnetinę indukciją.
Magnetinė indukcija yra žymima raide $B$. Sąryšis tarp magnetinės
indukcijos ir magnetinio lauko stiprio yra $B = \mu\mu_{0}H$.
$H$ yra matuojamas $\left[ \frac{A}{m} \right]$, $\mu$ – bedimensinis
dydis, o $\mu_{0} = 4 \pi \cdot 10 ^{-7}\frac{H}{m}$. Taigi
magnetinė indukcija yra matuojama teslomis $T$. Jeigu talpiname
feromagnetiką į magnetinį lauką ir didiname magnetinio lauko stipri,
tai esant tam tikram magnetinio lauko stipriui $H$ visi domenai išsidėstys
lauko kryptimi. Koercinis laukas (žymimas $H_{c}$) – lauko stipris,
kuriam esant orientuojasi visi domenai pagal išorinio lauko kryptį.
Gaunama monodomeninį (?) atvejį. Ši situacija yra įmanoma tik tada,
kai feromagnetikas yra feromagnetinėje fazėje.

% TODO: \mu priklausomybė nuo T (absoliučiosios temperatūros).
$T_{c}$ – temperatūra fazinio virsmo iš feromagnetinės į paramagnetinę
fazę.

% TODO: B priklausomybė nuo T.
Paramagnetinėje fazėje išnykstant domeninei struktūrai indukuotų
sričių magnetinė indukcija sumažėja iki 0.

% TODO: \mu priklausomybė nuo H.
Didėjant magnetinio lauko stipriui, magnetinė skvarba, iki tam tikros
magnetinio lauko stiprio reikšmės $H_{c}$ didėja, o vėliau didėjant
magnetinio lauko stipriui, feromagnetiko magnetinė skvarba mažėja.
(Ši priklausomybė yra galima tik tada, kai feromagnetikas yra 
feromagnetinėje fazėje.)
Priežastis: didėjant lauko stipriui vis daugiau ir daugiau domenų
orientuojasi magnetinio lauko kryptimi ir dėl šitos priežastis
magnetinė indukcija didėja ir savo ruožtu didėja magnetinė
skvarba.
\begin{align*}
  \mu &= \frac{B}{\mu_{0}H} & B = const\t{, kai} H \geq H_{c} \\
\end{align*}
Esant koerciniam laukui $H_{c}$ feromagnetikas tampa monodomeniniu ir
toliau didėjant magnetinio lauko stipriui, indukcija $B$ lieka konstanta.

%TODO: \mu priklausomybė nuo H, kai feromagnetikas yra paramagnetinėje
% fazėje (primena 1/x grafiką).

%TODO: B priklausomybė nuo H, kai talpiname į periodinį lauką.
% (Feromagnetinėje fazėje.)
Ši kreivė yra vadinama Histerezės kilpa.

%TODO: B priklausomybė nuo H, kai talpiname į periodinį lauką.
% (Paramagnetinėje fazėje.)
Paramagnetiko indukcija yra tiesinė jo magnetinio lauko stiprio
funkcija.

Svarbu:
\begin{itemize}
  \item kas yra būdinga feromagnetikams (kad jie priklausomai nuo
    temperatūros gali būti arba feromagnetinėje arba paramagnetinėje
    fazėje);
  \item vyksta fazinis virsmas iš feromagnetinės į paramagnetinę;
  \item domenai
  \item kaip kinta magnetinė skvarba priklausomai nuo T ir H;
  \item kaip kinta H nuo T;
  \item mokėti paaiškinti, kodėl yra tokia $\mu$ priklausomybė nuo H;
  \item kas darosi, jei feromagnetiką talpiname į periodinį magnetinį
    lauką.
\end{itemize}
Dažniausiai pasitaiko tik pirmos rūšies fazinis virsmas. (Bet pridėjus
priemaišų, galima gauti ir antros rūšies fazinė virsmą.)

\subsection{Paramagnetikai}

$C$ – Kiuri konstanta,
$T$ – temperatūra absoliučioje temperatūroje.

Įmagnetėjimas $\vec{I} = n_{0}\vec{p}_{m}L(a)$, čia
\begin{description}
  \item[$n_{0}$] – elektronų koncentracija;
  \item[$L(a)$] – klasikinė Lanžaveno funkcija,
    $L(a) = \cth a - \frac{1}{a}$. $\cth$ – hiperbolinis kotangentas.
\end{description}

Wėberis – magnetinio lauko linijų skaičius kertantis tam tikra ploto
vienetą.

Elektrono magnetinis momentas $p_{m} = \mu_{B}$.

Svarbu:
\begin{itemize}
  \item paramegnetikuose galime indukuoti magnetinį momentą esant
    tam tikriems lauko stipriams;
  \item kaip galima paskaičiuoti kilomolinį jautri remiantis Kuri
    formule;
  \item ką pasiūlė Lanžavenas;
  \item kaip galime surasti įmagnetėjimą, naudojantis dviem sąlygom
    kai magnetinio lauko energija yra žymiai mažesnė už šiluminę
    energiją, ir, kai magnetinio lauko energija yra žymiai didesnė
    už šiluminę energiją.
\end{itemize}

\section{Diamagnetikai}

Magnetikai, kuriuose negalima indukuoti magnetinio momento.

$\Xi_{kmol}$ – kilomagnetinis jautris?

Paprasčiausias iš diamagnetikų pavyzdys yra oras.

Kontroliniam darbe bus po 3 skirtingus klausimus. Bus 5 skirtingos
grupės. Atsinešti rašiklį ir savo smegenis.

\section{Superjonikai.}
Medžiagos, kuriose pagrindiniai krūvininkai yra katijonai ir medžiagos
anijonai. Masės ir elektrinio krūvio perneša. Elektroninis laidumas yra
žymiai mažesnis už joninį. Superjonikais tas medžiagas vadina fizikai,
o kietojo kūno specialistai juos vadina kietaisiais elektrolitais.
Neišeina aprašyti Faradėjaus dėsniais. Yra naudojamos … taškinio teorijos.
\section{Superlaidininkai.}
Skirstomi į dvi grupes:
\begin{itemize}
  \item žematemperatūriai – fazinis virsmas į superlaidžią būsena virsta
    4-6 K temperatūroje;
  \item aukštatemperatūriai – skysto azoto temperatūroje.
\end{itemize}
Meisnerio efektas – iš medžiagos yra išstumiamas magnetinis laukas.
(Jei patalpinsim medžiagą virš magneto ir sukelsim superlaidumą,
tai …)
