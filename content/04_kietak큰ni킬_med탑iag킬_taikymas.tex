\chapter{Kietakūnių medžiagų taikymas.}
\section{Kietakūniai elektros energijos generatoriai.}
Naudojami deguonies vakancijų superjonikai. Deguonės vakancija – ,
neturinti masės.
\section{Kietakūniai elektros energijos akumuliatoriai.}

Visi akumuliatoriai yra charakterizuojami trimis pagrindiniais parametrais:
\begin{description}
  \item[$W$]  – savitoji energetinė talpa
    $\left[ \frac{A \cdot h}{kg} \right]$;
  \item[$t$] – akumuliuoto krūvio saugojimo laikas;
  \item[$n$] – įkrovimų ir iškrovimų skaičius.
\end{description}

Tokiose akumuliatoriuose, kuriuose naudojami vienokie ar kitokie
superjonikai, krūvininkų akumuliacija yra atliekama jeigu sistemoje
vyksta cheminės reakcijos, kurių metu atsiranda elektronai, kurie
ir yra akumuliuojami. Kitu atveju, kai sistemoje viename iš
elektrodų kinta tam tikrų elementų valentingumai. Dėl valentingumų
kaitos, kad būtų išlaikytas krūvininkų išsilaikymo arba tvermės
dėsnis turi atsirasti elektronai, kurie yra akumuliuojami.
Visuomet akumuliatoriuose kadangi yra du elektrodai, tai yra anodai
ir katodai, tai tokiose sistemose vyksta dviejų tipų reakcijos:
anodinė ir katodinė.

Vienas iš pavyzdžių anodinės reakcijos \slide{110}.
Visais atvejais išlaikomas krūvininkų tvermės dėsnis.

Išsiaiškinti:
…
Kad išliktų $\beta$ modifikacija, tai $x$ $Na_{2}O-\beta-x Al_{2}O_{3}$,
$0 \leq x \leq 11$. Jei $x$ yra 11, tai ši medžiaga yra delektrikas,
nes visi Ant…Roso mazgai yra tušti, o elektronai gali keliauti tik
jais. Taigi mus tinka, tik kai $6 \leq x \leq 10$.

Pirmieji akumuliatoriai panaudoti mobiliuose telefonuos buvo nikelio
vandenio junginiai. Tada siekė maždaug $30$. Dabartinių ličio
jonų gali siekti iki $200$. Mes kalbėsime apie kristalinius
ličio, kurių gali siekti iki $400$.

Automobiliuose yra rūgštiniai akumuliatoriuose. Jeigu ant akumuliatoriaus
rašo 60 ampervalandžių, jo svoris 15 kg, tai gausime $W=4$.

Kitas svarbus parametras yra saugojimo laikas. Rūgštinių automobilinių
akumuliatorių saugojimo laikas yra penki metai. Dėl $Ag/I_{2}$, tai
jų trukmė iki 18 metų \slide{111}.

Yra naujos kartos akumuliatoriai, kurių tiek savitoji energetinė
skvarba pranoksta 100 kartų, įkrovimo iškrovimo laikas praktiškai
neribojimas (10000 kartų), akumuliuoto krūvio saugojimo laikas
kelis kartus pranoksta rūgštinius.

Sieros akumuliatorius \slide{112} $Na/S$. Ampulė iš
$Na-\beta-polialiuminato$. S ir C mišinys – sieros ir anglies mišinys.
Kai sumaišome sierą su smulkiagrūde anglimi, tai … plotas padidėja.
Dar naudojama natrio dagtis, tai yra … prisotinta natrio.

Na-$\beta$-polialiuminatas yra absoliutus dielektrikas elektronams.
Todėl įkraunat susidarę elektronai negali nutekėti atgal į natrį.

\slide{112} apatiniame paveikslėlyje $\sigma_{j}$ – joninio laidumo
sandas, $\sigma_{e}$ – elektroninio laidumo sandas.

\slide{114} įkraunant prie viršutinio jungiamas +, o prie apatinio
-.

\begin{remember}
  \item Nors vieną sakinį iš istorinės dalies. Priklausomai nuo to
    kokius elektrodus mes naudojame ir superjoniką, taip ir vadinasi
    akumuliatoriai.
  \item Dėl ko vyksta krūvininkų akumuliacija. Dėl dviejų reakcijų:
    kai vyksta betarpiška cheminė reakcija tarp elektroninių medžiagų,
    kai vyksta cheminių valentingumų kaita viename iš elektrodų.
  \item Kokie trys pagrindiniai parametrai nusako akumuliatorius
    (savitoji energetinė talpa, saugojimo laikas, įkrovimų – iškrovimų
    skaičius).
  \item Pakanka žinoti tik vieną iš akumuliatorių tipų (kokių),
    bet būtina plačiai.
\end{remember}

\section{Dujų jutikliai.}

Deguonies dujų jutikliuose yra naudojami deguonies vakancijų
superjonikai ($V_{O}^{..}$). Ką mes nagrinėjome buvo
$ZrO_{2}$ stabilizuotas 8 mol \% $Y_{2}O_{3}$. Priklausomai nuo
to kiek mol procentų įvedame, tai kiekvienas mol procentas skatina
pusės deguonies vakancijos atsiradimą. Jei 4 mol \% , tai …
O jau vakancijomis gali judėti deguonies antijonai ir tam tikrose
… gali kurti Nersto elektrovaros jėgą. \slide{115} yra Nersto
elektrovaros jėgos išraiška. $F$ – Faradėjaus skaičius, $T$ – temperatūra,
$R$ – universalioji dujų konstanta, $p'_{O_{2}}$ – dalinis deguonies
slėgis (etaloninės), $p'_{O_{2}}$ – dalinis deguonies slėgis dujose,
kuriose norime nustatyti deguonies dalį. \slide{115} deguonies jutiklio
paveikslėlis. Kai turime orą, tai $p'_{O_{2}} = 0,21 bar$.
26 – platina (Pt). Eina du platininiai laidai, tarp jų pajungę voltmetrą
matuojame Nersto elektrovaros jėgą. Sugradavus tokį jutiklį pagal žinomus
$p'_{O_{2}}$ galime jį naudoti nežinomiems $p''_{O_2}$.

Deguonies dujų jutikliai naudojami ten, kur vykdomas deginimas. Tokios
konstrukcijos, kuri pateikta \slide{115}, yra naudojami šilumos
tinkluose, taip pat ir Lietuvoje. Kodėl tai yra svarbu: deginant kurą
(oksiduojant jį), jei bus blogai sureguliuota maišyklė, tai į kaminą
išeis daug nesudegusio kuro. Tokie jutikliai buvo pagaminti Lietuvoje,
buvo patikrinti ES ekspertų ir buvo leista juos naudoti šilumos tinkluose.
Jie yra labai ilgaamžiai, jų yra labai lengva deregenaracija: reikia
atkaitinti ore ir jis vėl funkcionuoja, kaip naujas. Anksčiau nebuvo
naudojami jutikliai ir viską spręsdavo kurikas.

\begin{remember}
  \item Kas įeina į konstrukciją (YSZ).
  \item Jo veikimas yra paremtas Nersto elektrovaros jėgos atsiradimu
    esant dujų mišinyje tam tikram daliniam deguonies slėgiui.
  \item Į Nersto formulę įeina …
    Etaloninis dalinis slėgis yra slėgis ore.
\end{remember}

\slide{116} – sočiosios srovės deguonies dujų jutiklis. Yra
matuojamos voltamperinės charakteristikos elemento sudaryto iš dviejų
platinos elektrodų ir YSZ. Priklausomai nuo to kokia yra deguonies
koncentracija tyriamajame mišinyje, didinant įtampą atsiranda sritis,
kur srovė nepriklauso nuo įtampos. Didinant deguonies koncentraciją
mišinyje sočiosios srovės stipris didėja. Taip įsisotinimo sritis
srovės skalėje kyla į viršų. Sugradavus pagal žinoma
koncentraciją srovės sotį, turime galimybę žinoti kokia yra
deguonies dujų koncentracija tam tikrame dujų mišinyje.
Naudojama būtent platinos nes yra aukštatemperatūris metalas ir
jis nesioksiduoja.

\begin{remember}
  \item Žinoti struktūrinę \slide{116} schemą, kaip ji veikia, kaip
    priklauso srovė nuo įtampos, kai keičiam deguonies dujų
    koncentraciją.
\end{remember}

$\Lambda$-jutiklis. Naudojamas automobiliuose. \slide{117}.
Konstrukcijoje yra naudojami YSZ kietieji elektrolitai. Yra vidinis
elektrodas pagamintas iš platinos ir išorinis elektrodas pagamintas iš
platinos. $\Lambda$-jutiklis yra naudojamas, kad optimizuoti oro ir
kuro įpurškimą į deginimo sistemą. Priklausomai nuo to koks yra
įpurškiamas deginti mišinys, atitinkamai kis ta pati Nersto elektrovaros
jėga. Jeigu mes žinome kiek kuro įpurškiame, tai teorinis $\Lambda_{0}$
parodo, kiek reikia oro, kad kuras pilnai sudegtų. Kai $\Lambda$
teorinis sutampa su $\Lambda$ faktiniu, tai turime 1,0. Ampulė
pagaminta iš YSZ, kuri iš viršaus ir išorės padengta platina.

\begin{remember}
  \item $\Lambda_{0}$ kiekvienai kuro markei yra kitoks.
  \item Kas yra faktinė $\Lambda$.
  \item Kaip priklauso elektrovara nuo faktinės $\Lambda$.
\end{remember}

Anglies monoksido dujų jutiklis ($CO$ dujų jutiklis). Anglies
monoksidas dar vadinamas smalkėmis. Konstrukcija parodyta \slide{118}
paveiksle. Į konstrukciją įeina YSZ, platinos elektrodai, viršutinis
elektrodas yra padengtas katalizatoriumi. Katalizatorių pagrindinė
funkcija yra spartinti chemines reakcijas, tačiau nekeičia reakcijos
pobūdžio. Katalizatorius yra pagamintas iš platinos bei $\gamma$ aliuminio
trioksido kubinės modifikacijos. Jei turime aplinką, kurioje yra oras
sumaišytas su smalkėmis, tai esant didelei smalkių koncentracijai,
tai tokiu atveju turime mažą deguonies koncentraciją. Tokiu atveju
viršutinis katalizatorius spartina oksidacija ir $CO \to CO_{2}$.
Tačiau, jeigu mes sugraduosime, kaip kinta deguonies ar dalinis
s… turėsime tikrą vaizdą mišinyje esančių smalkių kiekio.

Visose šituose elementuose yra naudojami YSZ ir platina.

\begin{remember}
  \item Žinoti konstrukcinius ypatumas (naudojamas katalizatorius).
  \item Naudojama Nersto elektrovaros jėga.
\end{remember}

Anglies dioksido ($CO_{2}$) dujų matuoklis. Kam jis reikalingas:
$CO_{2}$ yra kenksmingos gyvam organizmui. Yra naudojami
NASICON (arba Natrio arba Ličio) superjonikai. Vokiečiai savo
pirmam jutiklyje naudojo natrio kietąjį elektrolitą. Vienas elektrodas
yra pagamintas iš natrio karbonato (paprasčiausia soda) ir uždengtas
$BaCO_{3}$ stiklu, o kitas iš platinos ir irgi uždengtas stiklu.
Patalpinus į $CO_{2}$ $NaCO_{3}$ skyla ir Na užima vakanciją.
Prie kito elektrodo niekas nekinta. Taigi atsiranda elektrovaros
jėga: $E = \varphi_{1} - \varphi_{2}$, kur $\varphi$ yra cheminis
potencialas.

\slide{119} pagamintas Lietuvoje $CO_{2}$ dujų jutiklis. Panaudotas
ličio superjonikas. Vienas elektrodas yra platinos, o kitas ličio
karbonatas su platinos elektrodu. Vėl susidaro potencialų skirtumas
patalpinus į $CO_{2}$ mišinį, nes skyla ličio karbonatas ir ličio
jonai užima vakancines vietas ličio superjonike. Sugradavus
galime matuoti $CO_{2}$ koncentracijas nežinomose dujose.

\begin{note}
  Per kontrolinį reikia žinoti tik vieną iš $CO_{2}$ jutiklių.
\end{note}

\begin{remember}
  \item Kokie yra naudojami superjonikai.
  \item Vienur pasikeičia, kitur nepasikeičia.
  \item Labanakt!
\end{remember}

\section{Elektrinės supertalpos (jonistoriai).}

TODO: Kažkiek praleidau.

Prie anglies elektrodo prijungiamas išorinio lauko $+$, o prie sidabro
$-$. Tokiu atveju argentum katijonai, būdami teigiamai įelektrinti
juda link argentum elektrodo, prisijungia elektroną ir tampa
neutraliais atomais, kurie nusėda ant elektrodo. Tokiu atveju, prie
anglies elektrodo susidaro argentum katijonų trūkumas. Taigi ten
lieka jonų vakancijos, kurios yra įelektrintos neigiamai, taigi
ta dalis įsielektrina neigiamai. $d$ lygu apie $1 \t{Å}$. … Tokiu atveju
išauga plotas. Paprastai, jei turime cilindrą ar $1 cm \times 1 cm$
plokštelę, tai dėl rėželių mes $S$ išdidiname iki $5m^{2}$.
$1\t{Å} \equiv 10^{-10} m$. Dabar galime apskaičiuoti talpą:
\begin{equation*}
  C = \frac{16 \cdot 8,85 \cdot 10^{-12}}{10^{-10}}
  \approx 7,08 F
\end{equation*}

Paprastai jonistorius naudoja kardio chirurgai širdies stimuliatoriams.
Jonistoriai yra įsiuvami po oda, jei žmogui pasireiškia širdies aritmija,
tai elektrodai prijungiami prie širdies ir ją sutraukia. Anksčiau
buvo naudojami akumuliatoriai, kuriuos tekdavo nešiotis išorėje.
Kadangi elektroninio laidumu sandas yra labai mažas, tai elektros
nutekėjimas yra mažiausias.

\begin{remember}
  \item Įprastai kodensatorių talpos yra nusakomos jų geometrija.
  \item Prisiminti kaip plokščio kodensatoriaus talpa susijusi su
    jo geometrija.
  \item Kaip veikia jonistorius, kaip jis įkraunas.
  \item Kokie yra geometriniai jonistoriaus parametrai (ploto ir atstumai
    tarp teigiamo ir neigiamo krūvininko jonistoriai).
  \item Su kuo susijusi ne tik talpa, bet ir jonistoriaus nuotekio
    srovė.
\end{remember}

\section{Deguonies langas (deguonies siurblys)}

Deguonies langai yra naudojami tose patalpose, kuriose reikalinga
papildyti deguonimi ir sudaryti normalią atmosferą, kad galėtų
egzistuoti gyvieji organizmai. Paprastai deguonies langai tarnauja
kaip po langai šviesai patekti į patalpą taip ir deguonies papildymui
iki normalios atmosferos (21\% deguonies, kitkas – azotas). Įprastai
deguonies langai yra naudojami tose laboratorijose, kuriose gaminami
vaistai arba medžiagos įvairioms pramonės sritims (švarios medžiagos).
Paprastai langų gamyboje yra naudojamos membranos pagamintos iš
YSZ (Itrio stabilizuoto Cirkoniu, $ZrO_{2}$ 8mol\% $Y_2O_{3}$). Jeigu
ši membrana yra apie dešimties mikronų storio, tai ji yra skaidri
šviesai. Ant membranos yra nusodinama kelių Å storio platinos sluoksnis,
tokiu atveju sistema platinos elektrodas, YSZ ir platinos elektrodas
yra skaidrūs šviesai.

\slide{TODO} Tarkime mums reikia papildyti viršutinę patalpą deguonimi.
Tokiu atveju reikia prijungti + prie viršutinio elektrodo, kuris
yra nukreiptas į patalpos vidų, o - prie atmosferos. Tokiu atveju
išoriniame sluoksnyje deguonis yra jonizuojamas ir deguonies jonai
deguonies vakancijomis juda į vidų, ten gauna elektroną ir tampa
neutraliu atomu.

Papildymo greitis $V = \frac{I}{4F}$, čia $F$ – Faradėjaus skaičius.
Jei $I = 1A$, tai 10 $cm^{2}$  dydžio langelis pompuoja 3 ml / min.
\begin{remember}
  \item Kokiam tikslui yra naudojami.
  \item Struktūrinė schema.
  \item Kodėl yra siurbiamas deguonis iš lauko į vidų.
  \item Kokie maždaug parametrai.
\end{remember}

\section{Elektrolyzeriai.}

Elektrolyzeriai yra naudojami gaminti iš vandens molekulių deguonį
ir vandenilį. Tai yra naudojami H20 molekulių skaidymui. Elektrolyzeriaus
struktūrinė schema parodyta \slide{TODO}, tai yra vamzdis pagamintas
iš YSZ. Jo išorinė ir vidinė pusė (vamzdžio) yra padengta platina
ir prie išorinės ir vidinės pusės yra prijungiamas elektros srovės
šaltinis $E$. Į vamzdį yra purškiami perkaitinti vandens garai.
Jų temperatūra siekia apie 1000 laipsnių Celsijaus. Kitaip sakant
esant tokiai temperatūrai, vandens molekulė yra ant skaidymosi
ribos. Elektrinis laukas stimuliuoja tą skaidymąsi papildomai ir
nukreipia deguonies anijonus iš vidinės vamzdžio pusės į išorinę.
Tai yra, prie vidinės pusės yra pajungiamas neigiamas potencialas,
o prie išorinės – teigiamas. Tokiu atveju jonizuoti deguonies
atomai juda iš vidinės į išorinę elektrolizerio pusę. Išorinėje
elektrolyzerio pusėje deguonis yra surenkamas į rezervuarus, o 
vandenilis juda toliau elektrolizeriu ir jo išėjime irgi yra
surenkamas į rezervuarus. Vidinio vamzdžio pusėje vyksta skaidrės
viršuje pateikta reakcija. Išorėje…

\begin{remember}
  \item Kam skiriamas.
  \item Struktūrinę schemą.
  \item Kaip pajungiamas laukas.
  \item Atsiimti, kad paduodami perkaitinti vandens garai.
\end{remember}

\section{Elektrochrominiai ekranai.}

Tokių ekranų gamybai yra naudojami superjonikai ir yra panaudotas
elektrochrominis efektas. Kvazicheminė reakcija iš \slide{TODO}.

TODO Juostinio modelio brėžinys.
Jei į Wolframą įvedame Litį, tai … Mažėjant draustinės juostos ilgiui,
mes slenkame nuo ultravioletinio prie infraraudonojo. Mes elektriniu
lauku keičiame draustinės juostos plotį ir tuo pačiu keičiame medžiagos
lūžio rodiklį. Vienu poliarumu įvedame litį į medžiagą ir tuo pačiu
mažiname draustinės juostos plotį, o kitu – išvedame litį iš medžiagos
ir tuo pačiu didiname draustinės juostos plotį. Kitaip sakant
su \slide{TODO} pateiktu įrenginiu galime keisti spalvą.

ECD yra naudojami labai dideliems pagal plotį ekranams, kadangi
gamyba yra labai pigu. Sekcijos yra gaminamos tiesiog nudažant.

Techninės charakteristikos:
\begin{enumerate}
  \item spalvų kaitos ciklų skaičius $n\approx 10^{7}$;
  \item spalvų kaitos laikas $t \approx 0,5 s$;
  \item optinio tankio pokytis, tenkantis …
  \item valdo nuo 85 iki 10 \% praeinančios matomos šviesos;
  \item $100 cm^{2}$ ECD naudoja 0,5 mwh elektros energijos.
\end{enumerate}

\begin{remember}
  \item Kas tai yra.
  \item Kad panaudotas elektrochrominis efektas (su elektriniu lauku
    į medžiagą (arba medžiagoje) galime sukurti priemaišinius
    lygmenis, konkrečiai įvedus litį galime keisti nuo violetinės iki
    raudonos).
\item Technines charakteristikas.
\end{remember}

\section{Atminties ląstelės.}

Paprastai integratorių (atminties ląstelių) gamyboje yra naudojami 
argentum superjonikai (TODO iš skaidrės). b paveikslėlyje 2 yra
elektrodai. 4 yra superjonikas. 5 – argentum rezervuaras. 3 – platininis
elektrodas. 1 – izoliatoriai izoliuojantys elektrodus nuo išorinio
korpuso. Pajungus elektrinį lauką prie apatinio elektrinio elektrodo
taip, kad ten būtų $+$, o prie viršutinio – $-$, tai argentum katijonai
per argentum superjoniką migruos į platinos elektrodą, ten prisijungs
elektoną ir nusės ant elektrodo. Kol iš apatinio rezervuaro pasišalins
visi galimi argentum jonai, mes turėsime (jei matuosime potencialą,
prijungtą prie integratoriaus) įrašo kreivę (skaidrėje a), kai
baigsis – potencialas šoktels iki maksimo vertės. Kai įtampa
yra konstanta, tai Būlio algebroje žymi 0, o kitu atveju – 1.

Iš integratorių galime susikonstruoti „ne“, „ir“ bei
„arba“ Būlio algebros veiksmus.

\begin{remember}
  \item Kad naudojami argentum superjonikai.
  \item Struktūrinę schemą.
  \item Jungiant $+$ prie argentum rezervuaro, kol teka srovė, tol yra
    …
  \item Įvairiais jungimo būdais jungdami integratorius galime gauti
    įvairius Būlio algebros veiksmus.
\end{remember}

16 dieną kontrolinis!
