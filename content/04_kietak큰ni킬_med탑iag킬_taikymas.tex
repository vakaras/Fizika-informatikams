\chapter{Kietakūnių medžiagų taikymas.}
\section{Kietakūniai elektros energijos generatoriai.}
Naudojami deguonies vakancijų superjonikai. Deguonės vakancija – , neturinti
masės.
\section{Kietakūniai elektros energijos akumuliatoriai.}
Savitoji energinė talpa (ampervalandžių kiekis). Saugojimo laikas.
Įkrovimo, iškrovimo skaičius.
\section{Dujų jutikliai.}
Deguonies, anglies dioksido, anglies monoksido jutikliai. Deguonies
jutikliai yra svarbus šilumos tinkluose.
\section{Elektrinės supertalpos (jonistoriai).}

\section{Deguonies siurbliai.}
Naudojant superjonikus.
\section{Elektrolyzeriai.}
Tinka skaidyti vandenį į deguonį ir vandenilį.
\section{Elektrochrominiai ekranai.}
\section{Drėgmės matuokliai.}

\section{Atminties ląstelės.}
\section{Lilie's neuroninis modelis.}
