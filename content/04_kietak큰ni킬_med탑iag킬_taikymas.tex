\chapter{Kietakūnių medžiagų taikymas.}

\TODO{Kažko trūksta.}

\section{Frenkelio taškinio defekto teorija.}

Ši teorija remiasi principu, kad kristalinės gardelės superjonikų
nėra idealios. Tai yra turi defektus. Turint idealią gardelę, tai
kai ją sudaro katijonai ir anijonai (\slide{84} „Ideal lattice“),
tai jie tam tikroje gardelės plokštumoje išsidėsto taip, kaip
yra pavaizduota. Tai yra kiekvienam katijonui (paveikslėlyje pažymėtam
„$+$“) atitinka po vieną anijoną (paveikslėlyje pažymėtam „$-$“).
Pagal Frenkelį nesant idealiai kristalinei gardelei
gali susidaryti atvejis, kai katijonas palieka reguliarųjį gardelės
mazgą ir patenka į tarpmazgį (\slide{84} „Point Frenkel – type defects“). 
Tokiu atveju turime defektą, tai yra vakancinis katijoninis mazgas
ir katijonas tarpmazgyje. Tai yra kristalinės gardelės mazge mes
neturime masės, o katijonas su mase ir krūviu atsiduria tarpmazgyje.
Toks defektas, kai gardelėje turime katijoninį vakancinį mazgą
ir katijoną tarpmazgyje yra vadinamas taškiniu Frenkelio defektu.

Gali būti ir toks atvejis, kai gardelėje reguliarųjį mazgą palieka
anijonas ir jis atsiduria tarpmazgyje. Vadinasi turime defektą:
vakancinis anijoninis mazgas ir anijonas tarpmazgyje. Toks defektas
yra vadinamas taškiniu anti-Frenkelio defektu.

Kaip tai atrodys energetiškai, taškinio Frenkelio defekto situaciją
atvaizduosime tokiose koordinatėse: energija $W$ ant koordinatės $x$
(\slide{85}). 1 – kai katijonas yra užėmęs mazgą, 2 – tuščia
tarpmazginė padėtis, 3 – katijonas yra tarpmazgyje, 4 – neužimta
tarpmazginė padėtis, 5 – katijonas palikęs reguliarųjį mazgą.
Pagal Frenkelį jeigu tiktai katijonai patenka į gardelės tarpmazgį,
tai jiems migruoti yra patogiau tarpmazginėmis padėtimis (Nes
mažesni energetiniai barjerai.)

Didinant kristalų temperatūrą, dalis jonų dėl šiluminio virpėjimo
palieka kristalinės gardelės mazgus ir peršoka į tarpmazgį.
\TODO{formulės}

Perėjus katijonui iš mazgo į tarpmazgį, jo vakancija tampa neigimai
įelektrinta. Kiekvienas kristalinės garedelės mazgą palikęs jonas
kuria defektų porą -- vakancinį mazgą (v) ir taprmazginį joną (j):
$N_j = N_v$

Esant šiluminei pusiausvyrai, defektų skaičisu kristale priklauso nuo
jo temperatūros. Jeigu, susidarius vienam taškiniam Frenkelio 
defektui Gibso energija pakinta dydžiu $G$, tai susidarius $N_F$
defektams, termodanaminis kristalo potencialas padidėja dydžiu
$N_F \cdot G_F$ ir sumažėja dydžiu $T \cdot S$ (T - temperatūra
kelvino skalėje, S - konfigūracinė entropija):
\begin{equation*}
	$F = N_F \cdot G_F - T \cdot S$
\end{equation*}

Iš Frenkelio teorijos reikia žinoti, kad Frenkelis rėmėsi termodinaminiais
modeliais, kur $F$ yra termodinaminis potencialas, kuris yra išreiškiamas
per Frenkelio taškinių defektų koncentraciją, Gibso energiją $G$, kuri
yra reikalinga sukurti defektą, ir minus temperatūra padauginta iš
entropijos. Gibso energija yra išreiškiama:
\begin{align*}
  G_{F} = H_{F} - T \cdot S_{F}, \\
\end{align*}
čia:
\begin{description}
  \item[$G_{F}$] – Gibso energija, reikalinga sukurti taškinį defektą;
  \item[$H_{F}$] – entalpija (fizikinis dydis, apibūdinantis
    termodinaminės sistemos būseną, išreiškiantis izoliuotos
    sistemos reiškinių negrįžtamumą);
  \item[$T$] – temperatūra;
  \item[$S_{F}$] – taškinio Frenkelio defekto susidarymo entropija
  (termodinaminės sistemos būseną apibūdinantis kintamasis dydis,
  kurio nepriklausomi argumentai yra termodinaminės sistemos
  entropija, slėgis, dalelių skaičius ir apibendrintosios jėgos)
\end{description}

Entropija tai yra šilumos pokytis $\Delta Q$ esant tam tikrai
sistemos temperatūrai $T$:
\begin{equation*}
  S_{F} = \frac{\Delta Q}{T}.
\end{equation*}

Entalpija:
\begin{equation*}
  H_{F} = U + pV,
\end{equation*}
čia:
\begin{description}]
  \item[$U$] – laisvoji energija;
  \item[$p$] – slėgis;
  \item[$V$] – tūris.
\end{description}

Didėjant temperatūrai Frenkelio taškinių krūvių koncentracija eksponentiškai
didėja, o eksponentės rodiklyje yra Gipso energija reikalinga sukurti
Frenkelio defektui, padalinta iš dviejų šiluminių energijų.

\begin{remember}
  \item Kokią idėja Frenkelis rėmėsi pateikdamas savo joninės pernašos
    teoriją, tai yra išskyrėme koncepciją, kad turime ne idealias
    kristalines gardeles, kuriose gali būti atvejai, kai kristalinėse
    gardelėse dominuoja Frenkelio arba anti-Frenkelio taškiniai defektai.
  \item Reziumuodamas savo teoriją Frenkelis gavo formulę
    $N_{F} = \sqrt{NN'}^{-\frac{G_F}{2kT}}$, kuria teigia, kad didėjant
    temperatūrai Frenkelio taškinių defektų koncentracija
    eksponentiškai didėja ir eksponentės laipsnis yra Gipso energija
    reikalinga sukurti Frenkelio defektui, padalinta iš dviejų
    šiluminių energijų.
\end{remember}

\section{Šotkio taškiniai defektai}

Plačiau: \url{http://en.wikipedia.org/wiki/Schottky_defect}.

Šotkis turėdamas daug eksperimentinių faktų, kurie byloja apie tai, kad
anijonai arba katijonai gali palikti kristalinę gardelę, tai yra
jos tūrį ir migruodami link kristalo paviršiaus, pasiekę
paviršių jame lokalizuojasi. Pavyzdžiui, jei anijonas palieka kristalinės
gardelės turį (\slide{87}) ir migruoja link paviršiaus ir pasiekęs
paviršių lokalizuojasi, tai turime taškinį defektą kristalinės
gardelės tūryje – vakancinį anijoninį mazgą ir anijoną kristalo
paviršiuje. Toks taškinis defektas yra vadinamas Šotkio taškiniu
defektu. Gali būti situacija kitokia: katijonas palieka kristalinės
gardelės tūrį, migruoja link kristalo paviršiaus, pasiekęs kristalo
paviršių lokalizuojasi jame ir turime defektą – vakancinį katijoninį
mazgą kristalinės gardelės tūryje ir lokalizuotas katijonas kristalo
paviršiuje. Toks taškinis defektas yra vadinamas anti-Šotkio taškiniu
defektu.

Šotkis laikydamasis šios koncepcijos pateikė modelį, kuris aprašė
taškinių defektų koncentracijos priklausomybę nuo temperatūros.
Šotkis, kaip ir Frenkelis naudojosi pagrindinėmis termodinaminėmis
lygtimis. Šotkio teorijos išvada yra, kad didėjant temperatūrai
Šotkio taškinių defektų koncentracija didėja eksponentiškai ir
eksponentės rodiklyje yra Gipso energija reikalinga sukurti
taškiniam Šotkio defektui, padalinta iš vienos šiluminės
energijos(\slide{88}):
\begin{equation*}
  N_{\check{S}} = N e^{-\frac{G_{\check{S}}}{kT}}
\end{equation*}
čia $N$ yra koncentracija.

\TODO{Įdėti formules su paaiškinimais.}
\begin{equation*}
  F = N_Š \cdot G_Š - T \cdot S
\end{equation*}
čia
\begin{description}
  \item[$F$] - termodinaminis potencialas
  \item[$N_Š$] – Šotkio defektų porų skaičius;
  \item[$G_Š$] – Gipso energija $G_Š = H_Š - T \cdot S_Š$;
  \item[$T$] – temperatūra, kelvino skalėje.
  \item[$S$] - konfigūracinė entropija
  \item[$S_Š$] - taškinio Šotkio defekto susidarymo entropija
  \item[$H_Š$] - entalpija
\end{description}

 
\begin{remember}
  \item Kokios idėjos vedamas Šotkis sukūrė savo teorija. Kitaip tariant
    reikia žinoti kas tai yra Šotkio taškiniai defektai, anti-Šotkio
    taškiniai defektai.
  \item Termodinamikoje yra įvedamos \FIXME{kas?}, kurios yra susietos
    su koncentracija, temperatūra, entropija.
  \item Koncentracijos ir temperatūros priklausomybė:
    $N_{\check{S}} = N e^{-\frac{G_{\check{S}}}{kT}}$.
\end{remember}

\section{Superjoninės keramikos}

Įvairiose technikos srityse superjoninės medžiagos yra naudojamos,
kaip iš jų pagaminti plonieji arba storieji sluoksniai, o taip
pat keramikos. Keramikos arba plonųjų, storųjų sluoksnių gamyboje
yra svarbios išeities medžiagos, kurios yra naudojamos keramikų
arba plonųjų, storųjų sluoksnių gamyboje. Paprastai tai yra naudojami
milteliai, kurių vienas pagrindinių parametrų yra ne tik tai tai
jų švara ir atitikimas cheminei formulei, tačiau yra svarbus
jų grūdėtumas. Grūdėtumui nusakyti yra naudojamas taip vadinamas
BET parametras. Šio parametro dimensija yra kvadratinis metras
atitinkantis medžiagos gramui. Tai reiškia, kad jei mes grūdus
sudėsime vieną prie kito, tai vienas gramas tų grūdelių gali
užimti tam tikrą plotą. Paprastai gamyboje yra naudojami milteliai,
kurių BET yra nuo kelių kvadratinių metrų iki šimtų kvadratinių
metrų. BET atsirado taip: pirmieji pateikė metodą, kaip nustatyti
grūdelių grudėtumą trys autoriai Stefan Brunauer, Hugh Emmet ir
Edward Teller. „BET“ yra jų pavardžių pirmosios raidės. (\slide{89})

\slide{90} pateikti deguonies vakancijų superjonikų lentelė.

\TODO{Trūksta dalies konspekto: [90;92] skaidrės.}
91-toje skaidrėje pavaizduota grudėtumo ir „BET“ priklausomybė.
92-antroje skaidrėje pavaizduotos monolitinės (dešinė viršus ir
apačia), bei tarpkristalinės sandūros („piltuvėlis“ kairėje).

… nusakomas ne tik tai parametru BET, bet ir chemine medžiagos švara,
tai yra priemaišų, nuo kurių niekaip negalima išsivaduoti. Tokiu
atveju formuojantis keramikai priemaišos yra iš monokristaliukų 
išstumiamos ir lokalizuojasi tarpkristalinėse sandūrose.
Jeigu išeities lentelėse(?)
yra daug priemaišų, jos lokalizuojasi dar ir rezervuaruose.
Kaip taisyklė tarpkristalinės sandūros neatitinka cheminės sudėtis
monokristalų, tai dažniausia būna stiklai. O stiklai, kaip
žinia yra amorfinės medžiagos neturinčios kristalinės struktūros,
o dalelę prie dalelės išlaiko tik artiveikės jėgos.

\begin{remember}
  \item Gaminant medžiagas ar tai keramikas, ar tai storuosius,
    plonuosius sluoksnius yra svarbu turėti išeities medžiagas,
    atitinkamos cheminės švaros ir grūdėtumo.
  \item
    \item Koks parametras nusako grūdėtumą („BET“),
    kokiais vienetais grūdėtumas yra matuojamas ($m^2/g$),
    kokią įtaką grūdėtumas turi keramikų tankio formavimui.
    \FIXME{Hipotezė: Kuo didesnis grūdėtumas, tuo tankesnė
    keramika. (Palyginti: \slide{91} a nuotrauką su b)}
    \item kokia yra
    keramikų nano ar tai mikro struktūra. Tai yra tarpkristalinės
    sandūros ir patys monokristaliukai sudaro keramikas. Tai yra
    dviejų bazių sistema: monokristaliukai ir tarp \FIXME{Kokios?}
    sandūros, kurios nėra kristaliukai. Sandūros priklauso nuo
    cheminės medžiagos švaros, jei priemaišų yra daug, tai susidaro
    ne tik plonos sandūros, bet susidaro netgi rezervuarai tom
    išstumtom priemaišom.
\end{remember}

Žr. 89 skaidrę (In 1938...)

Kai iš Jų pagaminami plonieji arba storieji sluoksniai (t. y. keramikos),
svarbios išeities medžiagos, kurios yra naudojamos keramikų (plonųjų ir
storųjų) gamybai ...

Paprastai naudojami milteliai, kurių vienas pagrindinių parametrų,
ne tik jų švara ir atitikimas cheminei formulei,
svarbus jų grudėtumas, kuris ir žymimas \textbf{BET} parametru.

Tankesnė keramika (žr. SEM images \textit{a}), tai didesnis skaičius.

Nano struktūra (žr. 92 skaidrę).
\begin{itemize}
  \item Monokristalai (dešinėje viršuje ir apačioje). Pasižymi chemine
    medžiagos švara
  \item Tarpkristalinės sandūros („piltuvėlis“ kairėje).
    Jose lokalizuojasi priemaišos, taigi sandūrų sudėtis neatitinka
    [bendros medžiagos struktūros (?)]. Viduje yra stiklai (amorfinės
    medžiagos)
\end{itemize}
Skirtingos struktūros skiriasi savo elektrine talpa ir varža.

\begin{remember}
  \item Gaminant medžiagas (storąsias ir plonąsias keramikas), reikia
    [atkreipti dėmesį į] išeities medžiagas, cheminę švarą ir grūdėtumą.
  \item Grūdėtumas – koks parametras (BET), vienetai ($\frac{m^2}{g}$),
  \item Kokia nano struktūra: monokristalinės ir tarpkristalinės.
\end{remember}

\section{Keramikų elektrinės savybės}

% Balsas-0002.mp3 5:35 – 32:00

Norint pagamintas medžiagas vienaip ar kitaip taikyti, reikia mokėti
matuoti jų elektrines savybes. Paprastai tam, kad pasinaudoti
superjoninėmis keramikomis, tai turi būti jų joninis laidumas
kuo didesnis, o aktyvacijos energija kuo mažesnė. Nagrinėsime keramikų
elektrines savybes. Kaip žinia joninis laidumas yra išreiškiamas,
kaip ir puslaidininkių atveju: elementarusis krūvis, koncentracija,
judrumas. Iš Frenkelio ar tai Šotkio teorijų žinome, kad defektų
koncentracija didėjant temperatūrai didėja eksponentiškai, o aktyvacijos
energija tai yra Gipso energija.
  {Antanas Garšva nukabina keltuvininko uniformą. Mėlynas kelnes su
  raudona siūle ir burokinės spalvos švarką su mėlynais atlapais,
  „auksinėmis“ sagomis, pintais antpečiais. Švarko atlapų kampuose
  blizga numeriai. 87 iš kairės, 87 iš dešinės. Jei svečias
  nepatenkintas keltuvininku, jis gali, prisiminęs numerį, įskųsti jį
  starteriui. „87 kalės vaikas, 87 užkėlė mane keturiais aukštais
  aukščiau, 87 87 87, aš sugaišau dvi minutes šioje dėžėje,
  prakeiktas kalės vaikas 87!“}%
\footnote{\url{http://antologija.lt/texts/58/tekstas/01.html}}
Tačiau visuomet, jeigu tik tai kalbama apie savybes sudėtingų
elektrinių medžiagų, tokių kaip keramikos, iš karto turime
įsivaizduoti, arba turėti elektrinį ekvivalentą, kurį galima
naudoti nusakant medžiagą. Paprastai medžiagos yra nusakomos arba
lygiagrečia, arba nuoseklia elektrine grandine, sudaryta iš
kondensatoriaus ir varžos (\slide{93}). Šiose medžiagose induktyvumo,
kaip elemento neturime.

Tokiu atveju, jeigu kalbėti apie diagramas, tai mes turime dvi
lygiagrečias grandinėles sujungtas nuosekliai, tai yra $C$
tarpkristalininių \FIXME{Kas?}, $R$ tarpkristalininių \FIXME{Kas?}
ir tai yra prijungta prie $C$ kristaliukų ir $R$ kristaliukų.
Paprastumo dėlei panagrinėkime vieną iš šių grandinėlių. Tarkime,
kad turime tik kristalinę. Paprastai talpa … prijungiame prie
generatoriaus, grandinė užsidaro … Generatorius yra sinusinių
arba kosinusinių signalų. Tokioje situacijoje mes turime, kad
įtampa generatoriaus, kadangi jis lygiagrečiai pajungtas prie
grandinės, tai kritimas įtampos bus toks pats tiek kondensatoriuje
tiek varžoje. Kitaip sakant, jeigu mes pažiūrėsime …, tai $U$
bus ta pati vertė kondensatoriuje ir varžoje. Tačiau, jeigu
turime lygiagretų jungimą, tai pro kondensatorių tekės srovės
stipris $I_{C}$, o per varžą $I_{R}$.  Jeigu turėtume nuoseklų
jungimą, tai turėtume, kad … Paprastai šis dalykas yra
analizuojamas pasinaudojant vektorinių diagramų metodu.
… Atidedam įtampa, tai srovė tekanti per aktyvųjį elementą sutampa
su faze su įtampa. \TODO{Įtampos srovės rezonansas.} …
Jeigu turim nuoseklų kontūrą, tai įvyksta įtampos rezonansas,
o jei lygiagretų, tai srovės stiprio rezonansas. Tekanti
srovė per talpą lenkia įtampą $90\deg$ kampu. … Jeigu kokį
vektorių kilosime lygiagrečiai jam pačiam kokioje nors plokštumoje, tai
nuo to sistema nesikeičia. Jeigu $I_{C}$ yra pagal modulį tokios
vertės ir lenkia $90\deg$ kampu $I_{R}$, kuris pagal modulį
tarkime yra tokios vertės, tai {
  Smagu kolioti skaičių. Smagu operuoti
  skaičiais. 24035 į Sibirą. Smagu. 47 žuvo lėktuvo katastrofoje.
  Smagu. Parduotos 7038456 adatos. Smagu. Šiąnakt Mister X buvo
  laimingas 3 kartus. Smagu. Šiandien Miss Y mirė 1 kartą. Smagu. Šiuo
  metu esu vienas ir prarysiu vieną tabletę. Ir man bus smagiau.}%
\footnote{\url{http://antologija.lt/texts/58/tekstas/01.html}}
mes galime $I_{R}$ perkelti lygiagrečiai jam pačiam tarkime į čia.
Sudedame vektorius $I_{C}$ ir $I_{R}$, gauname atstojamąją $I$.
Ką mes čia turime: kampą tarp $I$ ir $U$ bei kampą papildantį iki
$90\deg$ pažymėkime $\delta$. Jeigu skaičiuosime šito kampo tangentą,
tai būtų srovių santykis: $I_{R}$ su $I_{C}$. Įstačius $I_{R}$
iš Omo dėsnio ($I_{R} = \frac{U}{R}$), o reaktyvioji varža $I_{C}$
($I_{C} = \frac{U}{\frac{1}{\omega \cdot C}}$). Padalinę vieną
iš kito gausime:
\begin{align*}
  \tan \delta
  &= \frac{I_{R}}{I_{C}} \\
  &= \frac{\frac{U}{R}}{\frac{U}{\frac{1}{\omega \cdot C}}} \\
  &= \frac{1}{\omega R C}, \\
\end{align*}
čia $\omega (2 \pi f)$ yra generatoriaus ciklinis dažnis. …

Paprasčiausiu atveju mes turime plokščią kondensatorių, tai $C$ būtų:
\begin{equation*}
  C = \frac{\varepsilon \varepsilon_{0} S}{d},
\end{equation*}
čia:
\begin{description}
  \item[$\varepsilon$] – medžiagos santykinė dielektrinė skvarba;
  \item[$\varepsilon_{0}$] – dielektrinė konstanta;
  \item[$d$] – atstumas tarp plokštelių;
  \item[$S$] – plokštelių plotas.
\end{description}
Kaip žinia, dielektrinė skvarba yra kompleksinis dydis: reali dalis
$-$ menamoji dalis. … $\varepsilon''$ yra susijęs su medžiagos
laidumu $\sigma$:
\begin{equation*}
  \sigma = \varepsilon' \varepsilon_{0} \omega \tan \delta.
\end{equation*}
Dabar $\tan \delta$ gali išreikštas, kaip dielektrinių skvarbų santykis:
realios ir menamos. Tokiu atveju, jei vietoj $\tan \delta$ įstatysime
$\frac{\varepsilon''}{\varepsilon'}$, tai $\varepsilon'$ susiprastins
ir lieka:
\begin{equation*}
  \sigma = \varepsilon_{0} \omega \varepsilon''.
\end{equation*}
Iš tokių matavimų seka pagrindinis parametras, kuris yra labai
svarbus taikymo atveju, tai laidumas. … Kadangi šitas
priklauso eksponentiškai, šitas yra konstanta, šitas kaip puslaidininkų
teorijoje eksponentiškai priklauso, tai turime ir laidume …
gausime nuo atvirkštinės temperatūros eksponentinį kitimą \slide{94}.

… 

Joninių bei superjoninių kristalų, kuriuose vyrauja taškiniai
defektai, kompleksinė dielektrinė skvarba:
\begin{equation*}
  \tilde{\varepsilon} = \varepsilon' - i\varepsilon''
\end{equation*}
Čia
\begin{description}
 \item[$\epsilon'$]- laukų santykis (medžiaga-vakuumas).
 \item[$\epsilon''$] - medžiagos laidumas.
\end{description}

Kristalą esantį elektriniame lauke galima pavaizduoti lygiagriačiojo
arba nuosekliojo jungimo RC schemomis. Tokio jungimo schemomis
tekančios bendrosios srovės tankis
\begin{equation*}
  \tilde{j} = j_a + ij_r
\end{equation*}
Čia
\begin{description}
  \item[$j-a$] - aktyvusis tankio sandas
  \item[$j-r$] - reaktyvusis tankio sandas
\end{description}

$\sigma$ - nuostolių kampas

$tg\sigma$ - kokia elektros energijos dalis RC grandinėje virsta
šilumine energija

...

\begin{remember}
  \item Kadangi keramikos yra dviejų bazių visuma: tarkristalininių
    sandūrų ir kristalinių, tai jos gali būti elektriškai atvaizduotos,
    kaip lygiagrečiai sujungtos talpa ir varža, taip ir nuosekliai.
    (Nagrinėjom lygiagrečiai, nuosekliai nebūtina žinoti.)
  \item Vienas iš parametrų nusakančių keramiką yra laidumas, o laidumai
    plaukia iš srovės tekėjimo analizės esant kintamam lauke per
    lygiagrečiai sujungtus varžą ir kondensatorių. …
  \item Vektorinių diagramų metodu yra prieinama prie laidumo
    $\sigma$ išraiškos nuo dažnio ir menamos dalies kompleksinės
    dielektrinės skvarbos.
  \item Kaip matuoti laidumą taikant tokius analizės metodus.
  \item Kokias eksperimentines kreives gaunam, tai yra, kad
    priklausomybė nuo temperatūros yra eksponentinė.
  \item Kad priklauso laidumas nuo medžiagos, iš kurios yra pagaminta
    keramika, išeities parametrų.
\end{remember}


02.mp3: 5:50


Žr. 93 skaidrės (su elektros grandinėmis)

$C = \frac{\epsilon \cdot \epsilon_0 \cdot S}{d}$ - elektrinė talpa.
$S$ - plotas.
$d$ - atstumas tarp plokštelių.
$R$ - varža.
$U$ - įtampa.
$\omega = 2 \pi f$ - generatoriaus ciklinis dažnis.
$\epsilon'$ - laukų santykis (medžiaga-vakuumas).
$\epsilon''$ - medžiagos laidumas.
$\mu$ - judrumas

$I$ - srovės atstojamoji.
$I_R = \frac{U}{R}$
$I_C = \frac{U}{\frac{1}{\omega \cdot C}}$

\begin{remember}
  \item Keramikos (monokristalinės ir terpkristalinės sandūros) gali
  būti atvaizduotos nuosekliai ir lygiagrečiai.
  \item Laidumo parametras\begin{itemize}
    \item Kintamame lauke $I_C$ lenkia $I_R$ $90\deg$ kampu.
    \item Vektorių diagramų metodu: $I = tg \sigma = \frac{I_R}{I_C}$
    \item Dažnio, realios ir menamos dalies
    $\epsilon = \epsilon' - i\epsilon''$.
    $\sigma (?) = \epsilon \epsilon_0 \omega tg \sigma$ (?)
    \item Laidumo priklausomybė nuo išeities parametrų
    $\sigma = qN\mu$ (?)
  \end{itemize}
\end{remember}

\section{Plonieji sluoksniai}

% Balsas-0002.mp3 32:00

Plonieji sluoksniai priklausomai nuo … irgi yra nusakomi tam tikra
mikro arba nano struktūra. Tiek plonieji sluoksniai, tiek keramikos
yra naudojami įvairiose technikos srityse, kuriant vienokios arba
kitokios paskirties fizikinius įrenginius.
Plonieji sluoksniai gali būti polikristaliniai, turėti salelinę
(nuo žodžio „salelė“, maža sala) struktūrą, arba monokristaliniai.
\slide{95} pavaizduota plonųjų sluoksnių salelinė struktūra. Tai
yra į monokristalines sritis yra įsiterpusios salelės, įvairių dydžių,
kuriose yra polikristalinė struktūra. Jeigu čia yra monokristalinė,
tai yra vienalytė, tai čia jau turime salelėse polikristalus. Tai
yra ta pati keramika, kuri susideda iš monokristaliukų ir
tarpkristalinių sandūrų (nuotraukoje juodos dėmės). Tokiu atveju,
kai turime kelias fazes monokristalinę ir polikristalinę, tai jų
fizikinės savybės bus skirtingos. Tai yra laidumai pagal savo vertę
bus skirtingi. Taigi bendrasis laidumas bus šių skirtingų laidumų
suma. \slide{96} matosi, kad pavyko nusodinti ploną sluoksnį, kuris
tam tikrame plotelyje yra vienalytis arba monokristalinis. Aišku,
kad nuotrauka yra padaryta tam tikros to plono sluoksnio dalies,
kur pasisekė išskirti tą tarpkristalinę medžiagą.
% Balsas-0003.mp3
Priklausomai nuo to, kaip yra nusodinami tie sluoksniai, jų laidumas
irgi priklauso tiek nuo nusodinimo metodo, tiek nuo to ant kokio taip
vadinamo padėklo yra suformuojamas tas sluoksnis. Jeigu turime
homogoneninį atvejį, tai jo laidumas visada yra didesnis nei tokio su
salelinėm struktūrom. Kalbant apie plonuosius sluoksnius, reikėtų
žinoti, kad priklausomai nuo to, kokį naudojame metodą pagaminti
ploną sluoksnį, jie gali būti su salelėmis polikristalų vienalytėje
monokristalinėje matricoje. Tos salelės, kurios yra polikristalinės
yra nusakomos tokia nano ar mikro struktūra, kaip ir keramikos.
Monokristalinių plonųjų sluoksnių sričių laidumas visada pagal savo
vertę yra didesnis nei polikristalų. Didėjant temperatūrai tiek
homogeninių, tiek su saleline struktūra laidumai eksponentiškai
kyla.

\FIXME{Trūksta dalies konspekto.}

33:11

Žr. 95 skaidrę (\textit{SEM image of YSZ ...})

Priklauso nuo gavimo, nusakomi mikro arba nano strukūra.
Plonieji sluoksniai ir keramikos naudojami įvairiose technikos srityse,
kuriant vienokios arba kitokios pask (?) fizikinius įrangą (?).

Plonieji sluoksniai\begin{itemize}
  \item Polikristaliniai
  \item Turinčios saleles (jų struktūra) - polikristalinės
  (monokristalinės + tarpkristalinės sandūros)
  \item Monokristaliniai
\end{itemize}

Visų laidumai skirtingi, pagal tai, kokį naudojam metodą pagaminti su
salelėmis - polikristalinėmis vienalytėje monokristalinėje
matricoje.
Monokristalų laidumas visuomet didesnis negu polikristalų.
Didėjant temperatūrai, laidumai didėja eksponentiškai.

Salelių struktūra yra tokia pati, kaip ir keramikų.

\section{Kietakūniai elektros energijos generatoriai.}

Nernsto elektrovaros jėga:
\begin{equation*}
  \varepsilon = \frac{RT}{4F} \ln \frac{P_{O_{2}}}{P_{O_{2}}^{x}},
\end{equation*}
čia:
\begin{description}
  \item[$R$] – universalioji dujų būvio konstanta;
  \item[$T$] – temperatūra Kelvino skalėje;
  \item[$F$] – Faradėjaus skaičius;
  \item[$P_{O_{2}}$] – dalinis ore esančių $O_{2}$ dujų slėgis;
  \item[$P_{O_{2}}^{x}$] – dalinis dujose (kure) esančių $O_{2}$
    dujų slėgis.
\end{description}

Reakcija vykstanti oro-YSZ sandūroje (prie katodo):
\begin{equation*}
  O_{2} + 2\ovacancy + 4e^{-} \to 2O_{O}^{x}.
\end{equation*}
Reakcija vykstanti YSZ-kuro sandūroje (prie anodo):
\begin{equation*}
  O_{O}^{x} + H_{2} \to \ovacancy + H_{2}O + 2e^{-}.
\end{equation*}

Naudojami deguonies vakancijų superjonikai. Deguonės vakancija – ,
neturinti masės.

Plačiau: \url{http://gedmin.as/study/inf98/physics/Pask_sup_jung_taik.pdf}

$\epsilon$ - Nersto elektrovara.
$R$ - universalioji dujų konstanta.
$T$ - temperatūra (kelvinais).
$F$ - Faradėjaus skaičius.
$P_{O_2}$ - slėgis ore (prie katodo).
$P{O_2}^x$ - slėgis kure (prie anodo).

\begin{remember}
  \item Kokios medžiagos (pagrindiniai krūvininkai deguonies vakancijos
  ($V_{O}^{..}$) 
  \item Veikimo principas (Deguonis prie katodo → deguonis
  disocijuojasi ir jonizuojasi →  atomas per deguonies vakancijas
  iki anodo → susijungia su $H_2$ → rezultatas $H_2 O$ ir elektronas)
  \item Nusakomas Nersto elektrovaros jėga
  ($epsilon = \frac{R \cdot T}{4 \cdot F}ln \frac{P_{O_2}}{P_{O_2}^x}$)
\end{remember}

\section{Kietakūniai elektros energijos akumuliatoriai.}

Visi akumuliatoriai yra charakterizuojami trimis pagrindiniais parametrais:
\begin{description}
  \item[$W$]  – savitoji energetinė talpa
    $\left[ \frac{A \cdot h}{kg} \right]$;
  \item[$t$] – akumuliuoto krūvio saugojimo laikas;
  \item[$n$] – įkrovimų ir iškrovimų skaičius.
\end{description}

Tokiose akumuliatoriuose, kuriuose naudojami vienokie ar kitokie
superjonikai, krūvininkų akumuliacija yra atliekama jeigu sistemoje
vyksta cheminės reakcijos, kurių metu atsiranda elektronai, kurie
ir yra akumuliuojami. Kitu atveju, kai sistemoje viename iš
elektrodų kinta tam tikrų elementų valentingumai. Dėl valentingumų
kaitos, kad būtų išlaikytas krūvininkų išsilaikymo arba tvermės
dėsnis turi atsirasti elektronai, kurie yra akumuliuojami.
Visuomet akumuliatoriuose kadangi yra du elektrodai, tai yra anodai
ir katodai, tai tokiose sistemose vyksta dviejų tipų reakcijos:
anodinė ir katodinė.

Vienas iš pavyzdžių anodinės reakcijos \slide{110}.
Visais atvejais išlaikomas krūvininkų tvermės dėsnis.

Išsiaiškinti:
…
Kad išliktų $\beta$ modifikacija, tai $x$ $Na_{2}O-\beta-x Al_{2}O_{3}$,
$0 \leq x \leq 11$. Jei $x$ yra 11, tai ši medžiaga yra delektrikas,
nes visi Ant…Roso mazgai yra tušti, o elektronai gali keliauti tik
jais. Taigi mus tinka, tik kai $6 \leq x \leq 10$.

Pirmieji akumuliatoriai panaudoti mobiliuose telefonuos buvo nikelio
vandenio junginiai. Tada siekė maždaug $30$. Dabartinių ličio
jonų gali siekti iki $200$. Mes kalbėsime apie kristalinius
ličio, kurių gali siekti iki $400$.

Automobiliuose yra rūgštiniai akumuliatoriuose. Jeigu ant akumuliatoriaus
rašo 60 ampervalandžių, jo svoris 15 kg, tai gausime $W=4$.

Kitas svarbus parametras yra saugojimo laikas. Rūgštinių automobilinių
akumuliatorių saugojimo laikas yra penki metai. Dėl $Ag/I_{2}$, tai
jų trukmė iki 18 metų \slide{111}.

Yra naujos kartos akumuliatoriai, kurių tiek savitoji energetinė
skvarba pranoksta 100 kartų, įkrovimo iškrovimo laikas praktiškai
neribojimas (10000 kartų), akumuliuoto krūvio saugojimo laikas
kelis kartus pranoksta rūgštinius.

Sieros akumuliatorius \slide{112} $Na/S$. Ampulė iš
$Na-\beta-polialiuminato$. S ir C mišinys – sieros ir anglies mišinys.
Kai sumaišome sierą su smulkiagrūde anglimi, tai … plotas padidėja.
Dar naudojama natrio dagtis, tai yra … prisotinta natrio.

Na-$\beta$-polialiuminatas yra absoliutus dielektrikas elektronams.
Todėl įkraunat susidarę elektronai negali nutekėti atgal į natrį.

\slide{112} apatiniame paveikslėlyje $\sigma_{j}$ – joninio laidumo
sandas, $\sigma_{e}$ – elektroninio laidumo sandas.

\slide{114} įkraunant prie viršutinio jungiamas +, o prie apatinio
-.

\begin{remember}
  \item Nors vieną sakinį iš istorinės dalies. Priklausomai nuo to
    kokius elektrodus mes naudojame ir superjoniką, taip ir vadinasi
    akumuliatoriai. (\slide{109})
  \item Dėl ko vyksta krūvininkų akumuliacija. Dėl dviejų reakcijų:
    kai vyksta betarpiška cheminė reakcija tarp elektroninių medžiagų,
    kai vyksta cheminių valentingumų kaita viename iš elektrodų.
  \item Kokie trys pagrindiniai parametrai nusako akumuliatorius
    (savitoji energetinė talpa, saugojimo laikas, įkrovimų – iškrovimų
    skaičius).
  \item Pakanka žinoti tik vieną iš akumuliatorių tipų (kokių),
    bet būtina plačiai.
\end{remember}

\section{Dujų jutikliai.}

Deguonies dujų jutikliuose yra naudojami deguonies vakancijų
superjonikai ($V_{O}^{..}$). Ką mes nagrinėjome buvo
$ZrO_{2}$ stabilizuotas 8 mol \% $Y_{2}O_{3}$. Priklausomai nuo
to kiek mol procentų įvedame, tai kiekvienas mol procentas skatina
pusės deguonies vakancijos atsiradimą. Jei 4 mol \% , tai …
O jau vakancijomis gali judėti deguonies antijonai ir tam tikrose
… gali kurti Nersto elektrovaros jėgą. \slide{115} yra Nersto
elektrovaros jėgos išraiška. $F$ – Faradėjaus skaičius, $T$ – temperatūra,
$R$ – universalioji dujų konstanta, $p'_{O_{2}}$ – dalinis deguonies
slėgis (etaloninės), $p'_{O_{2}}$ – dalinis deguonies slėgis dujose,
kuriose norime nustatyti deguonies dalį. \slide{115} deguonies jutiklio
paveikslėlis. Kai turime orą, tai $p'_{O_{2}} = 0,21 bar$.
26 – platina (Pt). Eina du platininiai laidai, tarp jų pajungę voltmetrą
matuojame Nersto elektrovaros jėgą. Sugradavus tokį jutiklį pagal žinomus
$p'_{O_{2}}$ galime jį naudoti nežinomiems $p''_{O_2}$.

Deguonies dujų jutikliai naudojami ten, kur vykdomas deginimas. Tokios
konstrukcijos, kuri pateikta \slide{115}, yra naudojami šilumos
tinkluose, taip pat ir Lietuvoje. Kodėl tai yra svarbu: deginant kurą
(oksiduojant jį), jei bus blogai sureguliuota maišyklė, tai į kaminą
išeis daug nesudegusio kuro. Tokie jutikliai buvo pagaminti Lietuvoje,
buvo patikrinti ES ekspertų ir buvo leista juos naudoti šilumos tinkluose.
Jie yra labai ilgaamžiai, jų yra labai lengva deregenaracija: reikia
atkaitinti ore ir jis vėl funkcionuoja, kaip naujas. Anksčiau nebuvo
naudojami jutikliai ir viską spręsdavo kurikas.

\begin{remember}
  \item Kas įeina į konstrukciją (YSZ).
  \item Jo veikimas yra paremtas Nersto elektrovaros jėgos atsiradimu
    esant dujų mišinyje tam tikram daliniam deguonies slėgiui.
  \item Į Nersto formulę įeina …
    Etaloninis dalinis slėgis yra slėgis ore.
\end{remember}

\slide{116} – sočiosios srovės deguonies dujų jutiklis. Yra
matuojamos voltamperinės charakteristikos elemento sudaryto iš dviejų
platinos elektrodų ir YSZ. Priklausomai nuo to kokia yra deguonies
koncentracija tyriamajame mišinyje, didinant įtampą atsiranda sritis,
kur srovė nepriklauso nuo įtampos. Didinant deguonies koncentraciją
mišinyje sočiosios srovės stipris didėja. Taip įsisotinimo sritis
srovės skalėje kyla į viršų. Sugradavus pagal žinoma
koncentraciją srovės sotį, turime galimybę žinoti kokia yra
deguonies dujų koncentracija tam tikrame dujų mišinyje.
Naudojama būtent platinos nes yra aukštatemperatūris metalas ir
jis nesioksiduoja.

\begin{remember}
  \item Žinoti struktūrinę \slide{116} schemą, kaip ji veikia, kaip
    priklauso srovė nuo įtampos, kai keičiam deguonies dujų
    koncentraciją.
\end{remember}

$\Lambda$-jutiklis. Naudojamas automobiliuose. \slide{117}.
Konstrukcijoje yra naudojami YSZ kietieji elektrolitai. Yra vidinis
elektrodas pagamintas iš platinos ir išorinis elektrodas pagamintas iš
platinos. $\Lambda$-jutiklis yra naudojamas, kad optimizuoti oro ir
kuro įpurškimą į deginimo sistemą. Priklausomai nuo to koks yra
įpurškiamas deginti mišinys, atitinkamai kis ta pati Nersto elektrovaros
jėga. Jeigu mes žinome kiek kuro įpurškiame, tai teorinis $\Lambda_{0}$
parodo, kiek reikia oro, kad kuras pilnai sudegtų. Kai $\Lambda$
teorinis sutampa su $\Lambda$ faktiniu, tai turime 1,0. Ampulė
pagaminta iš YSZ, kuri iš viršaus ir išorės padengta platina.

\begin{remember}
  \item $\Lambda_{0}$ kiekvienai kuro markei yra kitoks.
  \item Kas yra faktinė $\Lambda$.
  \item Kaip priklauso elektrovara nuo faktinės $\Lambda$.
\end{remember}

Anglies monoksido dujų jutiklis ($CO$ dujų jutiklis). Anglies
monoksidas dar vadinamas smalkėmis. Konstrukcija parodyta \slide{118}
paveiksle. Į konstrukciją įeina YSZ, platinos elektrodai, viršutinis
elektrodas yra padengtas katalizatoriumi. Katalizatorių pagrindinė
funkcija yra spartinti chemines reakcijas, tačiau nekeičia reakcijos
pobūdžio. Katalizatorius yra pagamintas iš platinos bei $\gamma$ aliuminio
trioksido kubinės modifikacijos. Jei turime aplinką, kurioje yra oras
sumaišytas su smalkėmis, tai esant didelei smalkių koncentracijai,
tai tokiu atveju turime mažą deguonies koncentraciją. Tokiu atveju
viršutinis katalizatorius spartina oksidacija ir $CO \to CO_{2}$.
Tačiau, jeigu mes sugraduosime, kaip kinta deguonies ar dalinis
s… turėsime tikrą vaizdą mišinyje esančių smalkių kiekio.

Visose šituose elementuose yra naudojami YSZ ir platina.

\begin{remember}
  \item Žinoti konstrukcinius ypatumas (naudojamas katalizatorius).
  \item Naudojama Nersto elektrovaros jėga.
\end{remember}

Anglies dioksido ($CO_{2}$) dujų matuoklis. Kam jis reikalingas:
$CO_{2}$ yra kenksmingos gyvam organizmui. Yra naudojami
NASICON (arba Natrio arba Ličio) superjonikai. Vokiečiai savo
pirmam jutiklyje naudojo natrio kietąjį elektrolitą. Vienas elektrodas
yra pagamintas iš natrio karbonato (paprasčiausia soda) ir uždengtas
$BaCO_{3}$ stiklu, o kitas iš platinos ir irgi uždengtas stiklu.
Patalpinus į $CO_{2}$ $NaCO_{3}$ skyla ir Na užima vakanciją.
Prie kito elektrodo niekas nekinta. Taigi atsiranda elektrovaros
jėga: $E = \varphi_{1} - \varphi_{2}$, kur $\varphi$ yra cheminis
potencialas.

\slide{119} pagamintas Lietuvoje $CO_{2}$ dujų jutiklis. Panaudotas
ličio superjonikas. Vienas elektrodas yra platinos, o kitas ličio
karbonatas su platinos elektrodu. Vėl susidaro potencialų skirtumas
patalpinus į $CO_{2}$ mišinį, nes skyla ličio karbonatas ir ličio
jonai užima vakancines vietas ličio superjonike. Sugradavus
galime matuoti $CO_{2}$ koncentracijas nežinomose dujose.

\begin{note}
  Per kontrolinį reikia žinoti tik vieną iš $CO_{2}$ jutiklių.
\end{note}

\begin{remember}
  \item Kokie yra naudojami superjonikai.
  \item Vienur pasikeičia, kitur nepasikeičia.
  \item Labanakt!
\end{remember}

\section{Elektrinės supertalpos (jonistoriai).}

TODO: Kažkiek praleidau.

Prie anglies elektrodo prijungiamas išorinio lauko $+$, o prie sidabro
$-$. Tokiu atveju argentum katijonai, būdami teigiamai įelektrinti
juda link argentum elektrodo, prisijungia elektroną ir tampa
neutraliais atomais, kurie nusėda ant elektrodo. Tokiu atveju, prie
anglies elektrodo susidaro argentum katijonų trūkumas. Taigi ten
lieka jonų vakancijos, kurios yra įelektrintos neigiamai, taigi
ta dalis įsielektrina neigiamai. $d$ lygu apie $1 \t{Å}$. … Tokiu atveju
išauga plotas. Paprastai, jei turime cilindrą ar $1 cm \times 1 cm$
plokštelę, tai dėl rėželių mes $S$ išdidiname iki $5m^{2}$.
$1\t{Å} \equiv 10^{-10} m$. Dabar galime apskaičiuoti talpą:
\begin{equation*}
  C = \frac{16 \cdot 8,85 \cdot 10^{-12}}{10^{-10}}
  \approx 7,08 F
\end{equation*}

Paprastai jonistorius naudoja kardio chirurgai širdies stimuliatoriams.
Jonistoriai yra įsiuvami po oda, jei žmogui pasireiškia širdies aritmija,
tai elektrodai prijungiami prie širdies ir ją sutraukia. Anksčiau
buvo naudojami akumuliatoriai, kuriuos tekdavo nešiotis išorėje.
Kadangi elektroninio laidumu sandas yra labai mažas, tai elektros
nutekėjimas yra mažiausias.

\begin{remember}
  \item Įprastai kodensatorių talpos yra nusakomos jų geometrija.
  \item Prisiminti kaip plokščio kodensatoriaus talpa susijusi su
    jo geometrija.
  \item Kaip veikia jonistorius, kaip jis įkraunas.
  \item Kokie yra geometriniai jonistoriaus parametrai (ploto ir atstumai
    tarp teigiamo ir neigiamo krūvininko jonistoriai).
  \item Su kuo susijusi ne tik talpa, bet ir jonistoriaus nuotekio
    srovė.
\end{remember}

\section{Deguonies langas (deguonies siurblys)}

Deguonies langai yra naudojami tose patalpose, kuriose reikalinga
papildyti deguonimi ir sudaryti normalią atmosferą, kad galėtų
egzistuoti gyvieji organizmai. Paprastai deguonies langai tarnauja
kaip po langai šviesai patekti į patalpą taip ir deguonies papildymui
iki normalios atmosferos (21\% deguonies, kitkas – azotas). Įprastai
deguonies langai yra naudojami tose laboratorijose, kuriose gaminami
vaistai arba medžiagos įvairioms pramonės sritims (švarios medžiagos).
Paprastai langų gamyboje yra naudojamos membranos pagamintos iš
YSZ (Itrio stabilizuoto Cirkoniu, $ZrO_{2}$ 8mol\% $Y_2O_{3}$). Jeigu
ši membrana yra apie dešimties mikronų storio, tai ji yra skaidri
šviesai. Ant membranos yra nusodinama kelių Å storio platinos sluoksnis,
tokiu atveju sistema platinos elektrodas, YSZ ir platinos elektrodas
yra skaidrūs šviesai.

\slide{TODO} Tarkime mums reikia papildyti viršutinę patalpą deguonimi.
Tokiu atveju reikia prijungti + prie viršutinio elektrodo, kuris
yra nukreiptas į patalpos vidų, o - prie atmosferos. Tokiu atveju
išoriniame sluoksnyje deguonis yra jonizuojamas ir deguonies jonai
deguonies vakancijomis juda į vidų, ten gauna elektroną ir tampa
neutraliu atomu.

Papildymo greitis $V = \frac{I}{4F}$, čia $F$ – Faradėjaus skaičius.
Jei $I = 1A$, tai 10 $cm^{2}$  dydžio langelis pompuoja 3 ml / min.
\begin{remember}
  \item Kokiam tikslui yra naudojami.
  \item Struktūrinė schema.
  \item Kodėl yra siurbiamas deguonis iš lauko į vidų.
  \item Kokie maždaug parametrai.
\end{remember}

\section{Elektrolyzeriai.}

Elektrolyzeriai yra naudojami gaminti iš vandens molekulių deguonį
ir vandenilį. Tai yra naudojami H20 molekulių skaidymui. Elektrolyzeriaus
struktūrinė schema parodyta \slide{TODO}, tai yra vamzdis pagamintas
iš YSZ. Jo išorinė ir vidinė pusė (vamzdžio) yra padengta platina
ir prie išorinės ir vidinės pusės yra prijungiamas elektros srovės
šaltinis $E$. Į vamzdį yra purškiami perkaitinti vandens garai.
Jų temperatūra siekia apie 1000 laipsnių Celsijaus. Kitaip sakant
esant tokiai temperatūrai, vandens molekulė yra ant skaidymosi
ribos. Elektrinis laukas stimuliuoja tą skaidymąsi papildomai ir
nukreipia deguonies anijonus iš vidinės vamzdžio pusės į išorinę.
Tai yra, prie vidinės pusės yra pajungiamas neigiamas potencialas,
o prie išorinės – teigiamas. Tokiu atveju jonizuoti deguonies
atomai juda iš vidinės į išorinę elektrolizerio pusę. Išorinėje
elektrolyzerio pusėje deguonis yra surenkamas į rezervuarus, o 
vandenilis juda toliau elektrolizeriu ir jo išėjime irgi yra
surenkamas į rezervuarus. Vidinio vamzdžio pusėje vyksta skaidrės
viršuje pateikta reakcija. Išorėje…

\begin{remember}
  \item Kam skiriamas.
  \item Struktūrinę schemą.
  \item Kaip pajungiamas laukas.
  \item Atsiimti, kad paduodami perkaitinti vandens garai.
\end{remember}

\section{Elektrochrominiai ekranai.}

Tokių ekranų gamybai yra naudojami superjonikai ir yra panaudotas
elektrochrominis efektas. Kvazicheminė reakcija iš \slide{TODO}.

TODO Juostinio modelio brėžinys.
Jei į Wolframą įvedame Litį, tai … Mažėjant draustinės juostos ilgiui,
mes slenkame nuo ultravioletinio prie infraraudonojo. Mes elektriniu
lauku keičiame draustinės juostos plotį ir tuo pačiu keičiame medžiagos
lūžio rodiklį. Vienu poliarumu įvedame litį į medžiagą ir tuo pačiu
mažiname draustinės juostos plotį, o kitu – išvedame litį iš medžiagos
ir tuo pačiu didiname draustinės juostos plotį. Kitaip sakant
su \slide{TODO} pateiktu įrenginiu galime keisti spalvą.

ECD yra naudojami labai dideliems pagal plotį ekranams, kadangi
gamyba yra labai pigu. Sekcijos yra gaminamos tiesiog nudažant.

Techninės charakteristikos:
\begin{enumerate}
  \item spalvų kaitos ciklų skaičius $n\approx 10^{7}$;
  \item spalvų kaitos laikas $t \approx 0,5 s$;
  \item optinio tankio pokytis, tenkantis …
  \item valdo nuo 85 iki 10 \% praeinančios matomos šviesos;
  \item $100 cm^{2}$ ECD naudoja 0,5 mwh elektros energijos.
\end{enumerate}

\begin{remember}
  \item Kas tai yra.
  \item Kad panaudotas elektrochrominis efektas (su elektriniu lauku
    į medžiagą (arba medžiagoje) galime sukurti priemaišinius
    lygmenis, konkrečiai įvedus litį galime keisti nuo violetinės iki
    raudonos).
\item Technines charakteristikas.
\end{remember}

\section{Atminties ląstelės.}

Paprastai integratorių (atminties ląstelių) gamyboje yra naudojami 
argentum superjonikai (TODO iš skaidrės). b paveikslėlyje 2 yra
elektrodai. 4 yra superjonikas. 5 – argentum rezervuaras. 3 – platininis
elektrodas. 1 – izoliatoriai izoliuojantys elektrodus nuo išorinio
korpuso. Pajungus elektrinį lauką prie apatinio elektrinio elektrodo
taip, kad ten būtų $+$, o prie viršutinio – $-$, tai argentum katijonai
per argentum superjoniką migruos į platinos elektrodą, ten prisijungs
elektoną ir nusės ant elektrodo. Kol iš apatinio rezervuaro pasišalins
visi galimi argentum jonai, mes turėsime (jei matuosime potencialą,
prijungtą prie integratoriaus) įrašo kreivę (skaidrėje a), kai
baigsis – potencialas šoktels iki maksimo vertės. Kai įtampa
yra konstanta, tai Būlio algebroje žymi 0, o kitu atveju – 1.

Iš integratorių galime susikonstruoti „ne“, „ir“ bei
„arba“ Būlio algebros veiksmus.

\begin{remember}
  \item Kad naudojami argentum superjonikai.
  \item Struktūrinę schemą.
  \item Jungiant $+$ prie argentum rezervuaro, kol teka srovė, tol yra
    …
  \item Įvairiais jungimo būdais jungdami integratorius galime gauti
    įvairius Būlio algebros veiksmus.
\end{remember}

16 dieną kontrolinis!
