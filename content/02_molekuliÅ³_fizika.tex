\chapter{Molekulių fizika.}

\section{Medžiagos sandara.}

Elektroninės būsenos atome yra nusakomos kvantiniais skaičiais:
\begin{description}
  \item[$n$] – pagrindinis skaičius;
  \item[$l$] – orbitinis skaičius;
  \item[$m_{l}$] – magnetinis orbitinis;
  \item[$m_{s}$] – magnetinis sukinys.
\end{description}
Pagal Paulio principą kiekvienas elektronas yra nusakomas savuoju
kvantiniu skaičių $n, l, m_{l}, m_{s}$ rinkiniu, kuris skiriasi
nuo kito elektrono analogiško rinkinio. Bendras didžiausias
elektroninių būsenų skaičius esant pagrindiniams kvantiniam
skaičiui $n$ yra $i_{max} = 2n^{2}$. $n$ gali būti 1,2,3,4,5,6,7.
Apvalkalų simboliai yra $K, L, M, N, O, P, Q$. Kiekviename
apvalkale elektronai yra išsidėstę pogrupiuose: $s, p, d, f$.
Didžiausias elektronų skaičius pogrupyje yra
$i_{max} = 2(2l + 1), l = 0, 1, 2,\ldots$. Pogrupių simboliai
reiškia: $s$ – sharp, $p$ – principal, $d$ – diffuse, $f$ – fundamental.

Atomai jungiasi į molekules. Atomų masė, išreikšta atominiais masės
vienetais yra vadinama atomine mase.
\begin{equation*}
  1 a.m.v. = 1,66 \cdot 10^{-27} kg
\end{equation*}
Molekulės masė išreikšta atominiais masės vienetais vadinama
molekuline mase. Pavyzdžiui, $N_{2}O$ masė yra lygi
$2\cdot 14 a.m.v + 16 a.m.v = 44 a.m.v.$

Medžiagos kiekis, kurio masė išreikšta gramais ir atitinka jos
molekulinei arba atominei masei, vadinamas medžiagos moliu.
Pavyzdžiui, $C$ – $12\cdot10^{-3} kg/mol$, $O_{2}$ –
$32\cdot10-3 kg/mol$. Viename molyje yra $N_{A}$ dalelių.
$N_{A}$ – Avogadro skaičius, $N_{A} = 6,02 \cdot 10^{23} 1/mol$.
Molekulės masė arba neutrono masė $m_{n} = 1,0089 a.m.v.$, protono
masė $m_{p} = 1,0073 a.m.v.$, elektrono
$m_{e} = 5,485 \cdot 10^{-4} a.m.v.$

Molekulių arba atomų skaičius:
\begin{equation*}
  n = \frac{MN_{A}}{\mu},
\end{equation*}
čia:
\begin{description}
  \item[$\mu$] – medžiagos molinė masė;
  \item[$M$] – medžiagos masė.
\end{description}
Medžiagos tankis:
\begin{equation*}
  \rho = \frac{M}{V}.
\end{equation*}

\begin{remember}
  \item Kas nusako elektronus (kokie keturi parametrai: $n, l, m_{l},
    m_{s}$).
  \item Kaip yra žymimi apvalkalai.
  \item Kas yra $s, p, d, f$.
  \item Kaip galima paskaičiuoti maksimalų elektroninių būsenų skaičių.
  \item Kokie dar vieneta įjungiami nagrinėjant medžiagos sandūrą
    (atominiai masės vienetai).
  \item Kas yra Avogadro skaičius (dalelių skaičius molyje), ką jis nusako.
  \item Molio masė.
  \item Medžiagos tankis.
\end{remember}

\section{Termodinamikos dėsniai.}

Termodinamika neįskaito medžiagos sandūros arba struktūros.

Energija $W = kT$, kur $k$ – Bolcmano konstanta, $T$ – temperatūra
Kelvino skalėje.

Pirmas termodinamikos dėsnis:
\begin{equation*}
  \Delta Q = \Delta U + \Delta A
\end{equation*}

$S$ – entropija, šilumos pokytis esant tam tikrai temperatūrai. Tai yra
procesai vykstantys aplinkoje, keičia šilumos kiekį toje temperatūroje.

Termodinamikos dėsnis kartais vadinamas termodinamikos principu.

Antras termodinamikos dėsnis: TODO


\begin{itemize}
  \item Riša temperatūra, …
  \item Perpetum mobile – amžinasis variklis.
\end{itemize}

Trečiasis termodinamikos dėsnis: TODO

$H$ – entalpija, $H = U + pV$. (vidinė energija + slėgio ir tūrio sandauga)

\section{Kietųjų kūnų savybės.}

Kristaliniai ir amorfiniai. Kristalinės gardelės. Geometrijos rūšys.
Septynios singonijos. Kristaliniuose kūnuose tarp atomų esančios jėgos.
Artiveikės jėgos amorfiniuose kūnuose. Deformacijos. Jungo modulis.
Huko sąryšis.

Tamprioji deformacija – tokia deformacija, kai kūną paveikus jėga, jei
nustojus veikti, kūnas atgauna savo buvusią formą.

Plastinė deformacija – tokia deformacija, kai paveikus kūną išorine jėga,
kūnas nebesugeba atstatyti savo buvusios formos.

Deformacijos gali būti:
\begin{itemize}
  \item spaudimo,
  \item tempimo,
  \item šlyties,
  \item sukimo,
  \item lenkimo.
\end{itemize}

$l_{0}$ – pradinis kūno ilgis.

Absoliuti deformacija: $\Delta l = l - l_{0}$.

Santykinė deformacija: $\varepsilon = \frac{\Delta l}{l_{0}}$.

Vidinė įtampa matuojama paskaliais.

Įsidėmėti:
\begin{itemize}
  \item kokie būna kietieji kūnai (kristaliniai ir amorfiniai);
  \item kristaliniai kūnai yra nusakomi 7 singonijomis;
  \item … elementarioji kristalinė gardelė
  \item kristalinėje gardelėje atomai yra išlaikomi, nes juos veikia
    toliaveikės jėgos;
  \item amorfiniuose kūnuose veikia artiaveikės jėgos;
  \item tamprioji ir plastinė deformacijos;
  \item absoliuti, santykinė deformacijos;
  \item kokios gali būti deformacijos pagal aprašymą;
  \item kas yra Jungo modulis;
  \item kaip keičiasi kūnų matmenys juos šildant;
  \item kas yra linijinio ir kas tūrinio plėtimosi koeficientas.
\end{itemize}

\section{Skysčių savybės.}

Klampumo koeficientas:
\begin{equation*}
  \nu = \frac{Fl}{\Delta \vec{v}s}.
\end{equation*}

\begin{equation*}
  F = \nu \frac{\Delta \vec{v}s}{l}
\end{equation*}

\begin{equation*}
  F = 7 \pi \nu R \vec{v}
\end{equation*}

Skystis slegia į indo sieneles bei dugną:
\begin{equation*}
  P = \frac{F}{s}
\end{equation*}

Atmosferų ir paskalių sąryšys:
\begin{align*}
  1 \t{atmosfera}
    &= \frac{1 \t{jėgos kilogramas}}{1 \t{kvadratinis centimetras}} \\
    1 A = \frac{1 kG}{cm^{2}} &\to \frac{10 N}{10^{-4}m^{2}} \to 10^{5} Pa
\end{align*}

Skysčio paviršius nusakomas paviršiaus įtempimo koeficientu
$\xi' = \frac{F}{l}$, $l$ – paviršiaus kontūro ilgis, $F$ – paviršiaus
įtempimo jėga.

Skysčiai gali šlapinti kūnus arba jų nešlapinti.

Kapiliariškumas – skysčio pakilimas kapiliare.

$r$ – kapiliaro spindulys.

Prisiminti.
\begin{itemize}
  \item Skysčiai turi struktūra. (Sluoksnius.)
  \item Koks pagrindinis parametras nusako skystį (klampumas).
  \item Kaip galime išreikšti klampumą.
  \item Niutono jėga, kuri dominuoja skysčiuose, kai sluoksniai juda.
  \item Kokį eksperimentą atliko Stoksas.
  \item Ką jis surado eksperimentiniu būdu.
  \item Kaip siejasi skysčio paviršiaus įtempimas
  \item Kuo skiriasi drėkinantys ir nedrėkinantys skysčiai.
  \item Kapiliarinis reiškinys.
\end{itemize}

Klampumas. Trinties jėga (Niutono jėga). Kokius eksperimentus yra padaręs
Stoksas. Drėkinantys ir nedrėkinantys skysčiai paviršius. Kapiliarumo
reiškinys.

\section{Dujų savybės.}

\begin{equation*}
  \vec{K}_{1} = m \vec{V}
\end{equation*}
Atsitrenkusios dalelės judėjimo kiekis tampa lygus:
\begin{equation*}
  \vec{K}_{2} = -m \vec{V}
\end{equation*}

Jeigu dalelė per laiką $\Delta \tau$ juda greičiu $\vec{v}$, tai ji
nulėks nuotolį $\Delta \vec{r} = \vec{v} \Delta \tau$.

Čia $\Delta \vec{r}$ – laisvasis lėkio nuotolis, $\Delta \tau$ – laisvasis
lėkio laikas.
Indo tūris $V= \Delta r l^{2} = \Delta r S = S \Delta \tau \vec{v}$.

Jeigu kiekvienmae tūrio elemente yra $n_{0}$ molekulių, tai visame tūryje
turėsime $n = Vn_{0}$. Didžiausia tikimybė šiuo atveju, kad dalelės pasieks
kubo sieneles yra $\frac{1}{6}$, tokiu būdu:
\begin{equation}
  n = \frac{1}{6} \vec{v}S_{\Delta}\tau n_{0}
  \label{eq:dujos}
\end{equation}
Iš TODO sektų:
\begin{equation*}
  |\Delta K | = \vec{K}_{2} - \vec{K}_{1} = - m \vec{v} - m \vec{v} =
  |2m\vec{v}|
\end{equation*}
ir 
\begin{equation*}
  \Delta K = |\Delta K| n = 2m \vec{V} \frac{1}{6} TODO
\end{equation*}
…
Todėl praktiškai reikia vidurkinti:
\begin{equation*}
  <\vec{V}>^{2} = \frac{\vec{v}_{1}^{2} + \cdots \vec{v}^{2}_{n}}{n}
\end{equation*}

Padauginus ir padalijus dešinę lygties pusę iš 2 gausime:
\begin{equation*}
  p = \frac{2}{3} n_{0} \left( \frac{m <\vec{v}>^{2}}{2} \right)
\end{equation*}
TODO

„deg“ yra laipsnis.

$C$ – šiluminė talpa.

Jeigu $<E_{K}> = \frac{m <\vec{v}>^{2}}{2}$, o $n_{0}m = M$, tai…

Santykine oro drėgme $\beta = \frac{\rho'}{\rho_{0}} \cdot 100 \%$.
Drėgmė matuojama psichrometrais. Kalbant apie atmosferą svarbu
žinoti, koks oro molekulių skaičius supančios Žemę jos atmosferos
vienetiniame tūryje. Tai surandame taip:
\begin{align*}
  n
  &= \frac{N_{A}}{V} \\
  &= \frac{6,02 \cdot 10^{23}}{22,4 \cdot 10 ^{-3}} \\
  &= 2,7 \cdot 10 ^{25} m^{-3} \\
\end{align*}

Svarbu.
\begin{itemize}
  \item Kokios gali būti dujos (realios ir idealios).
  \item Kuo jos skiriasi.
  \item Prisiminti Klauzijaus sąryšį: $p = \frac{2}{3} n_{0} <E_{K}>$.
  \item Mendelejevo-Klapeirono lygtį: $pV = \frac{M}{\mu}RT$.
  \item Prisiminti tris būvius(?): izoterminį, izochorinį ir izobarinį.
  \item Klausimo pavyzdys: Šarlio dėsnis.
  \item Visur naudojamos suvidurkintos energijos!
\end{itemize}

Idealios dujos. Atitikmenys tarp realiųjų ir idealiųjų dujų. Nagrinėjant
realiasias, tai yra įskaičiuojamos sąveikos tarp dujų molekulių. Idealiuoju
atveju, tik tarp atomų? Vidutinė kinetinė energija. Sąryšiai tarp
temperatūros ir.
