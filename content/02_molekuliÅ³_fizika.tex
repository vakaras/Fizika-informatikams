\chapter{Molekulių fizika.}

\section{Medžiagos sandara.}

Elektroninės būsenos atome yra nusakomos kvantiniais skaičiais:
\begin{description}
  \item[$n$] – pagrindinis skaičius;
  \item[$l$] – orbitinis skaičius;
  \item[$m_{l}$] – magnetinis orbitinis;
  \item[$m_{s}$] – magnetinis sukinys.
\end{description}
Pagal Paulio principą kiekvienas elektronas yra nusakomas savuoju
kvantiniu skaičių $n, l, m_{l}, m_{s}$ rinkiniu, kuris skiriasi
nuo kito elektrono analogiško rinkinio. Bendras didžiausias
elektroninių būsenų skaičius esant pagrindiniams kvantiniam
skaičiui $n$ yra $i_{max} = 2n^{2}$. $n$ gali būti 1,2,3,4,5,6,7.
Apvalkalų simboliai yra $K, L, M, N, O, P, Q$. Kiekviename
apvalkale elektronai yra išsidėstę pogrupiuose: $s, p, d, f$.
Didžiausias elektronų skaičius pogrupyje yra
$i_{max} = 2(2l + 1), l = 0, 1, 2,\ldots$. Pogrupių simboliai
reiškia: $s$ – sharp, $p$ – principal, $d$ – diffuse, $f$ – fundamental.

Atomai jungiasi į molekules. Atomų masė, išreikšta atominiais masės
vienetais yra vadinama atomine mase.
\begin{equation*}
  1 a.m.v. = 1,66 \cdot 10^{-27} kg
\end{equation*}
Molekulės masė išreikšta atominiais masės vienetais vadinama
molekuline mase. Pavyzdžiui, $N_{2}O$ masė yra lygi
$2\cdot 14 a.m.v + 16 a.m.v = 44 a.m.v.$

Medžiagos kiekis, kurio masė išreikšta gramais ir atitinka jos
molekulinei arba atominei masei, vadinamas medžiagos moliu.
Pavyzdžiui, $C$ – $12\cdot10^{-3} kg/mol$, $O_{2}$ –
$32\cdot10-3 kg/mol$. Viename molyje yra $N_{A}$ dalelių.
$N_{A}$ – Avogadro skaičius, $N_{A} = 6,02 \cdot 10^{23} 1/mol$.
Molekulės masė arba neutrono masė $m_{n} = 1,0089 a.m.v.$, protono
masė $m_{p} = 1,0073 a.m.v.$, elektrono
$m_{e} = 5,485 \cdot 10^{-4} a.m.v.$

Molekulių arba atomų skaičius:
\begin{equation*}
  n = \frac{MN_{A}}{\mu},
\end{equation*}
čia:
\begin{description}
  \item[$\mu$] – medžiagos molinė masė;
  \item[$M$] – medžiagos masė.
\end{description}
Medžiagos tankis:
\begin{equation*}
  \rho = \frac{M}{V}.
\end{equation*}

\begin{remember}
  \item Kas nusako elektronus (kokie keturi parametrai: $n, l, m_{l},
    m_{s}$).
  \item Kaip yra žymimi apvalkalai.
  \item Kas yra $s, p, d, f$.
  \item Kaip galima paskaičiuoti maksimalų elektroninių būsenų skaičių.
  \item Kokie dar vieneta įjungiami nagrinėjant medžiagos sandūrą
    (atominiai masės vienetai).
  \item Kas yra Avogadro skaičius (dalelių skaičius molyje), ką jis nusako.
  \item Molio masė.
  \item Medžiagos tankis.
\end{remember}

\section{Termodinamikos dėsniai.}

Fizikos dalis, nagrinėjanti šiluminius procesus ir kūnų šilumines
savybes, neįskaitant medžiagos vidinės sandaros, vadinama
termodinamika. Chaotinis dalelių judėjimas vadinamas šiluminiu
judėjimu, o energija, kurią įgyja šios dalelės dėl tokio judėjimo,
vadinama šilumine energija. Kai dalelės judėdamos apsikeičia
vienodais energijos kiekiais, vadinama šilumine pusiausvyra.
Jeigu kūnai savo vidinę energiją perduoda vienas kitam neatlikdami
mechaninio darbo, tai yra vadinama šilumine apykaita.

Energija $W = kT$, kur $k$ – Bolcmano konstanta, $T$ – temperatūra
Kelvino skalėje.

$S$ – entropija, šilumos pokytis esant tam tikrai temperatūrai. Tai yra
procesai vykstantys aplinkoje, keičia šilumos kiekį toje temperatūroje.

$U$ – vidinė kūno energija.

$A$ – darbas.

$H$ – entalpija, $H = U + pV$. (vidinė energija + slėgio ir tūrio sandauga)

\begin{defn}[Nulinis termodinamikos dėsnis]
  Jei dvi termodinaminės sistemos yra šiluminėje pusiausvyroje su
  trečia, tai jos taip pat yra šiluminėje pusiausvyroje tarpusavyje.
\end{defn}

\begin{defn}[Pirmasis termodinamikos dėsnis]
  Bet kokio proceso metu bendras energijos kiekis išlieka pastovus.

  Arba kitaip: šilumos kiekis, kurį gauna kūnas šiluminės apykaitos
  būdu yra naudojamas jo vidinės energijos pokyčiui bei mechaninio
  darbo atlikimui. Išreiškus formule:
  \begin{equation*}
    \Delta Q = \Delta U + \Delta A
  \end{equation*}
\end{defn}

\begin{defn}[Antrasis termodinamikos dėsnis]
  Neįmanomas toks procesas, kurio vienintelis rezultatas būtų iš
  šildytuvo gautos šilumos pavertimas jai ekvivalentišku darbu.

  Kitaip: bet kurios uždaros sistemos, neesančios terminėje pusiausvyroje,
  entropija beveik visada didėja.

  \begin{equation*}
    dS = \frac{\delta Q}{T}
  \end{equation*}
\end{defn}

\begin{defn}[Trečiasis termodinamikos dėsnis]
  Temperatūrai artėjant prie absoliutaus nulio, sistemos entropija
  tampa pastovi.

  \begin{equation*}
    \Delta S_{T=0} = 0
  \end{equation*}
\end{defn}

Šiluminė apykaita yra susijusi su šiluminiu laidumu. Šiluminio
laidumo lygtis:
\begin{equation*}
  Q = \chi\frac{S \Delta t \tau}{d},
\end{equation*}
čia:
\begin{description}
  \item[$\chi$] – šiluminio laidumo koeficientas
    (matuojamas $\frac{W}{m \t{deg}}$);
  \item[$S$] – plotas;
  \item[$\tau$] – laikas;
  \item[$\Delta t$] – temperatūrų skirtumas;
  \item[$d$] – pernešamos medžiagos storis.
\end{description}

Svarbus parametras, kurio dėka nusakoma šilumos apykaita yra
savitoji šiluminė talpa:
\begin{equation*}
  c = \frac{Q}{m\Delta t} \left[ \frac{J}{kg\cdot\t{deg}} \right]
\end{equation*}
SI sistemos savituoju šiluminės talpos vienetu laikoma tokia kūno
šiluminė talpa, kurio $1 kg$ masės sugeria arba išspinduliuoja
$1 J$ energiją, jo temperatūrai pakitus 1 laipsniu.

Šilumos balanso lygtis:
\begin{equation*}
  \sum _{i=0} ^{N} Q_{i} = 0,
\end{equation*}
čia $Q_{i}$ yra $i$-ojo kūno gautas arba atiduotas šilumos kiekis.

Savitoji medžiagos degimo šiluma:
\begin{equation*}
  \lambda = \frac{Q}{m} \left[ \frac{J}{kg} \right].
\end{equation*}
SI sistemoje kaloringumo vienetu laikomas toks medžiagos kaloringumas,
kai pilnai sudegusios 1 kg medžiagos masė išskiria 1 J energiją.

\section{Kietųjų kūnų savybės.}

Kietieji kūnai, kurių atomai, molekulės ar jonai erdvėje yra
išsidėstę tvarkingai ir sudaro periodiškai pasikartojančią vidinę
struktūrą, vadinami kristalais. Kristalai gali turėti įvairių
prizmių ar piramidžių formą, tačiau jų pagrindą būtinai sudaro
taisyklingasis trikampis, kvadratas, lygiagretainis arba šešiakampis.

Svarbiausios kietų kristalinių kūnų savybės:
\begin{itemize}
  \item taisyklinga forma;
  \item anizotropija (kristalų fizinių savybių priklausomybė nuo
    pasirinktos krypties kristale);
  \item pastovi lydymosi temperatūra.
\end{itemize}

TODO: Kas yra kristaliniai, o kas yra amorfiniai kūnai. Dėl kokių
jėgų veikimo kūnai išlaiko kietą formą?

Visos kristalinės struktūros yra grupuojamos 7 singonijose bei
yra žinomos 32 jų taškinės simetrijos grupės. Singonijos:
\begin{enumerate}
  \item triklininė;
  \item monoklininė;
  \item rombinė;
  \item tetragoninė;
  \item trigoninė;
  \item heksagoninė;
  \item kūbinė.
\end{enumerate}

Visos 32 taškinės grupės turi tarptautinė žymėjimą bei žymėjimus
pagal Šubnikovą ir Šemflisą. Kiekviena taškinė grupė skirstoma
į simetrijos klases ir kiekviena klasė turi savo simbolį.
Visos taškinės simetrijos grupės yra skirstomos į $K^{I}$ ir $K^{II}$
atitinkamai pirmąsias ir antrąsias rūšis. Iš viso yra
$K^{I}$ – 11 ir $K^{II}$ – 21 grupė. $K^{I}$ rūšiai priklauso
taškinės grupės, kuriose yra tik posūkio ašys, o $K^{II}$
rūšiai priskiriamos grupės, kuriose yra inversijos bei atspindžio
centrai.

Visos kristalinės gardelės sugrupuojamos į 14 Bravės gardelių.

Gardelė, turinti tiek taškinės simetrijos operaciją, tiek poslinkius,
vadinama Bravės grupe, o begalinė gardelė, gauta iš Bravės grupės,
vadinama Bravės gardele.

Kristalinės gardelės geometrija nusakoma gardelės parametrų a, b, c
bei geometrinės figūros kampais.

Tamprioji deformacija – tokia deformacija, kai kūną paveikus jėga, jei
nustojus veikti, kūnas atgauna savo buvusią formą.

Plastinė deformacija – tokia deformacija, kai paveikus kūną išorine jėga,
kūnas nebesugeba atstatyti savo buvusios formos.

Deformacijos gali būti:
\begin{itemize}
  \item spaudimo,
  \item tempimo,
  \item šlyties (kampinė deformacija),
  \item sukimo,
  \item lenkimo.
\end{itemize}

Taip pat deformacijos gali būti:
\begin{itemize}
  \item absoliučios: $\Delta l = l - l_{0}$ ir
  \item santykinės: $\varepsilon = \frac{\Delta l}{l_{0}}$.
\end{itemize}
Čia $l_{0}$ yra pradinis kūno ilgis.

Atliekant deformaciją, kūnas priešinasi deformuojančiai jėgai. Ši
savybė vadinama kūnų vidine įtampa:
\begin{equation*}
  \sigma = \frac{F}{S}
\end{equation*}
Vidinė įtampa matuojama paskaliais ($1 Pa = 1 \frac{N}{m^{2}}$).

Esant spaudimo arba tempimo deformacijai:
\begin{equation*}
  \sigma = E \varepsilon = E \frac{\Delta l}{l_{0}},
\end{equation*}
čia $E$ – Jungo modulis. Jeigu $\Delta l = l_{0}$, tai $E = \sigma$.
Tokiu būdu Jungo modulis rodo tokį medžiagos mechaninį tempimą, kai
jos matmenys padidėja dvigubai.

Kūnai taip pat keičia savo matmenis juos šildant arba šaldant. Keičiasi
jų linijiniai matmenys:
\begin{equation*}
  l = l_{0}(1 + \alpha t),
\end{equation*}
čia:
\begin{description}
  \item[$\alpha$] – linijinio plėtimosi koeficientas
    ($\alpha = \frac{\Delta l}{l_{0} \Delta t}$);
  \item[$t$] – kūno temperatūra.
\end{description}
Kūnų tūris keičiasi pagal dėsnį:
\begin{equation*}
  V = V_{0}\left( 1 + \beta t \right),
\end{equation*}
čia $\beta \approx 3 \alpha$.

Šildant kietas medžiagas iki tam tikros temperatūros jos lydosi.
Medžiagų lydymasis yra nusakomas savitąja lydymosi šiluma:
\begin{equation*}
  q = \frac{Q}{m},
\end{equation*}
kuri yra matuojama $\frac{J}{kg}$.

\begin{remember}
  \item Kokie būna kietieji kūnai (kristaliniai ir amorfiniai).
  \item Kristaliniai kūnai yra nusakomi 7 singonijomis.
  \item … elementarioji kristalinė gardelė.
  \item Kristalinėje gardelėje atomai yra išlaikomi, nes juos veikia
    toliaveikės jėgos.
  \item Amorfiniuose kūnuose veikia artiaveikės jėgos.
  \item Tamprioji ir plastinė deformacijos.
  \item Absoliuti, santykinė deformacijos.
  \item Kokios gali būti deformacijos pagal aprašymą.
  \item Kas yra Jungo modulis.
  \item Kaip keičiasi kūnų matmenys juos šildant.
  \item Kas yra linijinio ir kas tūrinio plėtimosi koeficientas.
\end{remember}

\section{Skysčių savybės.}

Klampumo koeficientas:
\begin{equation*}
  \nu = \frac{Fl}{\Delta \vec{v}s}.
\end{equation*}

\begin{equation*}
  F = \nu \frac{\Delta \vec{v}s}{l}
\end{equation*}

\begin{equation*}
  F = 7 \pi \nu R \vec{v}
\end{equation*}

Skystis slegia į indo sieneles bei dugną:
\begin{equation*}
  P = \frac{F}{s}
\end{equation*}

Atmosferų ir paskalių sąryšys:
\begin{align*}
  1 \t{atmosfera}
    &= \frac{1 \t{jėgos kilogramas}}{1 \t{kvadratinis centimetras}} \\
    1 A = \frac{1 kG}{cm^{2}} &\to \frac{10 N}{10^{-4}m^{2}} \to 10^{5} Pa
\end{align*}

Skysčio paviršius nusakomas paviršiaus įtempimo koeficientu
$\xi' = \frac{F}{l}$, $l$ – paviršiaus kontūro ilgis, $F$ – paviršiaus
įtempimo jėga.

Skysčiai gali šlapinti kūnus arba jų nešlapinti.

Kapiliariškumas – skysčio pakilimas kapiliare.

$r$ – kapiliaro spindulys.

Prisiminti.
\begin{itemize}
  \item Skysčiai turi struktūra. (Sluoksnius.)
  \item Koks pagrindinis parametras nusako skystį (klampumas).
  \item Kaip galime išreikšti klampumą.
  \item Niutono jėga, kuri dominuoja skysčiuose, kai sluoksniai juda.
  \item Kokį eksperimentą atliko Stoksas.
  \item Ką jis surado eksperimentiniu būdu.
  \item Kaip siejasi skysčio paviršiaus įtempimas
  \item Kuo skiriasi drėkinantys ir nedrėkinantys skysčiai.
  \item Kapiliarinis reiškinys.
\end{itemize}

Klampumas. Trinties jėga (Niutono jėga). Kokius eksperimentus yra padaręs
Stoksas. Drėkinantys ir nedrėkinantys skysčiai paviršius. Kapiliarumo
reiškinys.

\section{Dujų savybės.}

\begin{equation*}
  \vec{K}_{1} = m \vec{V}
\end{equation*}
Atsitrenkusios dalelės judėjimo kiekis tampa lygus:
\begin{equation*}
  \vec{K}_{2} = -m \vec{V}
\end{equation*}

Jeigu dalelė per laiką $\Delta \tau$ juda greičiu $\vec{v}$, tai ji
nulėks nuotolį $\Delta \vec{r} = \vec{v} \Delta \tau$.

Čia $\Delta \vec{r}$ – laisvasis lėkio nuotolis, $\Delta \tau$ – laisvasis
lėkio laikas.
Indo tūris $V= \Delta r l^{2} = \Delta r S = S \Delta \tau \vec{v}$.

Jeigu kiekvienmae tūrio elemente yra $n_{0}$ molekulių, tai visame tūryje
turėsime $n = Vn_{0}$. Didžiausia tikimybė šiuo atveju, kad dalelės pasieks
kubo sieneles yra $\frac{1}{6}$, tokiu būdu:
\begin{equation}
  n = \frac{1}{6} \vec{v}S_{\Delta}\tau n_{0}
  \label{eq:dujos}
\end{equation}
Iš TODO sektų:
\begin{equation*}
  |\Delta K | = \vec{K}_{2} - \vec{K}_{1} = - m \vec{v} - m \vec{v} =
  |2m\vec{v}|
\end{equation*}
ir 
\begin{equation*}
  \Delta K = |\Delta K| n = 2m \vec{V} \frac{1}{6} TODO
\end{equation*}
…
Todėl praktiškai reikia vidurkinti:
\begin{equation*}
  <\vec{V}>^{2} = \frac{\vec{v}_{1}^{2} + \cdots \vec{v}^{2}_{n}}{n}
\end{equation*}

Padauginus ir padalijus dešinę lygties pusę iš 2 gausime:
\begin{equation*}
  p = \frac{2}{3} n_{0} \left( \frac{m <\vec{v}>^{2}}{2} \right)
\end{equation*}
TODO

„deg“ yra laipsnis.

$C$ – šiluminė talpa.

Jeigu $<E_{K}> = \frac{m <\vec{v}>^{2}}{2}$, o $n_{0}m = M$, tai…

Santykine oro drėgme $\beta = \frac{\rho'}{\rho_{0}} \cdot 100 \%$.
Drėgmė matuojama psichrometrais. Kalbant apie atmosferą svarbu
žinoti, koks oro molekulių skaičius supančios Žemę jos atmosferos
vienetiniame tūryje. Tai surandame taip:
\begin{align*}
  n
  &= \frac{N_{A}}{V} \\
  &= \frac{6,02 \cdot 10^{23}}{22,4 \cdot 10 ^{-3}} \\
  &= 2,7 \cdot 10 ^{25} m^{-3} \\
\end{align*}

Svarbu.
\begin{itemize}
  \item Kokios gali būti dujos (realios ir idealios).
  \item Kuo jos skiriasi.
  \item Prisiminti Klauzijaus sąryšį: $p = \frac{2}{3} n_{0} <E_{K}>$.
  \item Mendelejevo-Klapeirono lygtį: $pV = \frac{M}{\mu}RT$.
  \item Prisiminti tris būvius(?): izoterminį, izochorinį ir izobarinį.
  \item Klausimo pavyzdys: Šarlio dėsnis.
  \item Visur naudojamos suvidurkintos energijos!
\end{itemize}

Idealios dujos. Atitikmenys tarp realiųjų ir idealiųjų dujų. Nagrinėjant
realiasias, tai yra įskaičiuojamos sąveikos tarp dujų molekulių. Idealiuoju
atveju, tik tarp atomų? Vidutinė kinetinė energija. Sąryšiai tarp
temperatūros ir.
