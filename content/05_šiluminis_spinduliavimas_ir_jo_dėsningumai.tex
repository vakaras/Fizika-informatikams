\chapter{Šiluminis spinduliavimas ir jo dėsningumai.}

% Išdėstyta: 2012-05-02

\slide{126}

Visi kūnai, kurie yra gamtoje spinduliuoja elektromagnetines bangas.
Nepriklausomai nuo to ar mes matome tai, ar ne. Spinduliavimai
priklausomai nuo priežasties vadinami:
\begin{itemize}
  \item šiluminiu spinduliavimu;
  \item chemiliuminescencija;
  \item elektroliuminescencija;
  \item …
\end{itemize}

…

Energetinis kūno švytėjimas:
\begin{equation*}
  R_{\omega} = r_{\omega} d \omega,
\end{equation*}
čia $r_{\omega}$ yra temperatūros funkcija.

Kirchhofo dėsnis sako, kad spinduliuotės ir sugerties gebų santykis
nepriklauso nuo medžiagos.

\slide{128}
Sugerties geba skiria dvi medžiagų rūšis:
\begin{itemize}
  \item absoliučiai juodo kūno $\alpha = 1$;
  \item o visi kiti – pilki.
\end{itemize}

\slide{129}
Stefano Bolcmano dėsnis

Stefonas sukūrė kolbą, kuri iš vidaus buvo išklijuota juoduoju aksomu.
Į kolbą buvo nukreipta šiluminė spinduliuotė, pro angą. Prie išėjimo
buvo nustatyti indikatoriai, kurie matavo išeinančią energiją.
Jau tuo atveju, Stefanas nebegalėjo su to laikmečio įranga
užfiksuoti išeinančio srauto. Po penkių metų su panašia įrangą
eksperimentą pakartojo Bolcmanas. Po daugelio eksperimentų jie
nustatė, kad toks įrenginys prilygsta absoliučiai juodam kūnui.
Energinis švytėjimas gali būti išreikštas:
\begin{equation*}
  R^{*} = \int _{0} ^{\infty} f (\omega T) = \sigma T^{4},
\end{equation*}
čia $T$ – spinduliuojančio kūno temperatūra.

\begin{remember}
  \item Stefanas sukonstravo įrenginį prilygstantį absoliučiai juodam
    kūnui.
  \item Leisdamas šiluminę spinduliuotę į tokį kūną, jis nebefiksavo
    išeities. Kitaip tariant sugertis buvo absoliuti.
  \item Po to eksperimentą pakartojo Bolcmanas ir jie kartu sulyginę
    savo rezultatus sukonstravo formulę, kur $\sigma$ yra konstanta
    nustatyta iš eksperimentinių faktų.
\end{remember}

Šiluminė spinduliuotė - Vino poslinkis. \slide{130}
$b$ – Vino konstanta.

\slide{131}
Spinduliuotės geba yra dažnio ir temperatūros funkcija.

\begin{remember}
  \item Kad Vinas irgi atliko eksperimentinius tyrimus.
  \item Tyrė, kaip šilumnės spinduliuotės geba priklauso nuo dažnio esant
    skirtingoms temperatūroms.
  \item Kad nustatė, jog žemėjant temperatūrai maksimumas, kuris atitinka
    tam tikrą bangos ilgį, slenka į ilgųjų bangų pusę. (Prisiminti
    kreives.)
  \item Nustatė, kad $T\Lambda_{max} = b$, kur $b$ – Vino konstanta.
\end{remember}

\slide{132}
Relėjus ir Džinsas užrašė energijos tankio funkciją.

Plankui pavyko rasti energijos tankio funkcijos išraišką konstantos
tikslumu.

\begin{remember}
  \item Pateikė funkcijai išraišką.
  \item Iš jų skaičiavimų galima paskaičiuoti energijos tankį.
  \item Plankas parodė, kad energiją galima kvantuoti?
\end{remember}
