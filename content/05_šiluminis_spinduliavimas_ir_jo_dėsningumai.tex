\chapter{Šiluminis spinduliavimas ir jo dėsningumai.}

% Išdėstyta: 2012-05-02

\section{Elektromagnetinių spinduliavimų rūšys}

\slide{126}

Visi kūnai, kurie yra gamtoje spinduliuoja elektromagnetines bangas.
Nepriklausomai nuo to ar mes matome tai, ar ne. Spinduliavimai
priklausomai nuo priežasties vadinami:
\begin{itemize}
  \item šiluminiu spinduliavimu – spinduliavimą sąlygoja temperatūra;
  \item chemiliuminescencija – cheminės reakcijos;
  \item elektroliuminescencija – reiškiniai vykstantys tekant elektros
    srovei;
  \item katodinė liuminescencija – bombarduojant medžiagas elektronais;
  \item fotoliuminescencija – vykstant šviesos bangų sugerčiai.
\end{itemize}

\section{Šiluminis spinduliavimas}

Šiluminis spinduliavimas yra pusiausvyrinis procesas. Priklausomai
nuo temperatūros, kūnai gali spinduliuoti įvairaus ilgio
elektromagnetines bangas. Kad kūnas spinduliuotų ir nekistų jo
temperatūra, jo išspinduliuojamas energijos kiekis turi sutapti
su sugeriamos energijos kiekiu.

Energetinis kūno švytėjimas $R$ – tai energijos srautas, kurį
spinduliuoja vienetinis kūno ploto vienetas, aprėžtas erdviniu
$2\pi$ kampu. Kūnas spinduliuoja visą spektrą bangų ilgių ($\lambda$),
arba kitaip tariant jų dažnių ($\omega$):
\begin{equation*}
  d R_{\omega} = r_{\omega} d \omega,
\end{equation*}
čia $r_{\omega}$ yra spinduliavimo geba, kuri yra temperatūros funkcija.

$R_{\omega T}$ išraišką gauname integruodami pagal dažnį:
\begin{align*}
  R_{\omega T}
  &= \int d R_{\omega T} \\
  &= \int _{0} ^{\infty} r_{\omega T} d \omega.
\end{align*}
Išspinduliuotos bangos ilgį $\lambda$ su jos dažniu $\omega$ sieja
ryšiai:
\begin{align*}
  \lambda &= \frac{2 \pi c}{\omega} \\
  d \lambda
  &= - \frac{2 \pi c}{\omega^{2}}d \omega \\
  &= - \frac{\lambda^{2}}{2 \pi c} d \omega \\
\end{align*}
čia „$-$“ parodo tik tai, kad $\omega$ didėjant $\lambda$ mažėja ir
atvirkščiai.

\section{Spinduliavimo sugertis}

Kitas parametras, nusakantis spinduliavimą yra sugerties geba:
\begin{equation*}
  \alpha_{\omega T} = \frac{d \phi'_{\omega}}{d\phi},
\end{equation*}
čia:
\begin{description}
  \item[$d \phi'_{\omega}$] – energijos srautas, sugeriamas elementariuoju
    kūno paviršiaus plotu;
  \item[$d \phi_{\omega}$] – krentantis į šį plotą energijos srautas.
\end{description}

\section{Kirchofo dėsnis}

Kirchhofo dėsnis\footnote{
\url{http://en.wikipedia.org/wiki/Kirchhoff\%27s_law_of_thermal_radiation}}
teigia, kad spinduliuotės ir sugerties gebų santykis nepriklauso nuo
medžiagos (kitaip tariant nuo kūno prigimties):
\begin{equation*}
  \left( \frac{r_{\omega T}}{\alpha_{\omega T}} \right)_{1} =
  \left( \frac{r_{\omega T}}{\alpha_{\omega T}} \right)_{2} =
  \left( \frac{r_{\omega T}}{\alpha_{\omega T}} \right)_{3} = \cdots
\end{equation*}

Kirchofas taip pat nurodė, kad:
\begin{align*}
  r _{\omega T}
  &= f(\omega, T) \\
  &= \frac{\lambda^{2}}{2 \pi c} \varphi(\lambda, T), \\
\end{align*}
čia $\varphi$ yra nežinoma spinduliavimo gebos paskirstymo funkcija.

\section{Absoliučiai juodas kūnas}

\slide{128}
Pagal sugerties gabą kūnus galima skirstyti į dvi grupes:
\begin{itemize}
  \item absoliučiai juodus kūnus, kurių $\alpha_{\omega T} = 1$;
  \item pilkus (visi kiti).
\end{itemize}
Absoliučiai juodas kūnas yra fizikinė apbstrakcija, reiškianti
kūną, kuris sugeria visas į jį krentančias elektromagnetines bangas
bei bangų neatspindi. Nepaisant to, absoliučiai juodas kūnas gali
spinduliuoti elektromagnetines bangas ir turėti savo spalvą. Jo
spinduliavimo spektrą apsprendžia vienintelis parametras –
temperatūra, dėl to kartais jį siūloma vadinti idealiuoju
spinduliu.\footnote{
\url{http://lt.wikipedia.org/wiki/Absoliučiai_juodas_kūnas}}

Absoliučiai juodų kūnų gamtoje nėra, tačiau galima sukonstruoti
įrenginį, kurio $\alpha_{\omega T} \cong 1$. Jo paveikslėlis pateiktas
\slide{129}. Stefanas sukūrė kolbą, kuri iš vidaus buvo išklijuota
juoduoju aksomu. Į kolbą, pro angą, buvo nukreipta šiluminė
spinduliuotė. Prie išėjimo buvo nustatyti indikatoriai,
kurie matavo išeinančią energiją. Jau tuo atveju, Stefanas
nebegalėjo su to laikmečio įranga užfiksuoti išeinančio srauto. Po
penkių metų su panašia įranga eksperimentą pakartojo Bolcmanas. Po
daugelio eksperimentų jie nustatė, kad toks įrenginys prilygsta
absoliučiai juodam kūnui. Jie eksperimentiškai nustatė, kad 
energija išspinduliuojama absoliučiai juodo kūno paviršiaus
vieneto yra:
\begin{equation*}
  R^{*} = \sigma T^{4},
\end{equation*}
čia $T$ – spinduliuojančio kūno temperatūra, o
$\sigma (\sigma = 5,7 \cdot 10^{-8}\frac{W}{m^{2}K^{4}})$ yra 
Stefano-Bolcmano konstanta.

\begin{remember}
  \item Stefanas sukonstravo įrenginį prilygstantį absoliučiai juodam
    kūnui.
  \item Leisdamas šiluminę spinduliuotę į tokį kūną, jis nebefiksavo
    atspindžio. Kitaip tariant sugertis buvo absoliuti.
  \item Po to eksperimentą pakartojo Bolcmanas ir jie kartu palyginę
    savo rezultatus sukonstravo formulę $R^{*} = \sigma T^{4}$, kur
    $\sigma$ yra konstanta nustatyta iš eksperimentinių faktų.
\end{remember}

\section{Vino poslinkis}

Plačiau: \url{http://en.wikipedia.org/wiki/Wien\%27s_displacement_law}.

1893 metais Vinas parodė, kad spektrinio pasiskirstymo funkcija
\begin{equation*}
  f(\omega, T) = c \omega^{3} F \left( \frac{\omega}{T} \right),
\end{equation*}
o tuo tarpu
\begin{align*}
  \varphi(\lambda, T)
  &= \frac{2 \pi c}{\lambda^{2}} f(\omega(\lambda), T) \\
  &= \frac{2 \pi c}{\lambda^{2}}
    f\left( \frac{2\pi c}{\lambda}, T \right) \\
  &= \frac{2 \pi c}{\lambda^{2}} \left( \frac{2 \pi c}{\lambda} \right)^{3}
    F\left( \frac{2\pi c}{\lambda} \right) \\
  &= \frac{1}{\lambda^{5}} \psi\left( \lambda T \right)
\end{align*}
Pastaroji lygtis rodo sąryšį tarp bangos ilgio, kuriam tenka
sugerties $\phi(\lambda, T)$ maksimumas ir temperatūros.

Vino poslinkis yra sąryšis tarp absoliučiai juodo kūno temperatūros ir
bangos ilgio, ties kuriuo yra spinduliavimo maksimumas:
\begin{equation*}
  T\cdot \lambda_{max} = b,
\end{equation*}
čia $b = 2,898 \cdot 10^{-3} m\cdot K$. Išsireiškę $\lambda_{max}$
gauname:
\begin{equation*}
  \lambda_{max} = \frac{b}{T},
\end{equation*}
iš kurios matome, kad žemėjant temperatūrai $\lambda_{max}$ slenka
į ilgųjų bangų pusę.

\begin{remember}
  \item Kad Vinas irgi atliko eksperimentinius tyrimus.
  \item Tyrė, kaip šilumnės spinduliuotės geba priklauso nuo dažnio esant
    skirtingoms temperatūroms.
  \item Kad nustatė, jog žemėjant temperatūrai maksimumas, kuris atitinka
    tam tikrą bangos ilgį, slenka į ilgųjų bangų pusę. (Prisiminti
    kreives.)
  \item Nustatė, kad $T\lambda_{max} = b$, kur $b$ – Vino konstanta.
\end{remember}

\section{Relėjaus ir Džinso tyrimai}

\slide{132}

Relėjus ir Džinsas užrašė energijos tankio funkciją:
\begin{equation*}
  f(\omega, T) = \frac{\omega^{2}}{4 \pi^{2}c^{2}}kT.
\end{equation*}
Ši išraiška neprieštarauja tam, ką pavyko atrasti Vinui. Taip, pat
iš šios formulės sektų, kad pusiausvyrinis spinduliavimo energijos
tankis yra:
\begin{equation*}
  u(\omega, T) = \frac{\omega^{2}}{\pi^{2}c^{2}}k T,
\end{equation*}
čia $k$ yra Bolcmano konstanta.

1900 metais Plankui energijos tankio funkcijos $u(\omega, T)$
išraišką pavyko rasti konstantos tikslumu. Jis padarė prielaidą,
kad spinduliavimas vyksta kvantais su energija:
\begin{equation*}
  \varepsilon = \hbar \omega,
\end{equation*}
čia $\hbar = \frac{h}{2\pi}$, o $h = 6,62 \cdot 10^{-34} J\cdot s^{-1}$.

\begin{remember}
  \item Pateikė funkcijai išraišką.
  \item Iš jų skaičiavimų galima paskaičiuoti energijos tankį.
  \item Plankas parodė, kad energiją galima kvantuoti?
\end{remember}
