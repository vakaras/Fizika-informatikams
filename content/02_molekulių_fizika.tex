\chapter{Molekulių fizika.}
\section{Medžiagos sandara.}
Atomai, molekulės.
\section{Termodinamikos dėsniai.}
\begin{itemize}
  \item Riša temperatūra, …
  \item Perpetum mobile.
\end{itemize}<++>
\section{Kietųjų kūnų savybės.}
Kristaliniai ir amorfiniai. Kristalinės gardelės. Geometrijos rūšys.
Septynios simbolinės? Kristaliniuose kūnuose tarp atomų esančios jėgos.
Artiveikės jėgos amorfiniuose kūnuose. Deformacijos. Jungo modulis.
Huko sąryšis.
\section{Skysčių savybės.}
Klampumas. Trinties jėga (Niutono jėga). Kokius eksperimentus yra padaręs
Stoksas. Drėkinantys ir nedrėkinantys skysčiai paviršius. Kapiliarumo
reiškinys.
\section{Dujų savybės.}
Idealios dujos. Atitikmenys tarp realiųjų ir idealiųjų dujų. Nagrinėjant
realiasias, tai yra įskaičiuojamos sąveikos tarp dujų molekulių. Idealiuoju
atveju, tik tarp atomų? Vidutinė kinetinė energija. Sąryšiai tarp
temperatūros ir.
