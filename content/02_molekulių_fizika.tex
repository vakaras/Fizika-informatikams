\chapter{Molekulių fizika.}
\section{Medžiagos sandara.}
Atomai, molekulės.

Prisiminti:
\begin{itemize}
  \item kas nusako elektronus (kokie keturi parametrai: $n, l, m_{l},
    m_{s}$);
  \item kaip yra žymimi apvalkalai;
  \item kas yra s, p, d, f;
  \item kaip galima paskaičiuoti maksimalų elektroninių būsenų skaičių;
  \item kokie dar vieneta įjungiami nagrinėjant medžiagos sandūrą
    (atominiai masės vienetai);
  \item kas yra Avogadro skaičius (dalelių skaičius molyje), ką jis nusako;
  \item molio masė;
  \item medžiagos tankis.
\end{itemize}

\section{Termodinamikos dėsniai.}

Termodinamika neįskaito medžiagos sandūros arba struktūros.

Energija $W = kT$, kur $k$ – Bolcmano konstanta, $T$ – temperatūra
Kelvino skalėje.

Pirmas termodinamikos dėsnis:
\begin{equation*}
  \Delta Q = \Delta U + \Delta A
\end{equation*}

$S$ – entropija, šilumos pokytis esant tam tikrai temperatūrai. Tai yra
procesai vykstantys aplinkoje, keičia šilumos kiekį toje temperatūroje.

Termodinamikos dėsnis kartais vadinamas termodinamikos principu.

Antras termodinamikos dėsnis: TODO


\begin{itemize}
  \item Riša temperatūra, …
  \item Perpetum mobile – amžinasis variklis.
\end{itemize}

Trečiasis termodinamikos dėsnis: TODO

$H$ – entalpija, $H = U + pV$. (vidinė energija + slėgio ir tūrio sandauga)

\section{Kietųjų kūnų savybės.}
Kristaliniai ir amorfiniai. Kristalinės gardelės. Geometrijos rūšys.
Septynios simbolinės? Kristaliniuose kūnuose tarp atomų esančios jėgos.
Artiveikės jėgos amorfiniuose kūnuose. Deformacijos. Jungo modulis.
Huko sąryšis.
\section{Skysčių savybės.}
Klampumas. Trinties jėga (Niutono jėga). Kokius eksperimentus yra padaręs
Stoksas. Drėkinantys ir nedrėkinantys skysčiai paviršius. Kapiliarumo
reiškinys.
\section{Dujų savybės.}
Idealios dujos. Atitikmenys tarp realiųjų ir idealiųjų dujų. Nagrinėjant
realiasias, tai yra įskaičiuojamos sąveikos tarp dujų molekulių. Idealiuoju
atveju, tik tarp atomų? Vidutinė kinetinė energija. Sąryšiai tarp
temperatūros ir.
