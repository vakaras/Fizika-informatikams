\chapter{Atomo ir branduolio fizikos pagrindiniai dėsningumai ir jų
sąryšiai.}

% Išdėstyta: 2012-05-09

\section{Branduolio skilimas bei sintezė.}

Atomą sudaro branduolys ir apie pagal tam tikrą tikimybę išsidėstę
apvalkaluose elektronai. Pagrindinius teorinius pamatus atomo
fizikos moksle yra suformulavęs Nilsas Boras 1913 metais. Jis
pateikė labai aiškia ir suprantamą ir iki šiol ne… toriją, kuri
yra apjungta trim boro postulatais. Už juos jis gavo Nobelio premiją.
Jo sūnus irgi yra Nobelio premijos lauretas.

Pirmasis postulatas \slide{136}. Energija pažymėta taip pat, kaip ir
jo darbe $\varepsilon$, o mes labiau įpratę žymėti $E$.

Antrasis boro postulatas \slide{136}. Jei orbita yra apskritiminė,
tai jos ilgis yra $2 \pi R $.

Trečiasis boro postulatas \slide{136}.

\slide{137} $z$ – eilės numeris.

\slide{138}.
Yra vidinis ir išorinis fotoefektai. Už šiuos darbus Enšteinas gavo
Nobelio premiją. (Jis jos negavo už reliatyvumo teoriją.)

\slide{140} $d = 2,8 A^{\frac{1}{3}} \cdot 10 ^{-15}m$

Jei branduolį sudaro atitinkamas protonų skaičius, tai yra daug
atomų kurie turi skirtingas atomines mases. Tokie branduoliai vadinami
izotopais. Vandenilis turi 3 izotopus. O už uraniniai gali turėti
daug izotopų.
Nuklonas – protonas arba neutronas. $M$ – branduolio masė.

\slide{142} Šis reiškinys yra vadinamas masės defektu. Jį paaiškina
Einšteino teorija ir formulė $\Delta E = \Delta m c^{2}$. Kokia
yra fizikinė prasmė kalbant apie šią labai populiarią Einšteino
formulę. Jei turime $E = mc^{2}$, tai reiškia, kad masė yra tapati
energijai. Jei turime $\Delta E = \Delta mc^{2}$ tai reiškia, kad
sintezuojant branduolius dalis masės dingsta dėl elekromagnetinės
spinduliuotės. 

\slide{143} Formulė irgi plaukia iš Boro postulatų, ji yra empirinė.
$\Delta W$ – ryšio energija. Kalbant apie skilimus įvedamos
tokios sąvokos: motininis atomo branduolys, jam skylant atsiranda
dukterinis atomo branduolys ir priklausomai nuo to kokios yra skilimo
reakcijos, skilimo produktai gali būti įvairūs. Yra išskirtos
tokios branduolio skilimo grupės:
\begin{itemize}
  \item $\alpha$, $\beta^{-}$, $\beta^{-}$, $\gamma$ spinduliuotė.
\end{itemize}
Skilimai gali būti tiek natūralūs, tiek dirbtiniai.

\slide{144}
N parodo kiek suskilo branduolių po tam tikro laiko. $N_{0}$ startinis
skylančių branduolių kiekis. Laikas per kurį skyla pusė atomo branduolių
yra vadinamas skilimo pusperiodžiu.

\slide{145}
$\alpha$ skilimas.
$X$ – motininis atomo branduolys. Čia yra du žymėjimai, kai rašoma
iš kairės, tai yra amerikiečių rašymo stilius. Kai rašoma iš dešinės
yra kitų stilius. $Y$ – dukterinis atomo branduolys. Produktas yra
helio atomo branduolys. Kitaip sakant $\alpha$ dalelės yra helio
atomo branduoliai. Pavyzdys: skylant uranui susidaro toras ir
$\alpha$ dalelė. $\alpha$ dalelių masės yra gan didelės, tai toks
skilimo produktas greitai slopsta.

\slide{146}
$\beta^{-}.$ 
$\beta^{-}$ dalelės yra elektronai, arba. 
$X \to Y + e + \~{\nu}$. Produktas yra elektronas ir antineutrinas,
tai yra neutrino anti dalelė.

$\beta^{+}$ pozitroninis skilimas. Kai motininis atomo branduolys
skyla, tai atsiranda pozitronas $e$ ir neutrinas. Neutrinų masė
yra apie $5 eV$, jie sugeba perskrosti Žemės rutulį praktiškai
neprarasdami energijos. Jų masė yra iki $100$ karto mažesnė už
elektrono. Yra galvojama apie informacijos perdavimą panaudojant
neutrinus. Pozitronas yra elektrono antidalelė.
Yra specialus atvejis – elektroninis pagavimas: kai motininis
branduolys pasigauna pozitroną.

Vykstant branduoliniam skilimui mes gauname ir $\gamma$ spinduoliuotę,
tai yra labai trumpos bangos, kurių dažnis iki $10^{14}$ herco.

\slide{150}. 1 Rezerfordas yra $3,7 \cdot 10^{10}$ $\alpha$ skilimų
per sekundė.
$1mC$ – mili Kiuri.

\slide{151}

\section{$\alpha$-skilimas.}

\begin{remember}
  \item Boro postulatai. Išvardinti juos.
  \item 138 pilnoji atomo energija ir ką reiškia, kai yra 13,5 eV ir
    kai yra 10,15 eV atitinkamų n.
  \item 141 Apie izotopus ir kuo jie yra nusakomi. Reikia žinoti
    kad branduolį sudaro nuklonai neutronai ir protonai ir kad
    to paties branduolio masės gali būti skirtingos dėl to, kad
    yra skirtingas neutronų skaičius.
  \item Prisiminti vandenilio izotopus.
  \item 142 Masės defektas. Ką tai reiškia (fizikinė samprata) ir dėl
    ko atsiranda.
  \item 143 Nuklonų branduolyje ryšio energija.
  \item 144 Branduolių skilimo aprašas. Formulę, kas yra skilimo
    pastovioji, branduolio gyvavimo trukmė.
  \item Žinoti visus skilimus.
  \item 150 Branduolių skilimas veikiant neutronais.
  \item 150 Kokiais vienetais matuojama radiacija.
  \item 151 Kas yra dozė.
  \item 151 Skilimai branduolių veikiant $\beta$ dalelėms.
  \item 152 Branduolių skilimai veikiant juos $\alpha$ dalelėmis.
\end{remember}

\section{$\beta^{-}$-skilimas.}
\section{$\beta^{+}$-skilimas.}
\section{Elektroninis pagavimas.}
\section{$\gamma$-spinduliavimas.}
