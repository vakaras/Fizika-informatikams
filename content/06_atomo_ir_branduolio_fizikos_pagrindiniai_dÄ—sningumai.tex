\chapter{Atomo ir branduolio fizikos pagrindiniai dėsningumai ir jų
sąryšiai.}

% Išdėstyta: 2012-05-09

\section{Branduolio skilimas bei sintezė.}

Medžiagos atomą sudaro branduolys ir apie jį pagal tam tikrą tikimybę
apvalkaluose išsidėstę elektronai. Elektrono orbitos spindulys, tai
atstumas nuo atomo branduolio, kuriame su didžiausia tikimybe galima
tikėtis rasti elektroną. Pagrindinius teorinius pamatus atomo fizikos
moksle yra suformulavęs Nilsas Boras 1913 metais. Pagal jo teoriją
medžiagos atomai energiją sugeria ir išspinduliuoja kvantais. Nilso
Boro idėjos pateikiamos trimis postulatais:
\begin{enumerate}
  \item \strong{Stacionarių būsenų taisyklė.} Visų medžiagų atomai gali
    būti ypatingose stacionariose būsenose ir kiekviena būsena yra
    nusakoma energija
    $\varepsilon_{1}, \varepsilon_{2}, \ldots, \varepsilon_{n}$\footnote{
    Čia energija pažymėta, taip kaip buvo originaliame darbe. Mes
    esame labiau įpratę ją žymėti $E$.}. Šios energetinės būsenos
    yra leistinos. Tarpinės būsenos tarp energijų $\varepsilon_{1}$ ir
    $\varepsilon_{2}$, $\varepsilon_{2}$ ir $\varepsilon_{3}$ ir t.t.
    yra draustinės.
  \item \strong{Orbitų kvantavimo taisyklė.} Elektronai apie branduolius
    juda stacionariomis orbitomis, kurių judėjimo kiekio $mv$ ir
    orbitos ilgio sandauga yra lygi $nh$, kur $n$ yra natūralusis
    skaičius vadinamas kvantiniu skaičiumi. Jei elektronas juda
    $r$ spindulio apskritimine orbita, tai $(2 \pi r) m v = nh$.
  \item \strong{Dažnių taisyklė.} Visų medžiagų atomai išspinduliuoja
    ir sugeria energiją kvantais (fotonais), pereinant jiems iš
    vienos stacionarios būsenos į kitą:
    \begin{equation*}
      h \nu = \varepsilon_{n} - \varepsilon_{m},
    \end{equation*}
    čia $\nu$ yra išspinduliuotos bangos dažnis.
\end{enumerate}

Boro teorija leidžia gana tiksliai nustatyti orbitų spindulius ir
elektronų energiją, kurie juda branduolio elektriniame lauke,
kurio krūvis yra $ze$ (čia $z$ yra atomo eilės numeris, kitaip
tariant – jo protonų skaičius).

Kai $z=1$, tai turime vandenilio ($H$) atomą, o jo pirmosios orbitos
spindulys yra:
\begin{equation*}
  r_{1} = 0,528 \cdot 10^{-10} m.
\end{equation*}
Visų kitų orbitų spindulius galime rasti iš formulės:
\begin{equation*}
  r_{n} = r_{1}\frac{n^{2}}{z^{2}}.
\end{equation*}
Pavyzdžiui, vandenilio $r_{2} = 4r_{1}, r_{3} = 9r_{1}, r_{4} = 16r_{1}$.
Tokiu būdu, elektronas vandenilio atome gali būti trijose orbitose,
kurių spinduliai sudaro santykį lygų natūrinių skaičių eilės
kvadratams. Eelektronas, judėdamas apie branduolį, turi kinetinę
ir potencinę energiją. Pilnoji atomo energija yra lygi:
\begin{equation*}
  E = -E_{0} \frac{z^{2}}{n^{2}},
\end{equation*}
čia $E_{0}$ yra minimali elektrono energija (ją turintis elektronas
yra arčiausiai branduolio). Minuso ženklas formulėje
reiškia, kad norint pašalinti elektroną iš atomo reikia atlikti
darbą prieš branduolio jėgas.

Konkrečiam branduoliui $z$ ir $E_{0}$ yra konstantos, taigi realiai
konkrečiam branduoliui $E$ yra $n$ funkcija. Kaip ir 2-ojo Boro
postulato atveju, taip ir šiuo $n$ yra vadinamas pagrindiniu
kvantiniu skaičiumi. Pavyzdžiui, kai $n=1$, tai vandenilio
atomas turi minimalią energiją ($E_{1} = -13,53 eV$) ir yra normaliojoje
būsenoje. Kitų būsenų energijos yra $E_{2} = -3,38eV$, $E_{3} = -1,51eV$,
$E_{4} = -0,85eV$. Norėdami atomą pervesti iš pirmos stacionarios
būsenos į antrąją, mes jam turime suteikti
$\Delta E = E_{2} - E_{1} = -3,38 - (-13,53) = 10,15 eV$ energiją.
Jeigu bandome suteikti mažesnį kiekį nei $10,15eV$, tai atomas
apskritai nesugeria energijos. Jei atomui suteikiame energiją
$\Delta E > 13,5 eV$, tai elektronas palieka atomą ir išeina iš jo.

Sugėręs energijos kvantą atomas negali amžinai būti sužadintoje
būsenoje. Maždaug po $10^{-8}s$ atomas iš sužadintosios būsenos
grįžta atgal į normaliąją, tuo pačiu išspinduliuodamas elektromagnetinę
bangą nusakytą trečiuoju Boro postulatu:
\begin{align*}
  h \nu &= \Delta E \\
  h \frac{c}{\lambda} &= \Delta E \\
  \lambda &= \frac{hc}{\Delta E}, \\
\end{align*}
čia:
\begin{description}
  \item[$h$] – Planko konstanta;
  \item[$c$] – šviesos greitis vakuume;
  \item[$\Delta E$] – atomo energijos pokytis;
  \item[$\nu$] – išspinduliuotos elektromagnetinės bangos dažnis;
  \item[$\lambda$] – išspinduliuotos elektromagnetinės bangos ilgis.
\end{description}

Sustota \slide{140} ir
\url{http://gedmin.as/study/inf98/physics/silum\%20spind.pdf}
4 psl.

\slide{138}.
Yra vidinis ir išorinis fotoefektai. Už šiuos darbus Enšteinas gavo
Nobelio premiją. (Jis jos negavo už reliatyvumo teoriją.)

\slide{140} $d = 2,8 A^{\frac{1}{3}} \cdot 10 ^{-15}m$

Jei branduolį sudaro atitinkamas protonų skaičius, tai yra daug
atomų kurie turi skirtingas atomines mases. Tokie branduoliai vadinami
izotopais. Vandenilis turi 3 izotopus. O už uraniniai gali turėti
daug izotopų.
Nuklonas – protonas arba neutronas. $M$ – branduolio masė.

\slide{142} Šis reiškinys yra vadinamas masės defektu. Jį paaiškina
Einšteino teorija ir formulė $\Delta E = \Delta m c^{2}$. Kokia
yra fizikinė prasmė kalbant apie šią labai populiarią Einšteino
formulę. Jei turime $E = mc^{2}$, tai reiškia, kad masė yra tapati
energijai. Jei turime $\Delta E = \Delta mc^{2}$ tai reiškia, kad
sintezuojant branduolius dalis masės dingsta dėl elekromagnetinės
spinduliuotės. 

\slide{143} Formulė irgi plaukia iš Boro postulatų, ji yra empirinė.
$\Delta W$ – ryšio energija. Kalbant apie skilimus įvedamos
tokios sąvokos: motininis atomo branduolys, jam skylant atsiranda
dukterinis atomo branduolys ir priklausomai nuo to kokios yra skilimo
reakcijos, skilimo produktai gali būti įvairūs. Yra išskirtos
tokios branduolio skilimo grupės:
\begin{itemize}
  \item $\alpha$, $\beta^{-}$, $\beta^{-}$, $\gamma$ spinduliuotė.
\end{itemize}
Skilimai gali būti tiek natūralūs, tiek dirbtiniai.

\slide{144}
N parodo kiek suskilo branduolių po tam tikro laiko. $N_{0}$ startinis
skylančių branduolių kiekis. Laikas per kurį skyla pusė atomo branduolių
yra vadinamas skilimo pusperiodžiu.

\slide{145}
$\alpha$ skilimas.
$X$ – motininis atomo branduolys. Čia yra du žymėjimai, kai rašoma
iš kairės, tai yra amerikiečių rašymo stilius. Kai rašoma iš dešinės
yra kitų stilius. $Y$ – dukterinis atomo branduolys. Produktas yra
helio atomo branduolys. Kitaip sakant $\alpha$ dalelės yra helio
atomo branduoliai. Pavyzdys: skylant uranui susidaro toras ir
$\alpha$ dalelė. $\alpha$ dalelių masės yra gan didelės, tai toks
skilimo produktas greitai slopsta.

\slide{146}
$\beta^{-}.$ 
$\beta^{-}$ dalelės yra elektronai, arba. 
$X \to Y + e + \~{\nu}$. Produktas yra elektronas ir antineutrinas,
tai yra neutrino anti dalelė.

$\beta^{+}$ pozitroninis skilimas. Kai motininis atomo branduolys
skyla, tai atsiranda pozitronas $e$ ir neutrinas. Neutrinų masė
yra apie $5 eV$, jie sugeba perskrosti Žemės rutulį praktiškai
neprarasdami energijos. Jų masė yra iki $100$ karto mažesnė už
elektrono. Yra galvojama apie informacijos perdavimą panaudojant
neutrinus. Pozitronas yra elektrono antidalelė.
Yra specialus atvejis – elektroninis pagavimas: kai motininis
branduolys pasigauna pozitroną.

Vykstant branduoliniam skilimui mes gauname ir $\gamma$ spinduoliuotę,
tai yra labai trumpos bangos, kurių dažnis iki $10^{14}$ herco.

\slide{150}. 1 Rezerfordas yra $3,7 \cdot 10^{10}$ $\alpha$ skilimų
per sekundė.
$1mC$ – mili Kiuri.

\slide{151}

\section{$\alpha$-skilimas.}

\begin{remember}
  \item Boro postulatai. Išvardinti juos.
  \item 138 pilnoji atomo energija ir ką reiškia, kai yra 13,5 eV ir
    kai yra 10,15 eV atitinkamų n.
  \item 141 Apie izotopus ir kuo jie yra nusakomi. Reikia žinoti
    kad branduolį sudaro nuklonai neutronai ir protonai ir kad
    to paties branduolio masės gali būti skirtingos dėl to, kad
    yra skirtingas neutronų skaičius.
  \item Prisiminti vandenilio izotopus.
  \item 142 Masės defektas. Ką tai reiškia (fizikinė samprata) ir dėl
    ko atsiranda.
  \item 143 Nuklonų branduolyje ryšio energija.
  \item 144 Branduolių skilimo aprašas. Formulę, kas yra skilimo
    pastovioji, branduolio gyvavimo trukmė.
  \item Žinoti visus skilimus.
  \item 150 Branduolių skilimas veikiant neutronais.
  \item 150 Kokiais vienetais matuojama radiacija.
  \item 151 Kas yra dozė.
  \item 151 Skilimai branduolių veikiant $\beta$ dalelėms.
  \item 152 Branduolių skilimai veikiant juos $\alpha$ dalelėmis.
\end{remember}

\section{$\beta^{-}$-skilimas.}
\section{$\beta^{+}$-skilimas.}
\section{Elektroninis pagavimas.}
\section{$\gamma$-spinduliavimas.}
