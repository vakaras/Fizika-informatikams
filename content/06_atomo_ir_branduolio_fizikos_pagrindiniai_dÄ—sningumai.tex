\chapter{Atomo ir branduolio fizikos pagrindiniai dėsningumai ir jų
sąryšiai.}

% Išdėstyta: 2012-05-09

\section{Boro modelis.}

Medžiagos atomą sudaro branduolys ir apie jį pagal tam tikrą tikimybę
apvalkaluose išsidėstę elektronai. Elektrono orbitos spindulys, tai
atstumas nuo atomo branduolio, kuriame su didžiausia tikimybe galima
tikėtis rasti elektroną. Pagrindinius teorinius pamatus atomo fizikos
moksle yra suformulavęs Nilsas Boras 1913 metais. Pagal jo teoriją
medžiagos atomai energiją sugeria ir išspinduliuoja kvantais. Nilso
Boro idėjos pateikiamos trimis postulatais:
\begin{enumerate}
  \item \strong{Stacionarių būsenų taisyklė.} Visų medžiagų atomai gali
    būti ypatingose stacionariose būsenose ir kiekviena būsena yra
    nusakoma energija
    $\varepsilon_{1}, \varepsilon_{2}, \ldots, \varepsilon_{n}$\footnote{
    Čia energija pažymėta, taip kaip buvo originaliame darbe. Mes
    esame labiau įpratę ją žymėti $E$.}. Šios energetinės būsenos
    yra leistinos. Tarpinės būsenos tarp energijų $\varepsilon_{1}$ ir
    $\varepsilon_{2}$, $\varepsilon_{2}$ ir $\varepsilon_{3}$ ir t.t.
    yra draustinės.
  \item \strong{Orbitų kvantavimo taisyklė.} Elektronai apie branduolius
    juda stacionariomis orbitomis, kurių judėjimo kiekio $mv$ ir
    orbitos ilgio sandauga yra lygi $nh$, kur $n$ yra natūralusis
    skaičius vadinamas kvantiniu skaičiumi. Jei elektronas juda
    $r$ spindulio apskritimine orbita, tai $(2 \pi r) m v = nh$.
  \item \strong{Dažnių taisyklė.} Visų medžiagų atomai išspinduliuoja
    ir sugeria energiją kvantais (fotonais), pereinant jiems iš
    vienos stacionarios būsenos į kitą:
    \begin{equation*}
      h \nu = \varepsilon_{n} - \varepsilon_{m},
    \end{equation*}
    čia $\nu$ yra išspinduliuotos bangos dažnis.
\end{enumerate}

Boro teorija leidžia gana tiksliai nustatyti orbitų spindulius ir
elektronų energiją, kurie juda branduolio elektriniame lauke,
kurio krūvis yra $ze$ (čia $z$ yra atomo eilės numeris, kitaip
tariant – jo protonų skaičius).

Kai $z=1$, tai turime vandenilio ($H$) atomą, o jo pirmosios orbitos
spindulys yra:
\begin{equation*}
  r_{1} = 0,528 \cdot 10^{-10} m.
\end{equation*}
Visų kitų orbitų spindulius galime rasti iš formulės:
\begin{equation*}
  r_{n} = r_{1}\frac{n^{2}}{z^{2}}.
\end{equation*}
Pavyzdžiui, vandenilio $r_{2} = 4r_{1}, r_{3} = 9r_{1}, r_{4} = 16r_{1}$.
Tokiu būdu, elektronas vandenilio atome gali būti trijose orbitose,
kurių spinduliai sudaro santykį lygų natūrinių skaičių eilės
kvadratams. Eelektronas, judėdamas apie branduolį, turi kinetinę
ir potencinę energiją. Pilnoji atomo energija yra lygi:
\begin{equation*}
  E = -E_{0} \frac{z^{2}}{n^{2}},
\end{equation*}
čia $E_{0}$ yra minimali elektrono energija (ją turintis elektronas
yra arčiausiai branduolio). Minuso ženklas formulėje
reiškia, kad norint pašalinti elektroną iš atomo reikia atlikti
darbą prieš branduolio jėgas.

Konkrečiam branduoliui $z$ ir $E_{0}$ yra konstantos, taigi realiai
konkrečiam branduoliui $E$ yra $n$ funkcija. Kaip ir 2-ojo Boro
postulato atveju, taip ir šiuo $n$ yra vadinamas pagrindiniu
kvantiniu skaičiumi. Pavyzdžiui, kai $n=1$, tai vandenilio
atomas turi minimalią energiją ($E_{1} = -13,53 eV$) ir yra normaliojoje
būsenoje. Kitų būsenų energijos yra $E_{2} = -3,38eV$, $E_{3} = -1,51eV$,
$E_{4} = -0,85eV$. Norėdami atomą pervesti iš pirmos stacionarios
būsenos į antrąją, mes jam turime suteikti
$\Delta E = E_{2} - E_{1} = -3,38 - (-13,53) = 10,15 eV$ energiją.
Jeigu bandome suteikti mažesnį kiekį nei $10,15eV$, tai atomas
apskritai nesugeria energijos. Jei atomui suteikiame energiją
$\Delta E > 13,5 eV$, tai elektronas palieka atomą ir išeina iš jo.

Sugėręs energijos kvantą atomas negali amžinai būti sužadintoje
būsenoje. Maždaug po $10^{-8}s$ atomas iš sužadintosios būsenos
grįžta atgal į normaliąją, tuo pačiu išspinduliuodamas elektromagnetinę
bangą nusakytą trečiuoju Boro postulatu:
\begin{align*}
  h \nu &= \Delta E \\
  h \frac{c}{\lambda} &= \Delta E \\
  \lambda &= \frac{hc}{\Delta E}, \\
\end{align*}
čia:
\begin{description}
  \item[$h$] – Planko konstanta;
  \item[$c$] – šviesos greitis vakuume;
  \item[$\Delta E$] – atomo energijos pokytis;
  \item[$\nu$] – išspinduliuotos elektromagnetinės bangos dažnis;
  \item[$\lambda$] – išspinduliuotos elektromagnetinės bangos ilgis.
\end{description}
Iš medžiagos išspinduliuojamų medžiagų spektro galima nustatyti medžiagų
atomų rūšį.

\slide{138}.
Yra vidinis ir išorinis fotoefektai. Už šiuos darbus Enšteinas gavo
Nobelio premiją. (Jis jos negavo už reliatyvumo teoriją.)

\begin{remember}
  \item Boro postulatai. Išvardinti juos.
  \item Pilnoji atomo energija ir ką reiškia, kai yra $13,5 eV$ ir
    kai yra $10,15 eV$ atitinkamų $n$.
\end{remember}

\section{Branduolio skilimas bei sintezė.}

Visų atomų branduoliai susideda iš nuklonų (nucleas – branduolys),
kur nuklonas yra protonas arba neutronas. Protonas yra teigiamai
įelektrinta dalelė, kurio krūvis $q = 1,6 \cdot 10^{-19} C$ (lygus
elementariajam krūviui), o masė $m = 1,00728 a.m.v.$ (čia
$1 a.m.v. = 1,66\cdot 10^{-27}kg$). Neutronas neturi krūvio, jo
masė $m = 1,00854 a.m.v.$. Atomo branduolio masė atominiais masės
vienetais ($a.m.v.$) yra vadinama masės skaičiumi ir yra žymima
raide $A$, periodinės cheminių elementų lentelės kairėje pusėje.
Jei protonų skaičių pažymėsime $Z$, o neutronų $N$, tai gausime,
kad $A = Z + N$. $Z$ taip pat yra cheminio elemento eilės numeris
periodinėje lentelėje.

Rezerfordas nustatė, kad atomų branduolių geometriniai matmenys yra
apie $10^{4}$ karto mažesni už atomų. Empiriškai nustatyta, kad
branduolio skersmuo $d = 2,8\sqrt[3]{A} \cdot 10^{-15} m$.

Atomo branduolys, kaip ir elektronų orbitos, nėra griežtai
apibrėžtos. Tiksliai matuojant branduolių mases, paaiškėjo, kad tą
patį cheminio elemento numerį turinčių atomų  branduoliai gali
turėti skirtingas mases. Tai reiškia, kad branduoliai turi tokį patį
kiekį protonų, bet skirtingus skaičius neutronų. Tokie atomai,
kurių branduoliai turi tuos pačius krūvius, bet skirtingas mases yra
vadinami izotopais. Vidutiniškai kiekvienas atomas turi po 3 stabilius
izotopus (užuraniniai daugiau). Dabar yra žinoma apie 300 stabilių ir
800 nestabilių izotopų. To paties atomo izotopai pasižymi tomis pačiomis
cheminėmis savybėmis. Vandenilis, pavyzdžiui, turi tris izotopus:
\begin{equation*}
  \begin{array}[]{c c l}
    {}^{1}_{1}H & \t{arba} & {}^{1}_{1}P(A=1) \t{– protis} \\
    {}^{1}_{2}H & \t{arba} & {}^{1}_{2}D(A=2) \t{– deiteris} \\
    {}^{1}_{3}H & \t{arba} & {}^{1}_{3}T(A=3) \t{– tritis.} \\
  \end{array}
\end{equation*}

Gamtiniai cheminiai junginiai dažniausiai yra izotopų rinkinys
(pavyzdžiui, ${}_{92}^{231}U$ ir ${}_{92}^{238}U$), todėl atominis
masės vienetas nėra sveikas skaičius. Nuklonai branduolyje yra
išsidėstę diskretiniuose energetiniuose lygmenyse, kuriems yra būdingas
griežtas energetinis dydis. Atomo branduoliui pereinant iš vieno
energetinio lygmens į kitą yra išspinduliuojami fotonai (gama
kvantai) arba $\gamma$-spinduliavimas.

Jeigu atomo branduolys turi $Z$ protonų, kurių kiekvieno masė yra
$m_{p}$ ir $N$ neutronų, kurių kiekvieno masė yra $m_{n}$, tai
teorinę branduolio masę $M$ galime apskaičiuoti pagal formulę:
\begin{equation*}
  M = Z \cdot m_{p} + N \cdot m_{n}.
\end{equation*}
Pavyzdžiui, helio atomo ${}_{2}^{4}He$ branduolio masė turėtų būtų lygi:
\begin{equation*}
  M = 2m_{p} + 2m_{n} = 4,03164 a.m.v.
\end{equation*}
Atliekant matavimus buvo nustatyta, kad reali helio atomo branduolio
masė yra $0,0308 a.m.v.$ mažesnė, nei teoriškai apskaičiuota jo vertė.
Šį skirtumą galima paaiškinti pasinaudojant reliatyvumo teorija:
susidarant branduoliui dalis nuklonų kinetinės energijos yra
išspinduliuojama elektromagnetinėmis bangomis:
\begin{equation*}
  \Delta E = \Delta m c^{2},
\end{equation*}
čia:
\begin{description}
  \item[$\Delta E$] – išspinduliuota energija;
  \item[$\Delta m$] – masės defektas.
\end{description}
\begin{note}
  Kokia yra labai populiarios Einšteino formulės $E = mc^{2}$
  fizikinė prasmė? Jei turime $E = mc^{2}$, tai reiškia, kad masė
  yra tapati energijai. Jei turime $\Delta E = \Delta mc^{2}$ tai
  reiškia, kad sintezuojant branduolius dalis masės dingsta dėl
  elekromagnetinės spinduliuotės.
\end{note}

Nuklonai branduolyje yra surišti branduolinėmis jėgomis. Energijos
kiekis, kuri yra reikalingas tam, kad išskirti nuklonus, nesuteikiant
jiems kinetinės energijos, yra vadinamas branduolio ryšio energija.
Branduolinės energijos yra pačios didžiausios energijos lyg šiol
žinomos gamtoje. Analogiškų dydžių yra branduolių sintezės energijos.

Vienų atomų branduoliai skyla savaime, kitų – būna stabilūs. Apie
branduolių stabilumą galima spręsti iš jų savitosios nuklonų ryšio
energijos:
\begin{equation*}
  \Delta W = \frac{931,8 \Delta m}{Z + N},
\end{equation*}
čia $\Delta W$ yra nuklonų ryšio energija. Ši formulė irgi plaukia iš
Boro postulatų. Ji gauta empiriniu būdu.

Kadangi visos sintezės reakcijos vyksta aukštoje temperatūroje, jos
vadinamos termobranduolinėmis reakcijomis. Branduolių skilimo bei
sintezės procesai naudojami atominių bei termobranduolinių bombų
gamyboje.

Vykstant branduolių skilimui skylantys cheminiai elementai virsta kitais.
Atomo branduolys, kuris suskyla yra vadinamas motininiu, o naujasis
branduolys yra vadinamas dukteriniu. Kaip skilimo produktas pasireiškia
radioaktyvumas, kuris gali būti nusakomas:
\begin{enumerate}
  \item $\alpha$-skilimu;
  \item $\beta$-skilimu;
  \item $\gamma$-spinduliavimu;
  \item savaiminiu branduolių skilimu;
  \item protoniniu radioaktyvumu.
\end{enumerate}
Jeigu radioaktyvumas pasireiškia dėl savaiminio skilimo, jis vadinamas
natūraliu radioaktyvumu, o jeigu dirbtinai sukuriamas – dirbtiniu.
Branduolių skilimą nusakantis skirstinys yra eksponentinis:
\begin{equation*}
  N = N_{0} e^{-\lambda t},
\end{equation*}
šiuo atveju parametras $\lambda$ yra vadinamas skilimo pastoviąja.
$N_{0}$ – pradinis nesuskilusių atomų skaičius, $N$ – nesuskilusių
atomų skaičius laiko momentu $t$.

Laikas, per kurį skyla pusė medžiagoje esančių branduolių yra vadinamas
skilimo pusperiodžiu. Jį galime išsireikšti iš anksčiau paminėtos
lygties:
\begin{align*}
  \frac{N_{0}}{2} &= N_{0} e^{-\lambda T} \\
  \frac{1}{2} &= e^{-\lambda T} \\
  \ln \left( \frac{1}{2} \right) &= - \lambda T \\
  \lambda T &= \ln 2 - \underbrace{\ln 1}_{=0} \\
  T &= \frac{\ln 2}{\lambda} \\
\end{align*}

Iš tikimybių teorijos žinome, kad eksponentinio skirstinio vidurkis
yra lygus $\frac{1}{\lambda}$, taigi vidutinė radioktyviojo
branduolio gyvavimo trukmė yra:
\begin{equation*}
  \tau = \frac{1}{\lambda}.
\end{equation*}

\begin{remember}
  \item Apie izotopus ir kuo jie yra nusakomi. Reikia žinoti
    kad branduolį sudaro nuklonai neutronai ir protonai ir kad
    to paties branduolio masės gali būti skirtingos dėl to, kad
    yra skirtingas neutronų skaičius.
  \item Prisiminti vandenilio izotopus.
  \item Masės defektas. Ką tai reiškia (fizikinė samprata) ir dėl
    ko atsiranda.
  \item Nuklonų branduolyje ryšio energija.
  \item Branduolių skilimo aprašas. Formulę, kas yra skilimo
    pastovioji, branduolio gyvavimo trukmė.
\end{remember}

\section{$\alpha$-skilimas.}

\slide{145}

$\alpha$-skilimo metu atsiranda $\alpha$-spinduliai. $\alpha$-spinduliai
tai yra helio atomai, kurie reakcijos metu yra išspinduliuojami
vidutiniu $10^{7} \frac{m}{s}$ greičiu. Kadangi $\alpha$ dalelių masės
yra gana didelės, tai jų greitis greitai slopsta.

$\alpha$-skilimo reakcija:
\begin{equation*}
  {}_{Z}^{A}X \to {}_{Z-2}^{A-4}Y + {}_{2}^{4}He.
\end{equation*}
\begin{note}
  Yra du žymėjimai: kai rašoma iš kairės, tai yra amerikiečių
  rašymo stilius, bet taip pat yra galima skaičius nurodyti ir
  dešinėje.
\end{note}

$\alpha$-skilimo reakcijos pavyzdys, kai skylant uranui susidaro toras
ir $\alpha$ dalelė:
\begin{equation*}
  {}_{92}^{238}U \to {}_{90}^{234}Th + {}_{2}^{4}He.
\end{equation*}

\section{$\beta^{-}$-skilimas.}

$\beta^{-}$-skilimo reakcija:
\begin{equation*}
  {}_{Z}^{A}X \to {}_{Z+1}^{A}Y + {}_{-1}e + \tilde{\nu}.
\end{equation*}

Vykstant $\beta^{-}$ skilimo reakcijai vienas iš neutronų skyla
į elektroną ir protoną. Todėl dukterinis branduolys turi vienetu
aukštesnį eilės numerį nei motininis branduolys, nors jų masės
skaičiai ir yra vienodi. Be elektrono ${}_{-1}e$ yra išspinduliuojamas
ir antineutrinas $\tilde{\nu}$ (neutrino antidalelė).

$\beta^{-}$-skilimo reakcijos pavyzdys, kai skylant torui susidaro
praktinis:
\begin{equation*}
  {}_{90}^{234}Th \to {}_{91}^{234}Pa + {}_{-1}e + \tilde{\nu}.
\end{equation*}

\section{$\beta^{+}$-skilimas.}

$\beta^{+}$-skilimas dar yra vadinamas pozitroniniu skilimu. Kai
motininis atomo branduolys skyla, tai atsiranda pozitronas
${}_{+1}^{0}e$ (elektrono antidalelė) ir neutrinas $\nu$. Neutrinų masė
yra apie $5 eV$, jie sugeba perskrosti Žemės rutulį praktiškai
neprarasdami energijos. Jų masė yra iki $100$ karto mažesnė už
elektrono. Yra galvojama apie informacijos perdavimą panaudojant
neutrinus.

$\beta^{+}$-skilimo reakcija:
\begin{equation*}
  {}_{Z}^{A}X \to {}_{Z-1}^{A}Y + {}_{+1}^{0} + \nu.
\end{equation*}

$\beta^{+}$-skilimo reakcijos pavyzdys:
\begin{equation*}
  {}_{7}^{13}N \to {}_{6}^{13}C + {}_{+1}^{0} + \nu.
\end{equation*}

\section{Elektroninis pagavimas.}

Yra specialus atvejis – elektroninis pagavimas: kai motininis
branduolys pasigauna pozitroną.

Elektroninio pagavimo reakcija:
\begin{equation*}
  {}_{Z}^{A}X + {}_{+1}^{0}e \to {}_{Z-1}^{A}Y + \nu.
\end{equation*}

Elektroninio pagavimo reakcijos pavyzdys:
\begin{equation*}
  {}_{19}^{40}K + {}_{+1}^{0}e \to {}_{18}^{40}Ar + \nu.
\end{equation*}

\section{$\gamma$-spinduliavimas.}

Vykstant branduoliniam skilimui mes gauname ir $\gamma$ spinduoliuotę,
tai yra labai trumpos bangos, kurių dažnis iki $10^{14}$ herco.

\section{Kažkas}

Savaiminis branduolių skilimas buvo stebimas apie 1940 metais, kai
radioaktyviųjų elementų branduoliai skyla per pusę bombarduojant juos
neutronais ($n$), pavyzdžiui:
\begin{equation*}
  {}_{90}^{234}Th + n \to {}_{90}^{234}Th^{T=22 \t{minutės}}
  \to {}_{91}^{232}Ac^{T=27\t{paros}} \to {}_{92}^{233}U.
\end{equation*}
Ši reakcija yra grandininė. ${}_{90}^{233}U$ yra izotopas, kurio nėra
gamtoje ir jam yra būdingas $\alpha$-skilimas.

1963 metais pirmą kartą pastebėtas branduolių skilimas, kurio metu
atsiranda vienas arba du protonai. Radioaktyvių medžiagų aktyvumas
yra matuojamas Kiuri $[C]$, tai yra $3,7 \cdot 10^{10}$ $\alpha$ 
skilimų per sekundę. Kitas matavimo vienetas yra Rezerfordas
$[Rd]$, kuris yra $10^{6}$ $\alpha$ skilimų per sekundę.

Spinduliuotės dalis, kurią sugeria $m$ masės kūnas yra vadinamas
doze:
\begin{equation*}
  D = \frac{E}{m} \left[ \frac{J}{kg} \right].
\end{equation*}

Pavyzdys skilimų grandinės, kai medžiaga yra veikiama $\beta$
dalelėmis:
\begin{equation*}
  {}_{54}^{139}Xe \underset{\beta}{\to}
  {}_{55}^{139}Cs \underset{\beta}{\to}
  {}_{56}^{139}Ba \underset{\beta}{\to}
  {}_{57}^{139}La
\end{equation*}

Pavyzdys skilimų, kai medžiaga yra veikiama $\alpha$ dalelėmis:
\begin{align*}
  {}_{92}^{238}U + {}_{2}^{4}He &\to {}_{94}^{241} + {}_{0}^{1}n \\
  {}_{94}^{239}Pu + {}_{2}^{4}He &\to {}_{96}^{242}Cm + {}_{0}^{1}n \\
\end{align*}

\begin{remember}
  \item Branduolių skilimas veikiant neutronais.
  \item Kokiais vienetais matuojama radiacija.
  \item Kas yra dozė.
  \item Skilimai branduolių veikiant $\beta$ dalelėms.
  \item Branduolių skilimai veikiant juos $\alpha$ dalelėmis.
\end{remember}
